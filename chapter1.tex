%\setcounter{chapter}{1}
\chapter{Preliminaries}
%Paragraph A----------------------------------------------------------------------------------
\paragraph{A Structures}
\begin{enumerate}[1.]
%
\item \header{Cycles Have Length at Least $2$}
%
\item \header{An Example of $\strct{A} \ordsum \strct{B} \not\isom \strct{B} \ordsum \strct{A}$} Let $\tau \defas \sete{{<}, P}$ in which $P$ is unary, $\strct{A} = (\sete{a}, \emptyset, \emptyset), \strct{B} = (\sete{b}, \emptyset, \sete{b})$.
%
\item \header{Note on the Last Paragraph of A3 Operations on Structures} In fact, $\strct{A} \dsjuni \strct{B}$ and $\strct{B} \dsjuni \strct{A}$ are not only isomorphic but also \emph{identical} since $A \union B = B \union A$ and $\intpr{R}{\strct{A}} \union \intpr{R}{\strct{B}} = \intpr{R}{\strct{B}} \union \intpr{R}{\strct{A}}$ for $R \in \tau$.
%
\end{enumerate}
%End of Paragraph A---------------------------------------------------------------------------
%Paragraph C----------------------------------------------------------------------------------
\paragraph{C Some Classical Results of First-Order Logic}
\begin{enumerate}[1.]
%
\item \header{Note on the Proof of Lemma 1.0.6} It is easy to drop the assumption that $\varphi$ is a sentence: All variables occurring in $\varphi$, if any, can be replaced by new constant symbols, i.e.\ constants not in $\tau$.
%
\end{enumerate}
%End of Paragraph C---------------------------------------------------------------------------
%Paragraph D----------------------------------------------------------------------------------
\paragraph{D Model Classes and Global Relations}
\begin{enumerate}[1.]
%
\item \header{Note on Sets of Tuples Defined by Formulas} One can infer by context that in the definition of $\tuplesby{\varphi}{\strct{A}}$ in which $\varphi = \varphi(x_1, \etc, x_n)$, the free variables in $\varphi$ are among $x_1, \etc, x_n$.
%
\end{enumerate}
%End of Paragraph D-----------------------------------------------------------------------------