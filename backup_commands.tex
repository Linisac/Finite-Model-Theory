\newcommand{\setenum}[1]{\{#1\}} %the set of which the elements are enumerated by #1
\newcommand{\setsum}{\cup} %`A \setsum B' stands for the union of sets A and B
\newcommand{\setprod}{\cap} %`A \setprod B' stands for the intersection of sets A and B
\newcommand{\bsetsum}{\bigcup} %big operator of union
\newcommand{\bsetprod}{\bigcap} %big operator of intersection
\newcommand{\nat}{\mathbb{N}} %the set of natural numbers
\newcommand{\zah}{\mathbb{Z}} %the set of integers
\newcommand{\rat}{\mathbb{Q}} %the set of rational numbers
\newcommand{\real}{\mathbb{R}} %the set of real numbers
\newcommand{\cplx}{\mathbb{C}} %the set of complex numbers
\newcommand{\powerset}[1]{{\mathcal{P}\left(#1\right)}} %the power set of #1
\newcommand{\absval}[1]{{\left| #1 \right|}} %the absolute value of #1
\newcommand{\card}[1]{\absval{#1}} %the cardinality of the set #1
\newcommand{\limply}{\mathbin{\rightarrow}} %logical connective imply
\newcommand{\liff}{\mathbin{\leftrightarrow}} %logical connective if and only if
\newcommand{\blor}{\bigvee} %big operator or
\newcommand{\bland}{\bigwedge} %big operator and
\newcommand{\Iff}{\mbox{iff}} %!!
\newcommand{\scripttext}[1]{{\mbox{\scriptsize#1}}} %text of script size in math mode
\newcommand{\mbf}[1]{{\mbox{\boldmath\begin{math}#1\end{math}}}} %bold face in math mode
\newcommand{\mbfs}[1]{{\mbox{\scriptsize\boldmath\begin{math}\mathrm{#1}\end{math}}}} %bold face of script size in math mode
\newcommand{\mbff}[1]{{\mbox{\footnotesize\boldmath\begin{math}\mathrm{#1}\end{math}}}} %bold face of footnote size in math mode
\newcommand{\mbft}[1]{{\mbox{\tiny\boldmath\begin{math}\mathrm{#1}\end{math}}}} %bold face of tiny size in math mode
\newcommand{\formal}[1]{\bm{#1}} %formal version of the symbol #1
\newcommand{\reftitle}[1]{{\rm #1}} %reference title
\newcommand{\parenadj}[1]{\left(#1\right)} %adjustable parentheses
\newcommand{\brackadj}[1]{\left[#1\right]} %adjustable (square) brackets
\newcommand{\braceadj}[1]{\left\lbrace #1\right\rbrace} %adjustable (curly) braces
\newcommand{\angleadj}[1]{\left\langle #1 \right\rangle} %adjustable angular brackets
\newcommand{\opintv}[2]{\parenadj{#1, #2}} %the open interval (#1, #2)
\newcommand{\clintv}[2]{\brackadj{#1, #2}} %the closed interval [#1, #2]
\newcommand{\olintv}[2]{\left(#1, #2\right]} %the left-open right-closed interval (#1, #2]
\newcommand{\cpintv}[2]{\left[#1, #2\right)} %the left-closed right-open interval [#1, #2)
\newcommand{\pair}[2]{(#1, #2)} %the pair of #1 and #2
\newcommand{\tuple}[1]{(#1)} %tuple of #1; often used with \seq
\newcommand{\pairadj}[2]{\parenadj{#1, #2}} %the pair of #1 and #2 with adjustable parentheses
\newcommand{\tupleadj}[1]{\parenadj{#1}} %tuple of #1 with adjustable parentheses; often used with \seq
\newcommand{\seq}[3][0]{{#2}_{#1}, \ldots, {#2}_{#3}} %sequence; \seq[1]{a}{n} means the sequence a_1, \ldots, a_n
%\newcommand{\seqv}[2]{#1, \ldots, #2} %already abolished
\newcommand{\seqp}[2]{{#1}, \ldots, {#2}} %sequence with additional parameters; \seqv{a_1}{a_n} means the sequence a_1, \ldots, a_n
\newcommand{\seqi}[2]{(#1)_{#2}} %sequence #1 indexed by the set #2
\newcommand{\enum}[3][0]{{{#2}_{#1} \ldots {#2}_{#3}}} %enumeration; \enum[1]{t}{n} means the enumeration t_1 \ldots t_n
%\newcommand{\enumv}[2]{#1 \ldots #2} %already abolished
\newcommand{\enump}[2]{{{#1} \ldots {#2}}} %enumeration with parameter(s); \enump{a_1}{a_n} means the enumeration a_ \ldots a_n
\newcommand{\enumpop}[3]{#1 #2 \ldots #2 #3} %enumeration with operators; \enumpop{a}{op}{b} means repeated applications of the operator op to the sequence a, \ldots, b. For example, \enumpop{a}{+}{b} is a + ... + b.
%\newcommand{\enumprel}[3]{#1 #2 \cdots #2 #3} %??
\newcommand{\setm}[2]{\{ #1 \mid #2 \}} %set with description in math mode; \setm{a}{b} means the set \{ a \mid b \}
\newcommand{\sett}[2]{\{ #1 \mid \mbox{#2} \}} %set with description in text mode; \sett{a}{b} means the set \{ a \mid b \} with b a text
\newcommand{\restrict}[2]{{#1|_{#2}}} %the restriction of the function #1 to the domain #2
%%Chapter 2
\newcommand{\alphabet}{\mathcal{A}} %the general alphabet
\newcommand{\kleene}[1]{{#1}^\ast} %Kleene star operator
\newcommand{\nullstring}{\Box} %null, or empty, string
%\newcommand{\symbolset}{\mathit{S}} %a symbol set
\newcommand{\equal}{\equiv} %equal sign in object languages
\newcommand{\absymb}[1]{\alphabet_{#1}} %the alphabet for the first-order language determined by the symbol set #1 (see Section II.2)
\newcommand{\gr}{\mathrm{gr}} %an abbreviation for group theory
\newcommand{\eqr}{\mathrm{eq}} %an abbreviation for equivalence relation theory
\newcommand{\univsymb}{S_\infty} %the universal symbol set (see Section II.2)
\newcommand{\termbase}{\mathit{T}} %the base symbol for sets of terms
\newcommand{\term}[2][]{{\termbase_{#1}^{#2}}} %the set of terms
\newcommand{\languagebase}{\mathit{L}} %the base symbol for a (especially first-order) language
\newcommand{\fstordlang}[2][]{{\languagebase_{#1}^{#2}}} %first-order language
\newcommand{\fstordfmla}[2][0]{{\languagebase_{#1}^{#2}}} %!!
\newcommand{\calrule}[2]{{\displaystyle\frac{\;#1\;}{\;#2\;}}} %calculus rule in schema form
%%Chapter 3
\newcommand{\mul}{\cdot} %the multiplication operator used in arithemtics
\newcommand{\ar}{\mathrm{ar}} %the abbreviation for arithmetics
\newcommand{\arsymb}{S_\ar} %the symbol set for arithmetics

\newcommand{\grp}{\mathrm{grp}} %an abbreviation for group theory
\newcommand{\natstr}{\mathfrak{N}} %the structure (\nat, +, \cdot, 0, 1)
\newcommand{\zahstr}{\mathfrak{Z}} %the structure (\zah, +, \cdot, 0, 1)
\newcommand{\realstr}{\mathfrak{R}} %the structure (\real, +, \cdot, 0, 1)
\newcommand{\negfunc}{\mathord{\overset{.}{\neg}}} %the negation function in the metalanguage
\newcommand{\dsjfunc}{\mathbin{\overset{.}{\lor}}} %the disjunction function in the metalanguage
\newcommand{\cnjfunc}{\mathbin{\overset{.}{\land}}} %the conjunction function in the metalanguage
\newcommand{\impfunc}{\mathbin{\overset{.}{\rightarrow}}} %the implication function in the metalanguage
\newcommand{\bimfunc}{\mathbin{\overset{.}{\leftrightarrow}}} %the bi-implication function in the metalanguage
\newcommand{\sat}{\mathrel{\mathrm{Sat}}} %the relation satisfiable
\newcommand{\modeled}{\ \mathrm{\reflectbox{$\models$}}} %!!
\newcommand{\bimodels}{\mathrel{\mathord{\reflectbox{$\models$}}\mathord{\models}}} %to be \logequiv
\newcommand{\logequiv}{\mathrel{\mathord{\reflectbox{$\models$}}\mathord{\models}}} %the relation logically equivalent
%\newcommand{\STR}[1]{\mathfrak{#1}} %already abolished
\newcommand{\struct}[1]{\mathfrak{#1}} %the structure of #1
\newcommand{\intpted}[2]{{#1}^{#2}} %function, constant or relation symbol #1 interpreted as such in the domain of structure #2
\newcommand{\reduct}[2]{{#1|_{#2}}} %the #2-reduct of (the structure) #1
\newcommand{\assgn}{\beta} %a (first-order) assignment
\newcommand{\INT}{\mathfrak{I}} %to be \intp
\newcommand{\intp}{\mathfrak{I}} %an interpretation
\newcommand{\INTP}[2]{(#1, #2)} %to be \intparg
\newcommand{\intpp}[2]{(#1, #2)} %to be \intparg
\newcommand{\intparg}[2]{\pair{#1}{#2}} %the interpretation of (#1, #2), where the arguements #1 and #2 are a structure and an assignment in #1, respectively
\newcommand{\substr}[2]{{[#1]^{#2}}} %substructure generated by #1 in #2
\newcommand{\iso}{\cong} %isomorphic; DEPENDENCY: \finiso, \partiso
\newcommand{\ord}{\mathrm{ord}} %abbreviation for ordering
\newcommand{\field}[1]{\mathop{\mathrm{field}} \intpted{<}{#1}} %the field of < in the structure #1
\newcommand{\pord}{\mathrm{pord}} %abbreviation for partially defined ordering
\newcommand{\suc}{\sigma} %the successor function over \nat
\newcommand{\natsuc}{{\natstr_\suc}} %the structure (\nat, successor, 0)
\newcommand{\sndordpeanoarith}{\Pi} %second-order Peano arithmetics
\newcommand{\df}[2]{\displaystyle\frac{#1}{#2}} %fraction in displaystyle
\newcommand{\varbase}{\mathrm{var}} %the base symbol for the function var; DEPENDENCY: \var
\newcommand{\var}[1]{{\varbase(#1)}} %the set of variables occurring in the term #1
\newcommand{\freebase}{\mathrm{free}} %the base symbol for the function free; DEPENDENCY: \free
\newcommand{\free}[1]{{\freebase(#1)}} %the set of free variables occurring in the formula #1
\newcommand{\SF}[1]{\mathop{\mathrm{SF}}(#1)} %to be \sbfmla
\newcommand{\sbfmlabase}{\mathrm{SF}} %the base symbol for the function sbfmla; DEPENDENCY: \sbfmla
\newcommand{\sbfmla}[1]{{\sbfmlabase(#1)}} %the set of subformulas of the formula #1
\newcommand{\sbst}[2]{{\scriptstyle\frac{\displaystyle #1}{\displaystyle #2}}} %the substitution operation
\newcommand{\scriptsbst}[2]{{\scriptscriptstyle\frac{\scriptstyle #1}{\scriptstyle #2}}} %the substitution used in script style
\newcommand{\exactly}[1]{\exists^{\mathrel{=} #1}} %there are exactly #1 element(s) such that
\newcommand{\exuni}{\exists^{=1}} %there is a unique element such that
\newcommand{\atmost}[1]{\exists^{\mathrel{\leq} #1}} %there are at most #1 element(s) such that
\newcommand{\atleast}[1]{\exists^{\mathrel{\geq} #1}} %there are at least #1 element(s) such that
%%Chapter 4
\newcommand{\derives}{\vdash} %the relation formally derives
\newcommand{\derive}{\ | \hspace{-.4em} -} %an alternative symbol for \vdash
\newcommand{\assm}{{(\mathrm{Assm})}} %the sequent rule (Assm)
\newcommand{\ant}{{(\mathrm{Ant})}} %the sequent rule (Ant)
\newcommand{\pc}{{(\mathrm{PC})}} %the sequent rule (PC)
\newcommand{\ctr}{{(\mathrm{Ctr})}} %the sequent rule (Ctr)
\newcommand{\ora}{{(\lor\mathrm{A})}} %the sequent rule (\lor A)
\newcommand{\ors}{{(\lor\mathrm{S})}} %the sequent rule (\lor S)
\newcommand{\ea}{{(\exists\mathrm{A})}} %the sequent rule (\exists A)
\newcommand{\es}{{(\exists\mathrm{S})}} %the sequent rule (\exists S)
\newcommand{\eq}{{(\equiv)}} %the sequent rule (\equal)
\newcommand{\sub}{{(\mathrm{Sub})}} %the sequent rule (Sub)
\newcommand{\seqcal}{{\mathfrak{S}}} %the sequent calculus \mathfrak{S}
\newcommand{\con}{\mathrel{\mathrm{Con}}} %the relation consistent
\newcommand{\inc}{\mathrel{\mathrm{Inc}}} %the relation inconsistent
%%Chapter 6
\newcommand{\modelclassbase}{\mathrm{Mod}} %the base symbol for \modelclass and \modelclassarg; DEPENDENCY: \modelclass, \modelclassarg
\newcommand{\modelclass}[2]{{\mathop{\modelclassbase^{#1}} #2}} %the set of #1-structures that are models of the #1-formula #2
\newcommand{\thr}[1]{\mathop{\mathrm{Th}}(#1)} %to be \theoarg
\newcommand{\Th}{\mathop{\mathrm{Th}}} %!!
\newcommand{\theorybase}{\mathrm{Th}} %the base symbol for a theory
\newcommand{\theoarg}[1]{{\theorybase(#1)}} %the theory of #1, which takes argument(s), such as Th(\mathfrak{A})
%%Chapter 7
\newcommand{\zfc}{\mathrm{ZFC}} %the abbreviation for Zermelo-Fraenkel set theory with the axiom of choice
%%Chapter 8
\newcommand{\relational}[1]{#1^\mathit{r}} %the relational symbol set corresponding to #1
\newcommand{\invrelational}[1]{#1^{-\mathit{r}}} %the inverse operation of \relational to #1
\newcommand{\relativize}[2]{#1^{#2}} %the relativization of #1 to #2
%%Chapter 9
\newcommand{\logicalsystembase}{\mathcal{L}} %the base symbol for logical systems
\newcommand{\firstorder}{{\mathrm{I}}} %the abbreviation for first-order
\newcommand{\secondorder}{{\mathrm{II}}} %the abbreviation for second-order
\newcommand{\sndordassgn}{\gamma} %a second-order assignment
\newcommand{\freeII}{\free_\mathrm{II}} %to be replaced by \sndordfree
\newcommand{\sndordfree}[1]{{\freebase_\secondorder(#1)}} %the set of free relation variables occurring in the formula #1
\newcommand{\FOL}{\mathcal{L}_\mathrm{I}} %to be \fstordlog
\newcommand{\fstordlog}{{\logicalsystembase_\firstorder}} %the logical system first-order logic
\newcommand{\SOL}{\mathcal{L}_\mathrm{II}} %to be \sndordlog
\newcommand{\sndordlog}{{\logicalsystembase_\secondorder}} %the logical system second-order logic
\newcommand{\LII}{L_\mathrm{II}} %to be \sndordlang
\newcommand{\sndordlang}[1]{{\languagebase_\secondorder^{#1}}} %the second-order language with symbol set #1
\newcommand{\weak}{{\mathit{w}}}
\newcommand{\weaksndordlog}{{\logicalsystembase^\weak_\secondorder}} %weak second-order logic
\newcommand{\weaksndordlang}[1]{{\languagebase^{\weak, #1}_\secondorder}} %the weak second-order language with symbol set #1
\newcommand{\INFL}{\mathcal{L}_{\omega_1\omega}} %to be \infinlog
\newcommand{\infinlog}{{\logicalsystembase_{\omega_1\omega}}} %the logical system infinitary logic \omega_1\omega
\newcommand{\LINF}{L_{\omega_1\omega}} %to be replaced by \infinlang
\newcommand{\infin}{{\omega_1\omega}} %the modifier infinitary \omega_1 \omega
\newcommand{\infinlang}[1]{{\languagebase_\infin^{#1}}} %the infinitary language \omega_1\omega with symbol set #1
\newcommand{\qexist}{\mathit{Q}} %the Q quantifier in Q system
\newcommand{\QL}{\mathcal{L}_Q} %to be replaced by \qlog
\newcommand{\qlog}{{\logicalsystembase_\qexist}} %the logical system Q logic
\newcommand{\LQ}{L_Q} %to be replaced by \qlang
\newcommand{\qlang}[1]{{\languagebase_\qexist^{#1}}} %the Q language with symbol set #1
\newcommand{\varqlog}{{\logicalsystembase^\circ_\qexist}} %the variant of Q logic with Q quantifier interpreted as ``there are infinitarily many''
\newcommand{\domain}[1]{\mbox{the domain of } #1} %!!
\newcommand{\dist}{\mathrm{dist}} %!!
\newcommand{\nme}{\mathrm{NME}} %!!
\newcommand{\indexed}{\mathrm{index}} %!!
\newcommand{\fld}{\mathrm{field}} %!!
\newcommand{\abs}{\mathrm{abs}} %!!
%%Chapter 10
\newcommand{\procp}[1]{\mathfrak{#1}} %procedure that takes an argument
\newcommand{\proc}{\procp{P}} %a common procedure symbol
\newcommand{\R}{\mathrm{R}} %to be \REG
\newcommand{\REG}[1]{\mathrm{R}_{#1}} %register
\newcommand{\LET}{\mathrm{LET}}
\newcommand{\IF}{\mathrm{IF}}
\newcommand{\THEN}{\mathrm{THEN}}
\newcommand{\ELSE}{\mathrm{ELSE}}
\newcommand{\OR}{\mathrm{OR}}
\newcommand{\PRINT}{\mathrm{PRINT}}
\newcommand{\HALT}{\mathrm{HALT}}
\newcommand{\GOTO}{\mathrm{GOTO}}
\newcommand{\p}{\mathrm{P}} %to be \prog
\newcommand{\prog}{\mathrm{P}} %a (register) program
\newcommand{\halt}{\mathrm{halt}} %the abbreviation for halt
\newcommand{\length}{\mathit{l}}
\newcommand{\PA}[2]{\LET \ \R_{#1} = \R_{#1} + #2}
\newcommand{\PS}[2]{\LET \ \R_{#1} = \R_{#1} - #2}
\newcommand{\PI}[4]{\IF \ \R_{#1} = \Box \ \THEN \ #2 \ \ELSE \ #3 \ldots \ \OR \ #4}
\newcommand{\PII}[5]{\IF \ \R_{#1} = \Box \ \THEN \ #2 \ \ELSE \ #3 \ \OR \ldots \ #4 \ldots \ \OR \ #5}
\newcommand{\consqn}[1]{#1^{\models}} %consequence closure
\newcommand{\regdec}{R-decidable} %register-decidable
\newcommand{\regund}{R-undecidable} %register-undecidable
\newcommand{\regenum}{R-enumerable} %register-enumerable
\newcommand{\regcomp}{R-computable} %register-computable
\newcommand{\regaxm}{R-axiomatizable} %register-axiomatizable
\newcommand{\finsat}{fin-satisfiable} %satisfiable by a finite structure
\newcommand{\finval}{fin-valid} %satisfied by every finite structure
\newcommand{\theosub}[1]{\mathop{\mathrm{Th}}(#1)} %!!
\newcommand{\pa}{{\mathrm{PA}}} %the abbreviation of Peano
\newcommand{\peanotheory}{\theorybase_\pa} %the (first-order) Peano theory
\newcommand{\zfctheory}{\theorybase_\zfc} %the ZFC theory
\newcommand{\are}{{\ar^\prime}} %extended arithmetics
\newcommand{\goedel}[1]{n^{#1}} %the Goedel number of #1

\newcommand{\Der}[1]{\mathrm{Der}_{#1}}
\newcommand{\atm}{\mathrm{atm}}
\newcommand{\ngt}{\mathrm{ngt}}
\newcommand{\dsj}{\mathrm{dsj}}
\newcommand{\ext}{\mathrm{ext}}
\newcommand{\sbt}{\mathrm{sbt}}
\newcommand{\sbf}{\mathrm{sbf}}
\newcommand{\drn}{\mathrm{drn}}
\newcommand{\consis}[1]{\mathrm{Consis}_{#1}}
\newcommand{\der}[1]{\mathrm{der}(\underline{n^{#1}})}
\newcommand{\fvar}[1]{\mathop{\mathrm{fvar}(#1)}}
\newcommand{\rpl}{\mathop{\mathrm{rpl}}}
\newcommand{\sft}{\mathop{\mathrm{sft}}}
%%Chapter 11
\newcommand{\vect}[2]{{\overset{#2}{#1}}} %vector, \vect{a}{n} stands for the sequence a_1, \ldots, a_n
\newcommand{\pvarbase}{\mathrm{pvar}} %the base symbol for the function pvar; DEPENDENCY: \pvar
\newcommand{\pvar}[1]{{\pvarbase(#1)}} %the set of propositional variables occurring in the (propositional) formula #1
\newcommand{\pf}{\mathit{PF}} %to be \propfmla
\newcommand{\propfmla}{\mathit{PF}} %the set of propositional formulas
\newcommand{\clauses}{\mathfrak{K}} %a set of clauses
\newcommand{\pclauses}{\mathfrak{P}} %a set of positive clauses
\newcommand{\scls}[1]{{\mathfrak{K}(#1)}} %to be \setofclauses
\newcommand{\setofclauses}[1]{{\mathfrak{K}(#1)}} %the set of clauses associated to #1
\newcommand{\resolutionbase}{\mathrm{Res}} %the base symbol for the function Res; DEPENDENCY: \res, \resi
\newcommand{\res}[1]{{\resolutionbase(#1)}} %the resolution of #1
\newcommand{\resi}[2]{{\resolutionbase_{#1}(#2)}} %the resolution of #2 in at most #1 steps
\newcommand{\hornresolutionbase}{\mathrm{HRes}} %the base symbol for the function HRes; DEPENDENCY: \hres, \hresi
\newcommand{\hres}[1]{{\hornresolutionbase(#1)}} %the Horn-resolution of #1
\newcommand{\hresi}[2]{{\hornresolutionbase_{#1}(#2)}} %the Horn-resolution of #2 in at most #1 steps
\newcommand{\grndinstbase}{\mathrm{GI}} %the base symbol for the function GI; DEPENDENCY: \gi
\newcommand{\gi}[1]{{\grndinstbase(#1)}} %the set of ground instances of #1
\newcommand{\unifiedresolutionbase}{\mathrm{URes}} %the base symbol for the function URes; DEPENDENCY: \ures, \uresi
\newcommand{\ures}[1]{{\unifiedresolutionbase(#1)}} %the set of unified resolution of #1
\newcommand{\uresi}[2]{{\unifiedresolutionbase_{#1}(#2)}} %the set of unified resolution of #2 in at most #1 steps
\newcommand{\unifiedhornresolutionbase}{\mathrm{UHRes}} %the base symbol for the function UHRes; DEPENDENCY: \uhres, \uhresi
\newcommand{\uhres}[1]{{\unifiedhornresolutionbase(#1)}} %the set of unified Horn-resolution of #1
\newcommand{\uhresi}[2]{{\unifiedhornresolutionbase_{#1}(#2)}} %the set of unified Horn-resolution of #2 in at most #1 steps
%%Chapter 12
\newcommand{\partismbase}{\mathrm{Part}} %the base symbol for the function Part; DEPENDENCY: \partism
\newcommand{\partism}[2]{{\partismbase(#1, #2)}} %partial isomorphism from #1 to #2
\newcommand{\domainbase}{\mathrm{dom}} %the base symbol for the function dom; DEPENDENCY: \dom
\newcommand{\dom}[1]{{\domainbase(#1)}} %domain of the map #1
\newcommand{\rangebase}{\mathrm{rg}} %the base symbol for the function rg; DEPENDENCY: \rg
\newcommand{\rg}[1]{{\rangebase(#1)}} %range of the map #1
\newcommand{\isop}[1]{\iso_{#1}} %!!
\newcommand{\finiso}{\iso_\mathit{f}} %finitely isomorphic
\newcommand{\partiso}{\iso_\mathit{p}} %partially isomorphic
%HERE
\newcommand{\emb}{\rightarrow} %the relation embeddable; DEPENDENCY: \finemb, \partemb.
\newcommand{\finemb}{\mathrel{\emb_\mathit{f}}} %finitely embeddable
\newcommand{\partemb}{\mathrel{\emb_\mathit{p}}} %partially embeddable
\newcommand{\qrbase}{\mathrm{qr}} %the symbol for the function qr; DEPENDENCY: \qr
\newcommand{\qr}[1]{{\qrbase(#1)}} %quantifier rank
\newcommand{\mrkbase}{\mathrm{mrk}} %the symbol for the function mrk; DEPENDENCY: \mrk
\newcommand{\mrk}[1]{{\mrkbase(#1)}} %modified quantifier rank
\newcommand{\ehrenfeuchtgamebase}{\mathrm{G}} %the base symbol for the function G; DEPENDENCY: \egame, \egamep
\newcommand{\egame}[2]{{\ehrenfeuchtgamebase(#1, #2)}} %ordinary Ehrenfeucht game
\newcommand{\egamep}[3][]{{\ehrenfeuchtgamebase_{#1}(#2, #3)}} %Ehrenfeucht game with parameter(s)
%%Chapter 13
\newcommand{\logsys}{\logicalsystembase} %logical system
\newcommand{\modelclassarg}[3][]{{\modelclassbase^{#1}_{#2}(#3)}} %the class of #1-structures that are models of #3 with respect to the logical system #2
\newcommand{\boole}[1]{\mathrm{Boole}(#1)} %the logical system #1 satisfies the condition Boole
\newcommand{\rel}[1]{\mathrm{Rel}(#1)} %the logical system #1 satisfies the condition Rel
\newcommand{\repl}[1]{\mathrm{Repl}(#1)} %the logical system #1 satisfies the condition Repl
\newcommand{\losko}[1]{\mathrm{L\ddot{o}Sko}(#1)} %the logical system #1 satisfies the condition LoeSko
\newcommand{\comp}[1]{\mathrm{Comp}(#1)} %the logical system #1 satisfies the condition Comp
\newcommand{\weakereq}{\leq} %A \weakereq B: the logical system B is at least as strong as A (A is weaker than or equally strong as B)
\newcommand{\eqstrong}{\sim} %A is equally strong as B
\newcommand{\eff}{\mathrm{eff}} %the modifier eff
\newcommand{\effwkereq}{\mathrel{\weakereq_\eff}} %A \effwkereq B: B is at least as effectively strong as A
\newcommand{\effeqstrng}{\mathrel{\eqstrong_\eff}} %A \effeqstrng B: A is effectively equally strong as B

%New commands for Appendix A
\newcommand{\numl}[1]{\underline{#1}} %the numeral corresponding to #1 in underline
\newcommand{\numb}[1]{\mbf{#1}} %the numeral corresponding to #1 in boldface

%Redefining Commands
\renewcommand{\qedsymbol}{$\talloblong$}
\renewcommand{\thechapter}{\Roman{chapter}}
\renewcommand{\thesection}{\arabic{section}}
\renewcommand{\theequation}{\arabic{equation}}

%New Lengths
\newlength{\DefaultQuoteLength}
\settowidth{\DefaultQuoteLength}{note that the terms here are only slightly different from those in}
\newlength{\DefaultTabularizedArgumentLength}
\settowidth{\DefaultTabularizedArgumentLength}{note that the terms here are only slightly different from aaaa}
\newlength{\DefaultDefinitionItemLength}
\settowidth{\DefaultDefinitionItemLength}{note that the terms here are only slightly from}

%New Environments
%%General
\newenvironment{definition}[1]{\textbf{#1.}\ }{}
\newenvironment{theorem}[1]{\textbf{#1.}\ \begin{em}}{\end{em}}
%\newenvironment{smallcenter}{\smallskip\\\phantom{.}\hfill}{\hfill\phantom{.}\smallskip\\}
\newenvironment{medcenter}{ \medskip\\ \phantom{(} \hfill }{ \hfill \phantom{)} \medskip\\}
\newenvironment{quoteno}[1]{#1\hfill}{\hfill\phantom{(+)}}
\newenvironment{bquoteno}[2]{#2\hfill\begin{minipage}[c]{#1}}{\end{minipage}}
%%Chapter 4
\newenvironment{seqrule}{\begin{array}}{\end{array}}
\newenvironment{derivation}{\begin{tabular}{llll}}{\end{tabular}}
%%Chapter 10
\newenvironment{program}{\begin{array}{rl}}{\end{array}}
