%Author: Wei-Lin (Linisac) Wu
\documentclass[11pt, leqno]{report}


%Packages
\usepackage{amssymb}
\usepackage{enumerate}
\usepackage{graphicx}
\usepackage{stmaryrd}
\usepackage{colonequals}
\usepackage{amsthm}
\usepackage{array}
\usepackage{amsmath}
\usepackage{amscd}
\usepackage{paralist}
\usepackage{bm}
\usepackage{titlesec}
\usepackage[margin=2cm]{geometry}

%ifthen package
\usepackage{ifthen}


%New Commands for Abbreviations
%%Catalog <
%%%general
%%%definition
%%%enumeration, sequence, vector (tuple), support
%%%set operations
%%%function type, function operations
%%%special set(s), theory
%%%structure, symbol interpretation, structure operations, structure relations
%%%partial isomorphism
%%%elementary equivalence
%%%Ehrenfeucht-Fraisse game
%%%isomorphism types
%%%graph, graph functions, graph relations
%%%logic
%%%counting specifiers
%%%vocabulary, connectives, quantifiers, formula operations
%%%substitution
%%%logic relations
%%%logic operations
%%%finite notions
%%%nullary relations and predicates
%%%operations on formulas and structures
%%%global relations
%%%special classes of structures
%%%extension axioms
%%%labeled/unlabeled classes and probabilities
%% >

%%general <
\newcommand{\header}[1]{{\rm\textbf{#1.}}}
\newcommand{\refitem}[1]{{\rm#1}}
\newcommand{\mathmode}[1]{\begin{math}#1\end{math}}
\newcommand{\paren}[1]{\left(#1\right)} %flexible parentheses
\newcommand{\of}[1]{{[#1]}} %logic or class with vocabulary #1
\newcommand{\prmt}[2]{\mathrm{P}^{#1}_{#2}} %permutation (in combinatorics)
\newcommand{\cmbn}[2]{\mathrm{C}^{#1}_{#2}} %combination (in combinatorics)
%% >

%%definition <
\newcommand{\defas}{\colonequals}
%% >

%%enumeration, sequence, vector (tuple), support <
\newcommand{\etc}{\ldots} %et cetera
\newcommand{\length}{\mathrm{length}}
\newcommand{\emptyseq}{\emptyset} %empty sequence
\newcommand{\seq}[2]{(#1)_{#2}} %indexed sequence
\newcommand{\vect}[1]{\overline{#1}} %vector, \vect{a} = \overbar{a}
\newcommand{\supp}{\mathrm{supp}}
%% >

%%some common operators <
\newcommand{\mul}{\mathop{\cdot}} %multiplication
\newcommand{\abs}[1]{\left|#1\right|}
%% >

%%set operations <
\newcommand{\sete}[1]{{\{#1\}}} %set by enumeration. sete{a, b, c} = {a, b, c}
\newcommand{\setm}[2]{{\{#1 \mid #2\}}} %set by math description. setd{#1}{#2} = {#1 | #2}
\newcommand{\sett}[2]{{\{#1 \mid \text{\rm #2}\}}} %set by text description. sett{#1}{#2} = {#1 | #2}, where #2 is a text
\newcommand{\cart}{\mathop{\times}} %cartesian product of two sets
\newcommand{\cartpwr}[2]{#1^{#2}} %(#2)nd cartesian power of set #1
\newcommand{\intsc}{\cap} %intersection
\newcommand{\union}{\cup} %union
\newcommand{\bintsc}{\bigcap} %big intersection
\newcommand{\bunion}{\bigcup} %big union
\newcommand{\card}[1]{\|#1\|} %cardinality. card{A} = || A ||
\newcommand{\cmpl}[1]{#1^\mathit{c}} %complement of the set #1
%% >

%%function operations <
\newcommand{\dm}{\mathrm{do}} %domain (of a function)
\newcommand{\rg}{\mathrm{rg}} %range (of a function)
\newcommand{\emptymap}{\emptyset} %empty map
\newcommand{\inv}[1]{#1^{-1}} %inverse of function #1
%% >

%%special set(s), theory <
\newcommand{\nat}{\mathbb{N}} %set of natural numbers
\newcommand{\zah}{\mathbb{Z}} %set of integers
\newcommand{\posint}{\mathbb{Z}_+} %set of positive integers
\newcommand{\theory}{\mathit{T}} %theory
\newcommand{\rand}{\mathrm{rand}} %random
\newcommand{\spec}{\mathrm{Spec}} %spectrum
%% >

%%structure, symbol interpretation, structure operations, structure relations <
\newcommand{\strct}[1]{\mathcal{#1}} %structure
\newcommand{\intpr}[2]{#1^{#2}} %symbol #1 interpreted under #2
\newcommand{\sbstrct}[2]{#1^{#2}} %substructure of #2 induced by the subset #1
\newcommand{\isom}[1][]{\ifthenelse{\equal{#1}{}}{\cong}{\cong^{#1}}} %isomorphic in #1 number of variables
\newcommand{\product}{\cart} %product of two structures
\newcommand{\dunion}{\mathop{\dot{\cup}}} %disjoint union of two structures
\newcommand{\ordsum}{\mathop{\lhd}} %ordered sum of two ordered structures
\newcommand{\ultprd}[2]{#1^{#2}} %ultra product. \ultprd{A}{I} = A^I
\newcommand{\ball}[1][]{\ifthenelse{\equal{#1}{}}{{\mathit{S}}}{{\mathit{S}^{#1}}}}
\newcommand{\ballstrct}[1][]{\ifthenelse{\equal{#1}{}}{{\mathcal{S}}}{{\mathcal{S}^{#1}}}}
%% >

%%partial isomorphism <
\newcommand{\partisoms}{\mathrm{Part}} %set of partial isomorphisms between structures #1 and #2
\newcommand{\partially}{\mathrm{part}} %partially (isomorphic)
%% >

%%elementary equivalence <
\newcommand{\equv}[1][]{\ifthenelse{\equal{#1}{}}{\equiv}{\equiv^{#1}}} %equivalence in (#1)-logic
%% >

%%Ehrenfeucht-Fraisse games <
\newcommand{\efgame}{\mathrm{G}} %Ehrenfeucht-Fraisse game
\newcommand{\fingame}{\efgame} %finite Ehrenfeucht-Fraisse game
\newcommand{\game}[2][]{\ifthenelse{\equal{#1}{}}{{\efgame_{#2}}}{{\efgame^{#1}_{#2}}}} %the game G^#1_#2
\newcommand{\winpos}[2][]{\ifthenelse{\equal{#1}{}}{\mathit{W}_{#2}}{\mathit{W}^{#1}_{#2}}}
\newcommand{\msogame}[1]{\operatorname{\MSO-\mathrm{G}}_{#1}} %the game in monadic second-order logic
%% >

%%isomorphism types <
\newcommand{\hint}[2]{\varphi^{#1}_{#2}} %(#1)-isomorphic type (Hintikka formula) over #2 in first-order logic
\newcommand{\ityp}[3][]{\ifthenelse{\equal{#1}{}}{{\psi^{#2}_{#3}}}{{^{#1}\psi^{#2}_{#3}}}} %(#2)-isomorphism type over #2 in other logics in #1 number of variables
%% >

%%graph, graph functions, graph relations <
\newcommand{\graph}{\mathrm{GRAPH}} %the set of finite graphs
\newcommand{\conn}{\mathrm{CONN}} %the set of connected graphs
\newcommand{\dist}[1][]{\ifthenelse{\equal{#1}{}}{\mathit{d}}{\mathit{d}_{#1}}} %distance between two points in a graph
\newcommand{\tree}{\mathrm{TREE}} %the set of finite trees
\newcommand{\gaifman}{\mathcal{G}} %the Gaifman graph of #1 (structure)
%% >

%%logics <
\newcommand{\folog}[1][]{\ifthenelse{\equal{#1}{}}{\mathrm{FO}}{\mathrm{FO}^{#1}}} %first-order logic
\newcommand{\solog}{\mathrm{SO}} %second-order logic
\newcommand{\msolog}{\mathrm{MSO}} %monadic second-order logic
\newcommand{\inflog}[2][]{\ifthenelse{\equal{#1}{}}{\mathrm{L}_{#2\omega}}{\mathrm{L}^{#1}_{#2\omega}}} %infinitary logic parameterized by the number of distinct variables (#1) and the upper bound on the sizes of sets over which disjunctions or conjunctions can be taken (#2), the upper bound on the lengths of strings of quantifiers is fixed, i.e. \omega
%\newcommand{\logic}[2][]{\ifthenelse{\equal{#1}{}}{\mathrm{L}_{#2}}{\mathrm{L}^{#1}_{#2}}} %parameterized logic (like infinitary logics)
\newcommand{\logic}{\mathcal{L}} %(general) logic
%% >

%%counting specifiers <
\newcommand{\atleast}[1]{\geq #1} %at least #1
\newcommand{\atmost}[1]{\leq #1} %at most #1
\newcommand{\exactly}[1]{= #1} %exactly #1
\newcommand{\cardatleast}[1]{\varphi_{\atleast{#1}}} %cardinality at least #1
\newcommand{\cardatmost}[1]{\varphi_{\atmost{#1}}} %cardinality at most #1
\newcommand{\cardexactly}[1]{\varphi_{\exactly{#1}}} %cardinality exactly #1
%% >

%%vocabulary, connectives, quantifiers, formula operations <
\newcommand{\emptyvoc}{\emptyset} %empty vocabulary
\newcommand{\lthen}{\rightarrow} %logicial then
\newcommand{\liff}{\leftrightarrow} %logical iff (between two formulas)
\newcommand{\blor}{\bigvee} %big disjunction
\newcommand{\bland}{\bigwedge} %big conjunction
\newcommand{\existatleast}[1]{\exists^{\atleast{#1}}} %exist at least #1
\newcommand{\existatmost}[1]{\exists^{\atmost{#1}}} %exist at most #1
\newcommand{\existexactly}[1]{\exists^{\exactly{#1}}} %exist exactly #1
\newcommand{\free}{\mathrm{free}} %set of free variables of #1
\newcommand{\qr}{\mathrm{qr}} %quantifier rank (of a formula)
\newcommand{\modclass}{\mathrm{Mod}} %model class (of a formula)
%% >

%%substitution <
\newcommand{\sbst}[2]{{\scriptstyle\frac{#1}{#2}}} %substitution of #2 with #1
%% >
\newenvironment{perm}{\left(\begin{array}}{\end{array}\right)} %permutation of variables

%%logic relations <
\newcommand{\satis}{\models} %satisfaction relation
\newcommand{\consq}{\models} %consequence relation
%% >

%%logic operations <
\newcommand{\rel}[1]{#1^\mathit{r}} %#1 relationalized
\newcommand{\invrel}[1]{#1^{-\mathit{r}}} %#1 inverse-relationalized
%% >

%%finite notions <
\newcommand{\fin}{\mathrm{fin}} %finite
\newcommand{\consqfin}{\consq_\fin} %consequence relation restricted to finite structures
%% >

%%nullary relations and predicates <
\newcommand{\true}{\mathrm{TRUE}} %TRUE symbol
\newcommand{\false}{\mathrm{FALSE}} %FALSE symbol
\newcommand{\tr}{\mathrm{T}} %T symbol
\newcommand{\fls}{\mathrm{F}} %F symbol
%% >

%%operations on formulas and structures <
\newcommand{\tuplesby}[2]{{#1^{#2}(\_)}} %tuples defined by formula #1 and structure #2
\newcommand{\rltv}[2]{#1^{#2}} %formula #1 relativized to the set #2
%% >

%%global relations <
\newcommand{\transcls}{\mathrm{TC}} %transitive closure relation
%% >

%%special classes of structures <
\newcommand{\ordclass}{\mathcal{O}} %class of finite ordered structures (in a vocabulary)
\newcommand{\even}{\mathrm{EVEN}} %class of finite structures of even cardinality (in a vocabulary)
%% >

%%extension axioms <
\newcommand{\extaxm}[1][]{\ifthenelse{\equal{#1}{}}{\chi}{\chi_{#1}}} %extension axiom with optional parameter #1 (set of formulas)
\newcommand{\conjextaxm}[1]{\epsilon_{#1}} %conjunction of extension axioms with variables indexed no more than #1
%% >

%%labeled/unlabeled classes and probabilities <
\newcommand{\lclass}[1]{\mathit{L}_{#1}} %labeled class of strcutures
\newcommand{\lprob}[1][]{\ifthenelse{\equal{#1}{}}{\mathit{l}}{\mathit{l}_{#1}}} %labeled probability over the class of structures of cardinality #1 or labeld asymptotic probability
\newcommand{\uclass}[1]{\mathit{U}_{#1}} %unlabeled class of strcutures
\newcommand{\uprob}[1][]{\ifthenelse{\equal{#1}{}}{\mathit{u}}{\mathit{u}_{#1}}} %unlabeled probability over the class of structures of cardinality #1 or unlabeld asymptotic probability
%% >


\newenvironment{quoteno}[1]{(#1)\hfill}{\hfill\phantom{(+)}}


%if resetting of equation counter is required, use the instruction below:
%\setcounter{equation}{0}

\theoremstyle{plain}
\newtheorem*{fact}{Fact}

\theoremstyle{remark}
\newtheorem*{note}{Note}

\theoremstyle{definition}
\newtheorem*{remark}{Remark}

\begin{document}
%\begin{titlepage}
%\noindent\textsc{\LARGE Annotations to ``Finite Model Theory''}\bigskip\\
%\textsc{\large with Solutions Hints}
%\end{titlepage}
\titleformat{\chapter}[hang]{\Large\bfseries}{\thechapter.}{.5em}{}
\titleformat{\section}[hang]{\large\bfseries}{\thesection}{.5em}{}
%%\setcounter{chapter}{1}
\chapter{Preliminaries}
%Paragraph A----------------------------------------------------------------------------------
\paragraph{A Structures}
\begin{enumerate}[1.]
%
\item \header{Cycles Have Length at Least $2$}
%
\item \header{An Example of $\strct{A} \ordsum \strct{B} \not\isom \strct{B} \ordsum \strct{A}$} Let $\tau \defas \sete{{<}, P}$ in which $P$ is unary, $\strct{A} = (\sete{a}, \emptyset, \emptyset), \strct{B} = (\sete{b}, \emptyset, \sete{b})$.
%
\item \header{Note on the Last Paragraph of A3 Operations on Structures} In fact, $\strct{A} \dunion \strct{B}$ and $\strct{B} \dunion \strct{A}$ are not only isomorphic but also \emph{identical} since $A \union B = B \union A$ and $\intpr{R}{\strct{A}} \union \intpr{R}{\strct{B}} = \intpr{R}{\strct{B}} \union \intpr{R}{\strct{A}}$ for $R \in \tau$.
%
\end{enumerate}
%End of Paragraph A---------------------------------------------------------------------------
%Paragraph C----------------------------------------------------------------------------------
\paragraph{C Some Classical Results of First-Order Logic}
\begin{enumerate}[1.]
%
\item \header{Note on the Proof of Lemma 1.0.6} It is easy to drop the assumption that $\varphi$ is a sentence: All variables occurring in $\varphi$, if any, can be replaced by new constant symbols, i.e.\ constants not in $\tau$.
%
\end{enumerate}
%End of Paragraph C---------------------------------------------------------------------------
%Paragraph D----------------------------------------------------------------------------------
\paragraph{D Model Classes and Global Relations}
\begin{enumerate}[1.]
%
\item \header{Note on Sets of Tuples Defined by Formulas} One can infer by context that in the definition of $\tuplesby{\varphi}{\strct{A}}$ in which $\varphi = \varphi(x_1, \etc, x_n)$, the free variables in $\varphi$ are among $x_1, \etc, x_n$.
%
\end{enumerate}
%End of Paragraph D-----------------------------------------------------------------------------
%%\setcounter{chapter}{1}
\chapter{The Ehrenfeucht-Fra\"{i}ss\'{e} Method}
\setcounter{section}{1}
%section 2------------------------------------------------------------------------------------
\section{Ehrenfeucht's Theorem}
\begin{enumerate}[1.]
%
\item \header{Note on Definition 2.2.1} The condition for a map $p$ to be a partial isomorphism between $\strct{A}$ and $\strct{B}$ is slightly different in \cite{EFT}: the designated constants $\intpr{c}{A}$ are not required to be in $\dom{p}$ but if they are, the condition $p(\intpr{c}{A}) = \intpr{c}{B}$ must be satisfied.
%
\item \header{Note on Remarks 2.2.2} For part (a), note that the empty map $p = \emptyset$ is \emph{not} a partial isomorphism if $\tau$ does contain constants, by Definition 2.2.1.
\medskip\\
For part (c), the notation $\vect{a} \mapsto \vect{b}$ can be seen as a shorthand for the set
\[
\setm{(a_i, b_i)}{i \leq s} \union \setm{(\intpr{c}{\strct{A}}, \intpr{c}{\strct{B}})}{c \in \tau},\]
which may not define a map, nor a partial isomorphism. If the above set does define a map, then it is the graph of this map. Statement (i) says $\vect{a} \mapsto \vect{b}$ not only defines a map but even a partial isomorphism between $\strct{A}$ and $\strct{B}$.
\medskip\\
Also note that if $\tau$ contains constants then the notation $\vect{a} \mapsto \vect{b}$ contains mapping of the constants, which are \emph{hidden} from it; in this case, $\emptyseq \mapsto \emptyseq$ does not equal the empty map $\emptymap$. $\emptyseq \mapsto \emptyseq = \emptymap$ when $\tau$ contains no constants.
%
\item \header{Note on Ehrenfeucht Games.} It is clear from definition that at most one of the players has a winning strategy for the game $\game{m}{\strct{A}, \vect{a}, \strct{B}, \vect{b}}$.
\medskip\\
Moreover, it is also true that at least one of the players has a winning strategy for the game $\game{m}{\strct{A}, \vect{a}, \strct{B}, \vect{b}}$, i.e.\ \emph{if the duplicator does not win $\game{m}{\strct{A}, \vect{a}, \strct{B}, \vect{b}}$ then the spoiler wins it.} See 2.3.16 in text.
%
\item \header{Note on the Proof of Lemma 2.2.3(b)} The statement ``Then $p : \vect{a} \mapsto \vect{b} \in \partisoms{\strct{A}}{\strct{B}}$ with $\dom{p} = A$'' follows from the fact that a submap of a partial isomorphism is also a partial isomorphism (see the next note).
%
\item \header{Note on Lemma 2.2.4} Part (c) is equivalent to say that a submap (i.e.\ a map that is a subset of another map in terms of their graphs) of a partial isomorphism is also a partial isomorphism.
\medskip\\
On the other hand, it is useful to use the alternative logically equivalent forms (such as contraposition) of parts (b) or (c) in deriving properties of the Ehrenfeucht games.
%
\item \header{Note on Lemma 2.2.6 and the Remark below It} It is better to prove the lemma and the remark that ``the conjunctions and disjunctions in the definition of $\ityp{m}{\strct{A}, \vect{a}}$ are finite'' all at once:
\medskip\\
\emph{For $s, m \geq 0$, $\ityp{m}{\strct{A}, \vect{a}}$ is a well-formed formula - i.e.\ the conjunctions and disjunctions (if any) in it are finite - for any structure $\strct{A}$ and $\vect{a} \in \cartpwr{A}{s}$ and the set $\sett{\ityp{m}{\strct{A}, \vect{a}}}{\begin{math}\strct{A}\end{math} a structure and \begin{math}\vect{a} \in \cartpwr{A}{s}\end{math}}$ is finite.}
\medskip\\
The induction is performed on $m$. The base case follows from that for $s > 0$, the set $\sett{\varphi(\seq{v}{s})}{\begin{math}\varphi\end{math} atomic or negated atomic}$ is finite.
%
\item \header{Note on the Proof of Theorem 2.2.8 (Ehrenfeucht's Theorem)} To prove (i) implies (iii), the case $m = 0$ can be handled by applying 2.2.4(a) in addition to the equivalence between (i) and (ii) of 2.2.2(c).
\medskip\\
The induction part of (i) implying (iii) in fact proves $\strct{A} \satis \varphi[\vect{a}]$ iff $\strct{B} \satis \varphi[\vect{b}]$.
%
\item \header{Note on Theorem 2.2.11} There is typo in the statement of this theorem: ``$\vect{a} \in \strct{A}$'' should be replaced by ``$\vect{a} \in \cartpwr{A}{s}$''.
%
\end{enumerate}
%end of section 2-----------------------------------------------------------------------------


%section 3------------------------------------------------------------------------------------
\section{Examples and Fra\"{i}ss\'{e}'s Theorem}
\begin{enumerate}[1.]
%
\item \header{Hint to Exercise 2.3.2} Note that $I_j \subseteq \tilde{I}_j$ and that if $p \in \partisoms{\strct{A}}{\strct{B}}$ and if $q \subseteq p$ then $q \in \partisoms{\strct{A}}{\strct{B}}$.
\begin{remark}
There is a typo: ``$\emptyseq \mapsto \emptyseq \in I_j$'' should be replaced by ``$\emptyset \mapsto \emptyset \in \tilde{I}_j$''.
\end{remark}
%
\item \header{Note on Corollary 2.3.4} For $s = 0$ the statement of 2.3.3(iii) becomes
\begin{quoteno}{$\ast$}
There is $\seqi{I_j}{j \leq m}$ with $\emptyseq \mapsto \emptyseq \in I_m$ such that $\seqi{I_j}{j \leq m} : \strct{A} \isom[m] \strct{B}$.
\end{quoteno}
\medskip\\
It is equivalent to 2.3.4(iii): The latter obviously follows the former; to derive the former from the latter, note that by 2.3.2 if $\seqi{I_j}{j \leq m} : \strct{A} \isom[m] \strct{B}$ then $\seqi{\tilde{I}_j}{j \leq m} : \strct{A} \isom[m] \strct{B}$ and $\emptyseq \mapsto \emptyseq \in \tilde{I}_m$.
%
\item \header{Note on Example 2.3.5} Let $\tau$ be an arbitrary symbol set that consists of relation symbols $\seq{P}{n}$ and constants $\seq{c}{k}$ where $n, k \in \nat$.
\medskip\\
For any $m \in \nat$, consider the two $\tau$-structures $\strct{A}$ and $\strct{B}$ where $A$ consists of elements $\seq{a}{m + 1}$ other than $\seq{\intpr{c}{\strct{A}}}{n}$, $\intpr{c_i}{\strct{A}} \neq \intpr{c_j}{\strct{A}}$ if $i \neq j$, $\intpr{P_i}{\strct{A}} = \emptyset$, $B$ consists of elements $\seq{b}{m + 2}$ other than $\seq{\intpr{c}{\strct{B}}}{n}$, $\intpr{c_i}{\strct{B}} \neq \intpr{c_j}{\strct{B}}$ if $i \neq j$, $\intpr{P_i}{\strct{B}} = \emptyset$.
\medskip\\
Obviously, exactly one between $\strct{A}$ and $\strct{B}$ is a member of $\even{\tau}$. However, it is also true that $\strct{A} \equv[m] \strct{B}$: the map $\enum{a}{m + 1} \mapsto \enum{b}{m + 1}$ can be used as a winning strategy (note that implicitly $\intpr{c_i}{\strct{A}}$ is mapped to $\intpr{c_i}{\strct{B}}$, cf.\ 2.2.2(c)(i)) for the duplicator in the game $\game{m}{\strct{A}, \strct{B}}$. Thus $\even{\tau}$ is not axiomatizable by 2.2.12.
%
\item \header{Note on Example 2.3.8} There is a typo: ``$\dist[j](a, a^\prime)$'' appearing in the definition of distance function should be replaced by ``$\dist[j](a, b)$''.
%
\item \header{Hint to Exercise 2.3.9} For simplicity, take the structure $\strct{C}_l$ that is isomorphic to $\strct{B}_l \dsjuni \strct{D}_l$, where $C_l \defas \sete{0, \ldots, 2l + 1}$ and the substructures $\substr{\sete{0, \ldots, l}}{\strct{C}_l}$ induced by $\sete{0, \ldots, l}$ and $\substr{\sete{l + 1, \ldots, 2l + 1}}{\strct{C}_l}$ induced by $\sete{l + 1, \ldots, 2l + 1}$ are isomorphic to $\strct{B}_l$ and $\strct{D}_l$, respectively.
\medskip\\
For $\strct{B}_l$ and $\strct{C}_l$ define the distance function $d$ on $B_l \cart B_l$ and on $C_l \cart C_l$ as
\[
\dist(h, k) \defas
\begin{cases}
\mbox{length of the shortest path from \begin{math}h\end{math} to \begin{math}k\end{math}} & \mbox{if there is one} \cr
\infty & \mbox{else},
\end{cases}
\]
and take the truncated version $\dist[j]$ where
\[
\dist[j](h, k) \defas
\begin{cases}
\dist(h, k) & \mbox{if \begin{math}\dist(h, k) < 2^j\end{math}} \cr
\infty & \mbox{else}.
\end{cases}
\]
For $m \geq 0$ choose $l \geq 2^m$. Consider $\seqi{I_j}{j \leq m}$ where $p \in I_j$ if and only if $p$ is a partial isomorphism between $\strct{B}_l$ and $\strct{C}_l$ such that $\card{p} \leq m - j + 2$, $p(0) = 0$, $p(l) = l$ and $\dist[j](h, k) = \dist[j](p(h), p(k))$ for $h, k \in \dom{p}$.
\medskip\\
It remains to verify $\seqi{I_j}{j \leq m} : \strct{B}_l \isom[m] \strct{C}_l$, which is omitted here. (For the forth-property, if $p \in I_{j + 1}$ and $b \in B_l$ then distinguish two cases according to whether it is true that ``there is a $b^\prime \in B_l$ such that $\dist[j](b, b^\prime) < 2^j$ or $\dist[j](b^\prime, b) < 2^j$'', a technique used in 2.3.6.)
%
\item \header{Note on Corollary 2.3.11} Here $(\strct{A}, \vect{a}) \equv[m] (\strct{B}, \vect{b})$ means ``$\vect{a}$ satisfies in $\strct{A}$ the same formulas of quantifier rank $\leq m$ as $\vect{b}$ in $\strct{B}$.'' (Consider the transition from 2.2.8(iii) to 2.2.9(iii).)
%
\item \header{Note on Corollary 2.3.11} Is it true that
\begin{quote}
``\emph{If $(\strct{A}_1, \vect{a}_1) \equv[m] (\strct{B}_1, \vect{b}_1)$ and $(\strct{A}_2, \vect{a}_2) \equv[m] (\strct{B}_2, \vect{b}_2)$ then $(\strct{A}_1 \cart \strct{A}_2, \vect{a}_1 \cart \vect{a}_2) \equv[m] (\strct{B}_1 \cart \strct{B}_2, \vect{b}_1 \cart \vect{b}_2)$}''?
\end{quote}
My guess is yes. Note that if $\vect{a}_1 \isom \vect{b}_1$ and $\vect{a}_2 \isom \vect{b}_2$ then $\vect{a}_1 \cart \vect{a}_2 \isom \vect{b}_1 \cart \vect{b}_2$.
%
\item \header{Hint to Exercise 2.3.12} An equivalent condition to ``$\min\sete{\card{A_\alpha}, m} = \min\sete{\card{B_\alpha}, m}$'' is:
\begin{quote}
$\card{B_\alpha} = \card{A_\alpha}$ if $\card{A_\alpha} < m$, and $\card{B_\alpha} \geq m$ otherwise.
\end{quote}
In addition, an alternative statement to the condition ``$\card{B_\alpha} \geq m$'' is:
\begin{quote}
For $0 \leq j < m$, $\card{B_\alpha} \neq j$.
\end{quote}
And for this exercise it is appropriate to define $\existexactly{0}x Rx \defas \neg\exists x Rx$.
\begin{remark}
A similar exercise is XII.3.17 in \cite{EFT}.
\end{remark}
%
\item \header{Hint to Exercise 2.3.13} (INCOMPLETE)
Think of $\ordsum^n \strct{A}$ as an $n$-element linear ordering in which every point ``expands'' to $\strct{A}$.\\
\medskip\\
More precisely, define for $a, a^\prime$ in the domain of $\ordsum^n \strct{A}$ the distance function
\[
\dist(a, a^\prime) \defas \abs{i - j},
\]
where $a$ is an element from the $i$th copy of $\strct{A}$ and $a^\prime$ from the $j$th. The truncated versions of distance function are defined analogously. A winning strategy for the duplicator is the same as that in 2.3.6 except that if the spoiler chooses an element from a copy $\strct{A}$ of $\ordsum^l \strct{A}$ (or $\ordsum^k \strct{A}$) then the duplicator chooses exactly the same element from the corresponding copy $\strct{A}$ of $\ordsum^k \strct{A}$ (or $\ordsum^l \strct{A}$, respectively).
%
\item \header{Hint to Exercise 2.3.14} (INCOMPLETE)
The notation ``$(\strct{A}, \vect{a}) \isom[m] (\strct{B}, \vect{b})$'' is undefined in text, however it can be understood as:
\begin{quote}
There is $\seqi{I_j}{j \leq m}$ with $\vect{a} \mapsto \vect{b} \in I_m$ such that $\seqi{I_j}{j \leq m} : \strct{A} \isom[m] \strct{B}$.
\end{quote}
This is statement (iii) of 2.3.3. Therefore this exercise is an immediate consequence of 2.3.3.
\begin{remark}
The premise ``for $\vect{a} \mapsto \vect{b} \in \partisoms{\strct{A}}{\strct{B}}$'' is implied by the statements on both sides of ``iff'', so it can be weakened to ``for $\vect{a} \in A, \vect{b} \in B$''.
\end{remark}
%
\item \header{Note on Remark 2.3.16} As an alternative proof for the last part of 2.2.8, one can show, contrapositively, that \emph{not \refitem{(iii)} implies not \refitem{(i)}}. To be more precise, in the inductive step let $m > 0, \qr{\varphi} = m$, and suppose that $\varphi(\vect{x}) = \exists y \psi$ where $\strct{A} \satis \varphi[\vect{a}]$ but not $\strct{B} \satis \varphi[\vect{b}]$. It follows that there is an $a \in A$ such that for all $b \in B$, $\strct{A} \satis \psi[\vect{a}a]$ but not $\strct{B} \satis \psi[\vect{b}b]$, hence by induction hypothesis the duplicator does not win the game $\game{m - 1}{\strct{A}, \vect{a}a, \strct{B}, \vect{b}b}$. By 2.2.4(b), therefore, we have that the duplicator does not win the game $\game{m}{\strct{A}, \vect{a}, \strct{B}, \vect{b}}$. The other case where $\strct{B} \satis \varphi[\vect{b}]$ but not $\strct{A} \satis \varphi[\vect{a}]$ can be done symmetrically.
\medskip\\
The above argument suggests (the essential part of) a winning strategy of the game $\game{m}{\strct{A}, \vect{a}, \strct{B}, \vect{b}}$ for the spoiler when (iii) in 2.2.8 is false: If a formula $\varphi = \exists y \psi$ has quantifier rank $m > 0$ and if, say, $\strct{A} \satis \varphi[\vect{a}]$, then he picks an element $a \in A$ such that $\strct{A} \satis \psi[\vect{a}a]$; if $\varphi = \forall y \chi$ has quantifier rank $m > 0$ and if, say, not $\strct{B} \satis \varphi[\vect{b}]$ (namely if $\strct{B} \satis \exists y \neg\chi[\vect{b}]$), then he picks an element $b \in B$ such that $\strct{B} \satis \neg\chi[\vect{b}b]$. The spoiler then makes his successive choices accordingly in his turns in every play of the game.
%
\end{enumerate}
%end of section 3-----------------------------------------------------------------------------


%section 4------------------------------------------------------------------------------------
\section{Hanf's Theorem}
\begin{enumerate}[1.]
%
\item \header{Note on the First Paragraph on Page 27} Here the \emph{isomorphism type} of $(\ballstrct{r, a}, a)$ might refer to the equivalent class of $(\ballstrct{r, a}, a)$ induced by the isomorphism relation, i.e.\ the set of structures that are isomorphic to $(\ballstrct{r, a}, a)$, where $(\ballstrct{r, a}, a)$ might refer to the expansion of the structure $\ballstrct{r, a}$ in which $a$ is a distinguished constant.
%
\item \header{Note on the Proof of 2.4.1} Here $\length{\vect{a}}$ refers to the length of the tuple $\vect{a}$, see the first paragraph on page 6.
\medskip\\
On the other hand, by the proof the requirement on the cardinality of $3^m$-balls may be weakened to ``at most $e$ elements.''
%
\item \header{Note on 2.4.2 and 2.4.3} For the proof of 2.4.3 to be valid, in the structure $(D_l, E^\prime_l, P_1, \etl, P_r)$ there must be a point on each of the paths from $a$ to $b_-$ and from $b$ to $a_-$ that is in neither of the $3^m$-balls of $a$ and $b$, in other words, both cycles in the structure must have length greater than $2 \mul 3^m + 1$; otherwise the $3^m$-ball type of $a$ (or $b$) would be different from that of $a$ (or $b$, respectively) in $(\strct{D}_l, P_1, \etl, P_r)$ - the former $3^m$-ball is a cycle, whereas the latter is not.
\medskip\\
In fact, 2.4.3 can be strengthened to allow such points.
%
\item \header{Note on 2.4.4} By definition, the Gaifman graph $\gaifman{\strct{A}}$ of a digraph $\strct{A}$ is the associated (undirected) graph of $\strct{A}$.
%
\item \header{Note on 2.4.5} By the same argument in the proof, it follows that \emph{the class of finite graphs that are not connected cannot be axiomatized by a formula of the form $\exists P_1 \etl \exists P_r \psi$, either.}
\medskip\\
As an immediate consequence, we have that \emph{both classes cannot be axiomatized by a formula of the form $\forall P_1 \etl \forall P_r \chi$.}
%
\item \header{Note on 2.4.6} In the proof there is a (possible) typo: $\intpr{R}{A}$ in $(\strct{G}, \intpr{R}{A})$ should be replaced by $\intpr{R}{G}$ or $\intpr{R}{\strct{G}}$.
\medskip\\
On the other hand, this proposition implies that \emph{the class of finite graphs that are not connected can be axiomatized by a formula of the form $\forall R \chi$.} (Just take the negation of $\exists R \psi$.)
%
\item \header{Brief Solution to Exercise 2.4.7} Using a similar method (basically the pigeonhole principle) we can obtain the corresponding result to 2.4.2, in which the distance $\dist(a, b)$ is greater than $2 \mul 3^m + 1$ (cf.\ \header{Note on 2.4.2 and 2.4.3}). Now let $a_-$ and $b_-$ be the two points such that $(a_-, a)$ and $(b_-, b)$ are edges in $\strct{H}_l$. Obtain the structure $\strct{H}_l^\prime$ from $\strct{H}_l$ by removing these two edges and adding $(b_-, a)$ and $(a_-, b)$. Then likewise we obtain the corresponding result to 2.4.3, and hence that to 2.4.5.
\medskip\\
As for the corresponding result to 2.4.6, note that a digraph is cyclic if and only if there is a linear ordering over a set of at least two points in which there is an edge from $x$ to $y$ if and only if $y$ is immediately greater than $x$ or $x$ is the greatest element and $y$ the least in the linear ordering. Finally, formulate the condition and take the negation.
\begin{note}
It does not seem possible to axiomatize this class by a formula of the form $\forall P \psi$ where $P$ is unary and $\psi$ first-order (by the corresponding result to 2.4.5 mentioned above and the discussion in \header{Note on 2.4.5}) or the form $\exists R \psi$ where $R$ is binary and $\psi$ first-order as required (this is conjectured, however).
\end{note}
%
\item \header{Brief Solution to Exercise 2.4.8} An informal formulation for $\psi(x, y)$ is already present in the description of the exercise.
\medskip\\
Also observe that the formula $\forall x \forall y \exists P \varphi$ axiomatizes the class of finite and connected graphs, and hence by 2.4.5 is not equivalent to a sentence of the form $\exists P_1 \etl \exists P_r \chi$.
%
\end{enumerate}
%end of section 4-----------------------------------------------------------------------------
%\setcounter{chapter}{2}
\chapter{More on Games}
%section 1-------------------------------------------------------------------------------------
\section{Second-Order Logic}
\begin{enumerate}[1.]
%
\item \header{Hint to Exercise 3.1.2} Argue as in 2.2.2, 2.2.4, and 2.2.6-8.
%
\end{enumerate}
%end of section 1-----------------------------------------------------------------------------


%section 2------------------------------------------------------------------------------------
\section{Infinitary Logic: The Logics $\logic{\infty\omega}$ and $\logic{\omega_1\omega}$}
\begin{enumerate}[1.]
%
\item \header{Subformulas of $\logic{\infty\omega}$-Sentences Only Have Finitely Many Free Variables} In fact, one can show by induction on the formation of formulas that if $\varphi \in \logic{\infty\omega}$ has infinitely many free variables, then any $\logic{\infty\omega}$-formula having $\varphi$ as a subformula must also have infinitely many free variables.
%
\item \header{Note on 3.2.3} One can easily see that $\varphi(\vect{x})$ is equivalent to a countable conjunction of first-order formulas:
\[
\bland \setm{\cardexactly{n} \lthen \blor\setm{\hint{\card{A} + 1}{\strct{A}, \vect{a}}(\vect{x})}{\card{A} = n, \vect{a} \in A, \strct{A} \satis \varphi[\vect{a}]}}{n \geq 1}.
\]
The above disjunction is finite.
%
\item \header{Note on the Proof of 3.2.7} In the direction from \refitem{(iii)} to \refitem{(iv)}, the length $s$ of tuples in \refitem{($\ast$)} is better to be replaced by say $r$, since $s$ is fixed for the length of $\vect{a}$ and $\vect{b}$ whereas the tuples in \refitem{($\ast$)} may have different lengths.
\medskip\\
There is a typo in the direction from \refitem{(iv)} to \refitem{(iii)}: ``$a \in I$'' should change to ``$a \in A$''.
%
\item \header{Note on \refitem{(iii)} of 3.2.8} To transition from \refitem{(iii)} of 3.2.7 for $s = 0$, one refers to 2.3.2 (also cf.\ the transition from 2.3.3 to 2.3.4 using 2.3.2).
%
\item \header{Note on 3.2.11}
\begin{enumerate}[(a)]
%%
\item For $r \geq 0$, the set $\Delta_{r + 1}$ is finite.
%%
\item In case $r = 1$ an extension axiom has the form
\[
\forall v_1 \exists v_2 (v_1 \neq v_2 \land \bland_{\varphi \in \Phi} \varphi \land \bland_{\varphi \in \cmpl{\Phi}} \neg\varphi).
\]
%%
\item The condition ``$\hint{0}{\strct{A}, \vect{a}} = \hint{0}{\strct{B}, \vect{b}}$'' in the definition of the set $I$ of maps is equivalent to ``$\vect{a} \mapsto \vect{b} \in \partisoms(\strct{A}, \strct{B})$'', by 2.2.5 and parts (b) and (c) of 2.2.7.
%%
\item Here we may assume $\tr, \fls$ are included as \emph{atomic sentences} in the language (cf.\ part B in chapter 1) so that $\hint{0}{\strct{A}, \emptyseq}, \hint{0}{\strct{B}, \emptyseq}$ are defined; in fact, both formulas are equal to $\tr \land \neg\fls$.
%%
\end{enumerate}
%
\item \header{Hint to Exercise 3.2.13} Note that $x_1$ is the first element of the tuple $\vect{x}$ and therefore the sentence $\forall \vect{x} (\existexactly{1}x F\vect{x}x \land F\vect{x}x_1)$ formalizes the idea that $F$ is a (total) function that projects a tuple of parameter(s) onto its first parameter.
%
\item \header{Solution to Exercise 3.2.14} Without loss of generality let us assume that $A$ is finite and that $\card{A} = \min\sete{\card{A}, \card{B}}$. We also assume, for simplicity, that partial isomorphisms from $\strct{A}$ to $\strct{B}$ take the form $\vect{a} \mapsto \vect{b}$ (having finite domains and ranges) in which $\vect{a}$ and $\vect{b}$ consist of distinct elements.
\medskip\\
By ?? we can distinguish three cases:
\begin{enumerate}[(1)]
%%
\item $\strct{A} \isom \strct{B}$.
%%
\item For $m \in \nat$, not $\strct{A} \isom_m \strct{B}$.
%%
\item For some $m \leq \card{A}$, $\strct{A} \isom_m \strct{B}$ but not $\strct{A} \isom_{m + 1} \strct{B}$.
%%
\end{enumerate}
In case (1), we have $\winpos{0}(\strct{A}, \strct{B}) = \partisoms(\strct{A}, \strct{B}) = \winpos{\infty}(\strct{A}, \strct{B})$, which contains all isomorphisms from $\strct{A}$ to $\strct{B}$. Thus, set $m_0 \defas 0$.
\medskip\\
In case (2), we have $\winpos{0}(\strct{A}, \strct{B}) = \emptyset = \winpos{\infty}(\strct{A}, \strct{B})$. Thus, set $m_0 \defas 0$.
\medskip\\
In case (3), observe that for $j < m$,
\[
\winpos{j}(\strct{A}, \strct{B}) \supsetneq \winpos{j + 1}(\strct{A}, \strct{B}).
\]
In fact, the sets $\winpos{0}(\strct{A}, \strct{B}), \etc, \winpos{m}(\strct{A}, \strct{B})$ are all nonempty (??). And for $\winpos{j + 1}(\strct{A}, \strct{B})$ there must be a maximal (in the sense of set inclusion) partial isomorphism $q$ in it, i.e.\ there is $q \in \winpos{j + 1}(\strct{A}, \strct{B})$ such that there is no $q' \in \winpos{j + 1}(\strct{A}, \strct{B})$ with $q' \supset q$; otherwise there would be an isomorphism from $\strct{A}$ to $\strct{B}$ (??). By definition of $\winpos{j + 1}(\strct{A}, \strct{B})$ there must be an $a \in A$ so that $q$ can be extended to a $p \in \winpos{j}(\strct{A}, \strct{B})$ with $\dm(p) = \dm(q) \union \sete{a}$ and moreover this $a$ can be chosen not in $\dm(q)$; if there is no such $a$ then we would have that $q : \strct{A} \isom \strct{B}$ or $\strct{A} \isom_j \strct{B}$ (??). Note that $p \notin \winpos{j + 1}(\strct{A}, \strct{B})$ which would otherwise contradict the maximality of $q \in \winpos{j + 1}(\strct{A}, \strct{B})$; thereby we have $\winpos{j}(\strct{A}, \strct{B}) \neq \winpos{j + 1}(\strct{A}, \strct{B})$. To obtain that $\winpos{j}(\strct{A}, \strct{B}) \supseteq \winpos{j + 1}(\strct{A}, \strct{B})$, we use ??. Finally, it is clear that $\winpos{m + 1}(\strct{A}, \strct{B}) = \emptyset = \winpos{\infty}(\strct{A}, \strct{B})$. Thus, set $m_0 \defas m + 1$.
%
\end{enumerate}
%end of section 2-----------------------------------------------------------------------------
\setcounter{chapter}{3}
\chapter{0-1 Laws}
%section 1------------------------------------------------------------------------------------
\begin{enumerate}[1.]
%
\item \header{Note on the Proof of Lemma 4.1.2} Denote
\[
\psi \defas \blor_{\varphi \in \Phi} \neg\varphi \lor \blor_{\varphi \in \cmpl{\Phi}} \varphi
\]
and let $\strct{A}$ with domain $A = \sete{1, \etc, n}$ be a random structure. Then
\[
\begin{array}{ll}
\    & \lprob[n](\neg\extaxm[\Phi]) \cr
=    & \paren{\mbox{the probability that \mathmode{\strct{A} \satis \neg\extaxm[\Phi]}}} \cr
=    & \paren{\parbox{40em}{the probability that there is an injective \mathmode{f : \sete{1, \etc, r} \to A} such that for every \mathmode{a \in A \setminus \rg(f)}, \mathmode{\strct{A} \satis \psi[f(1), \etc, f(r), a]}}} \cr
\leq & \sum\limits_{\text{injective \mathmode{f : \sete{1, \etc, r} \to A}}} \paren{\mbox{the probability that for every \mathmode{a \in A \setminus \rg(f)}, \mathmode{\strct{A} \satis \psi[f(1), \etc, f(r), a]}}} \cr
=    & \prmt{n}{r} \mul \paren{\mbox{the probability that for every \mathmode{a \in A \setminus \rg(f)} given injective \mathmode{f}, \mathmode{\strct{A} \satis \psi[f(1), \etc, f(r), a]}}} \cr
=    & \prmt{n}{r} \mul \paren{\parbox{35em}{the probability that for every \mathmode{a \in A \setminus \rg(f)} given injective \mathmode{f}, the assignment \mathmode{f(1), \etc, f(r), a} does not satisfy in \mathmode{\strct{A}} the type characterized by \mathmode{\Phi}}} \cr
=    & \prmt{n}{r} \mul \prod\limits_{\text{\mathmode{a \in A \setminus \rg(f)} given injective \mathmode{f}}} \paren{\parbox{25em}{the probability that the assignment \mathmode{f(1), \etc, f(r), a} does not satisfy in \mathmode{\strct{A}} the type characterized by \mathmode{\Phi}}} \cr
=    & \prmt{n}{r} \mul (\frac{c - 1}{c})^{n - r} \cr
\leq & n^r \mul (\frac{c - 1}{c})^{n - r}. \cr
\end{array}
\]
%
\item \header{Hint to Exercise 4.1.6} By 2.2.6, the set
\[
\Phi \defas \sett{\hint{m}{\strct{C}}}{\mathmode{\strct{C}} is a structure}
\]
is finite. Thus, for $s \geq 1$, the binary relation $\isom_m$ induces a finite partition of $\lclass{s}(\tau)$ due to the equivalence between \refitem{(iii)} and \refitem{(iv)} of 2.3.4. Also by 4.1.5 the 0-1 law of $\folog\of{\tau}$, there is exactly one sentence $\varphi_0 \in \Phi$ with $\lprob(\varphi_0) = 1$ because
\[
1 = \lprob(\bunion_{\varphi \in \Phi} \modclass(\varphi)) = \sum_{\varphi \in \Phi} \lprob(\varphi).
\]
Finally, observe that the probability $\strct{A} \isom_m \strct{B}$ is $\geq$ the probability that both $\strct{A} \satis \varphi_0$ and $\strct{B} \satis \varphi_0$.
%
\item \header{Solution to Exercise 4.1.7} Let $\varphi \defas \extaxm[\Phi]$ be an $(r + 1)$-extension axiom, where $\Phi$ is a subset of $\Delta_{r + 1}$ as defined in 3.2.11. Assume that $\card{\Delta_{r + 1}} = c$ and $\card{\Phi} = k$.

Let $\strct{A}$ with domain $A = \sete{1, \etc, n}$ be a random structure constructed with a biased coin so that $R i_1 \etc i_m$ holds with probability $p$ where $0 < p < 1$. Then
\[
\begin{array}{ll}
\    & \lprob[n](\neg\extaxm[\Phi]) \cr
=    & \paren{\mbox{the probability that \mathmode{\strct{A} \satis \neg\extaxm[\Phi]}}} \cr
=    & \paren{\parbox{40em}{the probability that there is an injective \mathmode{f : \sete{1, \etc, r} \to A} such that for every \mathmode{a \in A \setminus \rg(f)}, \mathmode{\strct{A} \satis \psi[f(1), \etc, f(r), a]}}} \cr
\leq & \sum\limits_{\text{injective \mathmode{f : \sete{1, \etc, r} \to A}}} \paren{\mbox{the probability that for every \mathmode{a \in A \setminus \rg(f)}, \mathmode{\strct{A} \satis \psi[f(1), \etc, f(r), a]}}} \cr
=    & \prmt{n}{r} \mul \paren{\mbox{the probability that for every \mathmode{a \in A \setminus \rg(f)} given injective \mathmode{f}, \mathmode{\strct{A} \satis \psi[f(1), \etc, f(r), a]}}} \cr
=    & \prmt{n}{r} \mul \paren{\parbox{35em}{the probability that for every \mathmode{a \in A \setminus \rg(f)} given injective \mathmode{f}, the assignment \mathmode{f(1), \etc, f(r), a} does not satisfy in \mathmode{\strct{A}} the type characterized by \mathmode{\Phi}}} \cr
=    & \prmt{n}{r} \mul \prod\limits_{\text{\mathmode{a \in A \setminus \rg(f)} given injective \mathmode{f}}} \paren{\parbox{25em}{the probability that the assignment \mathmode{f(1), \etc, f(r), a} does not satisfy in \mathmode{\strct{A}} the type characterized by \mathmode{\Phi}}} \cr
=    & \prmt{n}{r} \mul (1 - p^k(1 - p)^{c - k})^{n - r} \cr
\leq & n^r \mul (1 - p^k(1 - p)^{c - k})^{n - r}. \cr
\end{array}
\]
It follows that $\lprob(\neg\extaxm[\Phi]) = \lim_{n \to \infty} \lprob[n](\neg\extaxm[\Phi]) = 0$.
%
\item \header{Solution to Exercise 4.1.8} Let $\strct{R}'$ denote the random structure defined as in the description of the exercise. Note that $\strct{R}'$ is a \emph{random variable}. In the following for $n > 0$ and for any (finite or infinite) structure $\strct{A}$ we let $\strct{A}_n$ be the substructure of $\strct{A}$ induced by the domain $\sete{1, \etc, n}$, provided that the domain $A$ of $\strct{A}$ contains $\sete{1, \etc, n}$ as a subset.

Observe that there are finitely many $\tau$-structures with domain $\sete{1, \etc, n}$, i.e.\ the class $\lclass{n}(\tau)$ is finite. Since $\tau$ is relational, the structures $\strct{B} \in \lclass{n}(\tau)$ altogether induce an \emph{equal partition} of
\[
C \defas \sett{\strct{A}}{\mathmode{\strct{A}} is a \mathmode{\tau}-structure of domain \mathmode{A = \sete{1, 2, \etc}}}
\]
such that the equivalence classes
\[
\setm{\strct{A} \in C}{\strct{A}_n = \strct{B}}
\]
for the structures $\strct{B}$ have the same size. Thus, if $K$ is a class of (finite) $\tau$-structures, then
\begin{equation}\label{e4_1_8+1}
(\text{the probability that for the random structure \mathmode{\strct{R}'}, its substructure \mathmode{\strct{R}'_n} is in \mathmode{K}}) = \lprob[n](K). \tag{$+$}
\end{equation}

Now consider any $(r + 1)$-extension axiom $\chi$ so that $\neg\chi$ is logically equivalent to
\[
\exists v_1 \etc \exists v_r \rho,
\]
where
\[
\rho = (\bland_{1 \leq i < j \leq r} \neg v_i = v_j \land \forall v_{r + 1} (\bland_{1 \leq i \leq r} \neg v_{r + 1} = v_i \lthen (\blor_{\varphi \in \Phi} \neg\varphi \lor \blor_{\varphi \in \cmpl{\Phi}} \varphi)))
\]
and $\Phi \subseteq \Delta_{r + 1}$. For every $r$-tuple $\vect{i} = (i_1, \etc, i_r) \in \cartpwr{\posint}{r}$ of positive integers, let
\[
m(\vect{i}) \defas
\begin{cases}
\max(i_1, \etc, i_r) & \text{if \mathmode{r > 0}} \cr
1                    & \text{otherwise (i.e.\ \mathmode{\vect{i} = \emptyseq})} \cr
\end{cases}
\]
and further for every positive integer $j \in \posint$ let $E^{\vect{i}}_j$ be the event that $\strct{R}'_{m(\vect{i}) + j} \satis \rho[\vect{i}]$. Then we have
\[
E^{\vect{i}}_1 \supseteq E^{\vect{i}}_2 \supseteq \etc
\]
and hence
\begin{equation}\label{e4_1_8+2}
(\text{the probability for the event \mathmode{\bintsc^\infty_{j = 1} E^{\vect{i}}_j}}) = \lim_{j \to \infty} (\text{the probability for the event \mathmode{E^{\vect{i}}_j}}). \tag{$\ast$}
\end{equation}
Moreover, it is true that
\[
\begin{array}{lll}
\    & (\text{the probability for the event \mathmode{E^{\vect{i}}_j}}) & \cr
\leq & (\text{the probability for the event that \mathmode{\strct{R}'_{m(\vect{i}) + j} \satis \exists v_1 \etc \exists v_r \rho}}) & \cr
=    & (\text{the probability for the event that \mathmode{\strct{R}'_{m(\vect{i}) + j} \satis \neg\chi}}) & \cr
=    & \lprob[m(\vect{i}) + j](\neg\chi) & \text{(by (\ref{e4_1_8+1}))} \cr
\end{array} 
\]
and hence
\begin{equation}\label{e4_1_8+3}
\lim_{j \to \infty} (\text{the probability for the event \mathmode{E^{\vect{i}}_j}}) \leq \lim_{j \to \infty} \lprob[m(\vect{i}) + j](\neg\chi) = 0 \tag{$\ast\ast$}
\end{equation}
by (the proof of) Lemma 4.1.2.

Finally, for brevity, for every $\vect{i} \in \cartpwr{\posint}{r}$ let $F^{\vect{i}}$ be the event that $\strct{R}' \satis \rho[\vect{i}]$; then we have
\begin{equation}\label{e4_1_8+4}
F^{\vect{i}} = \bintsc^\infty_{j = 1} E^{\vect{i}}_j. \tag{$\ast\ast\ast$}
\end{equation}
It follows that
\[
\begin{array}{lll}
\    & (\text{the probability that \mathmode{\strct{R}' \satis \neg\chi}}) & \cr
=    & (\text{the probability for the event \mathmode{\bunion_{\vect{i} \in \cartpwr{\posint}{r}} F^{\vect{i}}}}) & \cr
\leq & \sum_{\vect{i} \in \cartpwr{\posint}{r}} (\text{the probability for the event \mathmode{F^{\vect{i}}}}) & \cr
=    & \sum_{\vect{i} \in \cartpwr{\posint}{r}} (\text{the probability for the event \mathmode{\bintsc^\infty_{j = 1} E^{\vect{i}}_j}}) & \text{(by (\ref{e4_1_8+4}))} \cr
=    & \sum_{\vect{i} \in \cartpwr{\posint}{r}} \lim_{j \to \infty} (\text{the probability for the event \mathmode{E^{\vect{i}}_j}}) & \text{(by (\ref{e4_1_8+2}))} \cr
\leq & \sum_{\vect{i} \in \cartpwr{\posint}{r}} 0 & \text{(by (\ref{e4_1_8+3}))} \cr
=    & 0.
\end{array}
\]

Therefore, we conclude that, with probability $1$, $\strct{R}'$ is a model of any extension axiom $\chi$ and hence of $\randstrtheory$.

\begin{note}
The \emph{values} of the random variable $\strct{R}'$ for which $\strct{R}'$ is a model of $\randstrtheory$ are exactly those $\tau$-structures that are isomorphic to the infinite random structure $\infrandstr$.
\end{note}
%
\item \header{Hint to Exercise 4.1.9} (INCOMPLETE)
%
\end{enumerate}
%end of section 1-----------------------------------------------------------------------------


%section 2------------------------------------------------------------------------------------
\setcounter{section}{1}
\section{Parametric Classes}
\begin{enumerate}[1.]
%
\item \header{Note on Definition 4.2.1 and the Arguments Leading to Theorem 4.2.3} A parametric sentence may be a conjunction of sentences that involve different numbers of variables, in which the boolean combinations may contain different relation symbols. For example,
\[
\forall x Rxx \land \forall \distinct x y (\neg Rxy \lor Tyxy)
\]
is a parametric sentence.

In the construction process leading to statement (1) before 4.2.3, notice that whether or not an $s$-tuple $b_1, \etc, b_s$ is in a relation $\intpr{R}{B}$ is independent of other tuples (not necessarily having length $s$) due to the fact that every atomic subformula of $\varphi_0$ contains exactly the variables that appear in its corresponding quantification and that every conjunct of $\varphi_0$ is a universal formula. This explains why the process works for parametric classes and why it may not work for classes of transitive properties (hence those properties are excluded in the consideration of parametric classes).

To obtain (2), let $\strct{A}, \strct{B}$ be two models of $\randstrtheory(\varphi_0)$ and choose $I$ as in 3.2.11. Obviously $I \neq \emptyset$ as $\emptymap \mapsto \emptymap \in I$ since $\strct{A} \satis \tr \land \neg\fls$ and $\strct{B} \satis \tr \land \neg\fls$. For the forth property, let $\vect{a} \mapsto \vect{b} \in I$ and $\vect{a} = a_1, \etc, a_r$ and $\vect{b} = b_1, \etc, b_r$ each consist of distinct entries. Let $a_{r + 1} \in A$ be different from entries of $\vect{a}$ and consider $\Phi \subseteq \Delta_{r + 1}$ such that
\[
\strct{A} \satis (\bland_{\varphi \in \Phi} \varphi \land \bland_{\varphi \in \cmpl{\Phi}} \neg\varphi)\withassgn{\vect{a}a_{r + 1}}.
\]
It follows that $\strct{A} \satis \exists \distinct v_1 \etc v_rv_{r + 1} (\bland_{\varphi \in \Phi} \varphi \land \bland_{\varphi \in \cmpl{\Phi}} \neg\varphi)$ and hence $\extaxm[\Phi]$ is compatible with $\varphi_0$, which gives $\extaxm[\Phi] \in \randstrtheory(\varphi_0)$. As $\strct{B} \satis \randstrtheory(\varphi_0)$, there is a $b_{r + 1} \in B$ different from the entries of $\vect{b}$ such that
\[
\strct{B} \satis (\bland_{\varphi \in \Phi} \varphi \land \bland_{\varphi \in \cmpl{\Phi}} \neg\varphi)\withassgn{\vect{b}b_{r + 1}}.
\]
Therefore, $\vect{a}a_{r + 1} \mapsto \vect{b}b_{r + 1} \in I$. Similarly for the back property.

On the other hand, for every $r \geq 0$ the theory $\randstrtheory(\varphi_0)$ contains at least one $(r + 1)$-extension axiom: By (1) there is a model $\strct{A}$ of $\varphi_0$ of cardinality $r + 1$, say, having domain $A = \sete{a_1, \etc, a_{r + 1}}$. Let $\Phi \subseteq \Delta_{r + 1}$ such that
\[
\strct{A} \satis (\bland_{\varphi \in \Phi} \varphi \land \bland_{\varphi \in \cmpl{\Phi}} \neg\varphi) \withassgn{a_1, \etc, a_{r + 1}}.
\]
We then have $\strct{A} \satis \exists \distinct v_1 \etc v_{r + 1} (\bland_{\varphi \in \Phi} \varphi \land \bland_{\varphi \in \cmpl{\Phi}} \neg\varphi)$ and hence the $(r + 1)$-extension axiom $\extaxm[\Phi]$ is compatible with $\varphi_0$ and is in $\randstrtheory(\varphi_0)$.

As for (3), a slight modification can be made to the construction procedure for the countable model of $\randstrtheory$ in 3.2.11 to obtain one for $\randstrtheory(\varphi_0)$: Let $\seq{\alpha_n}{n \geq 0}$ be as in 3.2.11 except that now we only consider extension axioms $\extaxm \in \randstrtheory(\varphi_0)$. We choose $\strct{A}_0$ with $A_0 = \sete{0}$ so that $\strct{A}_0 \satis \varphi_0$ according to (1). Suppose that $\strct{A}_n$ has been defined such that $\strct{A}_n \satis \varphi_0$. For $\alpha_n = (\vect{m}, \extaxm)$ where $\vect{m} = m_1, \etc, m_r$ and
\[
\extaxm \defas \forall \distinct v_1 \etc v_r \exists v_{r + 1} (\bland_{1 \leq i \leq r} \neg v_{r + 1} = v_i \land \bland_{\varphi \in \Phi} \varphi \land \bland_{\varphi \in \cmpl{\Phi}} \neg\varphi),
\]
define the relationship between $n + 1$ and $\vect{m}$ in $\strct{A}_{n + 1}$ as in 3.2.11 so that
\[
\strct{A}_{n + 1} \satis (\bland_{\varphi \in \Phi} \varphi \land \bland_{\varphi \in \cmpl{\Phi}} \neg\varphi) \withassgn{m_1, \etc, m_r, n + 1};
\]
and, in case $k \geq 2$, for every $2 \leq s \leq k$ and every $s$-tuple $\vect{m}'$ of distinct entries that contains $n + 1$ and an entry not in $\vect{m}$, define the ``shape'' of $\vect{m}'$ in $\strct{A}_{n + 1}$ in the same way as in the construction process leading to (1). To verify $\strct{A}_{n + 1} \satis \varphi_0$, it remains to show the substructure $\strct{A}'$ of $\strct{A}_{n + 1}$ with domain $A' = \sete{m_1, \etc, m_r, n + 1}$ is a model of $\varphi_0$: Since $\extaxm \in \randstrtheory(\varphi_0)$ for the extension axiom $\extaxm$ in $\alpha_n$, we have $\extaxm$ is compatible with $\varphi_0$ and thus
\[
\varphi_0 \land \exists \distinct v_1 \etc v_r v_{r + 1} (\bland_{\varphi \in \Phi} \varphi \land \bland_{\varphi \in \cmpl{\Phi}} \neg\varphi)
\]
is satisfiable, say, by $\strct{B}$. Suppose $b_1, \etc, b_r, b_{r + 1}$ are elements in $B$ such that
\[
\strct{B} \satis (\bland_{\varphi \in \Phi} \varphi \land \bland_{\varphi \in \cmpl{\Phi}} \neg\varphi) \withassgn{b_1, \etc, b_r, b_{r + 1}},
\]
then we have that the substructure $\strct{B}'$ of $\strct{B}$ with domain $B' = \sete{b_1, \etc, b_r, b_{r + 1}}$ is a model of $\varphi_0$ (since $\varphi_0$ is a conjunction of universal sentences) and $\strct{A}'$ and $\strct{B}'$ are isomorphic. We conclude that $\strct{A}' \satis \varphi_0$.
%
\end{enumerate}
%end of section 2-----------------------------------------------------------------------------


%section 5------------------------------------------------------------------------------------
\setcounter{section}{4}
\section{Probabilities of Monadic Second Order Properties}
\begin{enumerate}[1.]
%
\item \header{Note on the Proof of Lemma 4.5.3} From the proof of Claim 1, it looks like either $\psi^Y_<(x, y)$, $\psi^Z_<(x, y)$ and $\psi^U_<(x, y)$ mean $y <^Y x$, $y <^Z x$ and $y <^U x$, respectively or the statement on the right-hand side of ($\ast$) should be $a \geq b$.

There is a typo on line 4 of page 92: ``$i \in \sete{1, \etc, n}$'' should be ``$i \in \sete{1, \etc, t}$''. In fact, the probability $q$ that a given $i \in \sete{1, \etc, t}$ satisfies ($\ast$) is $p^{\frac{1}{2}r(r + 1)}(1 - p)^{\frac{1}{2}r(r - 1)}$.
%
\end{enumerate}
%end of section 5-----------------------------------------------------------------------------




\begin{thebibliography}{10}
\bibitem{EFT} H.-D.\ Ebbinghaus, J.\ Flum and W.\ Thomas \textsl{Mathematical Logic} 2nd edition, Springer, 1995.
%
\end{thebibliography}
\end{document}