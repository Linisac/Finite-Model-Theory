%Author: Wei-Lin (Linisac) Wu
\documentclass[11pt, leqno]{report}


%Packages
\usepackage{amssymb}
\usepackage{enumerate}
\usepackage{graphicx}
\usepackage{stmaryrd}
\usepackage{colonequals}
\usepackage{amsthm}
\usepackage{array}
\usepackage{amsmath}
\usepackage{amscd}
\usepackage{paralist}
\usepackage{bm}
\usepackage{titlesec}
\usepackage[margin=2cm]{geometry}

%ifthen package
\usepackage{ifthen}


%New Commands for Abbreviations
%%Catalog <
%%%general
%%%definition
%%%enumeration, sequence, vector (tuple), support
%%%set operations
%%%function type, function operations
%%%special set(s)
%%%structure, symbol interpretation, structure operations, structure relations
%%%partial isomorphism
%%%elementary equivalence
%%%Ehrenfeucht-Fraisse game
%%%isomorphism types
%%%graph, graph functions, graph relations
%%%logic
%%%counting specifiers
%%%vocabulary, connectives, quantifiers, formula operations
%%%substitution
%%%logic relations
%%%logic operations
%%%finite notions
%%%nullary relations and predicates
%%%operations on formulas and structures
%%%global relations
%%%special classes of structures
%%%extension axioms
%%%labeled/unlabeled classes and probabilities
%% >

%%general <
\newcommand{\header}[1]{{\rm\textbf{#1.}}}
\newcommand{\refitem}[1]{{\rm#1}}
\newcommand{\mathmode}[1]{\begin{math}#1\end{math}}
\newcommand{\paren}[1]{\left(#1\right)} %flexible parentheses
\newcommand{\of}[1]{{[#1]}} %logic or class with vocabulary #1
\newcommand{\prmt}[2]{\mathrm{P}^{#1}_{#2}} %permutation (in combinatorics)
\newcommand{\cmbn}[2]{\mathrm{C}^{#1}_{#2}} %combination (in combinatorics)
%% >

%%definition <
\newcommand{\defas}{\colonequals}
%% >

%%enumeration, sequence, vector (tuple), support <
\newcommand{\etc}{\ldots} %et cetera
\newcommand{\length}{\mathrm{length}}
\newcommand{\emptyseq}{\emptyset} %empty sequence
\newcommand{\seq}[2]{(#1)_{#2}} %indexed sequence
\newcommand{\vect}[1]{\overline{#1}} %vector, \vect{a} = \overbar{a}
\newcommand{\supp}{\mathrm{supp}}
%% >

%%some common operators <
\newcommand{\mul}{\mathop{\cdot}} %multiplication
\newcommand{\abs}[1]{\left|#1\right|}
%% >

%%set operations <
\newcommand{\sete}[1]{{\{#1\}}} %set by enumeration. sete{a, b, c} = {a, b, c}
\newcommand{\setm}[2]{{\{#1 \mid #2\}}} %set by math description. setd{#1}{#2} = {#1 | #2}
\newcommand{\sett}[2]{{\{#1 \mid \text{\rm #2}\}}} %set by text description. sett{#1}{#2} = {#1 | #2}, where #2 is a text
\newcommand{\cart}{\mathop{\times}} %cartesian product of two sets
\newcommand{\cartpwr}[2]{#1^{#2}} %(#2)nd cartesian power of set #1
\newcommand{\intsc}{\cap} %intersection
\newcommand{\union}{\cup} %union
\newcommand{\bintsc}{\bigcap} %big intersection
\newcommand{\bunion}{\bigcup} %big union
\newcommand{\card}[1]{\|#1\|} %cardinality. card{A} = || A ||
\newcommand{\cmpl}[1]{#1^\mathit{c}} %complement of the set #1
%% >

%%function operations <
\newcommand{\dm}{\mathrm{do}} %domain (of a function)
\newcommand{\rg}{\mathrm{rg}} %range (of a function)
\newcommand{\emptymap}{\emptyset} %empty map
\newcommand{\inv}[1]{#1^{-1}} %inverse of function #1
%% >

%%special set(s) <
\newcommand{\nat}{\mathbb{N}} %set of natural numbers
\newcommand{\rand}{\mathrm{rand}} %random
\newcommand{\spec}{\mathrm{Spec}} %spectrum
%% >

%%structure, symbol interpretation, structure operations, structure relations <
\newcommand{\strct}[1]{\mathcal{#1}} %structure
\newcommand{\intpr}[2]{#1^{#2}} %symbol #1 interpreted under #2
\newcommand{\sbstrct}[2]{#1^{#2}} %substructure of #2 induced by the subset #1
\newcommand{\isom}[1][]{\ifthenelse{\equal{#1}{}}{\cong}{\cong^{#1}}} %isomorphic in #1 number of variables
\newcommand{\product}{\cart} %product of two structures
\newcommand{\dunion}{\mathop{\dot{\cup}}} %disjoint union of two structures
\newcommand{\ordsum}{\mathop{\lhd}} %ordered sum of two ordered structures
\newcommand{\ultprd}[2]{#1^{#2}} %ultra product. \ultprd{A}{I} = A^I
\newcommand{\ball}[1][]{\ifthenelse{\equal{#1}{}}{{\mathit{S}}}{{\mathit{S}^{#1}}}}
\newcommand{\ballstrct}[1][]{\ifthenelse{\equal{#1}{}}{{\mathcal{S}}}{{\mathcal{S}^{#1}}}}
%% >

%%partial isomorphism <
\newcommand{\partisoms}{\mathrm{Part}} %set of partial isomorphisms between structures #1 and #2
\newcommand{\partially}{\mathrm{part}} %partially (isomorphic)
%% >

%%elementary equivalence <
\newcommand{\equv}[1][]{\ifthenelse{\equal{#1}{}}{\equiv}{\equiv^{#1}}} %equivalence in (#1)-logic
%% >

%%Ehrenfeucht-Fraisse games <
\newcommand{\efgame}{\mathrm{G}} %Ehrenfeucht-Fraisse game
\newcommand{\fingame}{\efgame} %finite Ehrenfeucht-Fraisse game
\newcommand{\game}[2][]{\ifthenelse{\equal{#1}{}}{{\efgame_{#2}}}{{\efgame^{#1}_{#2}}}} %the game G^#1_#2
\newcommand{\winpos}[2][]{\ifthenelse{\equal{#1}{}}{\mathit{W}_{#2}}{\mathit{W}^{#1}_{#2}}}
\newcommand{\msogame}[1]{\operatorname{\MSO-\mathrm{G}}_{#1}} %the game in monadic second-order logic
%% >

%%isomorphism types <
\newcommand{\hint}[2]{\varphi^{#1}_{#2}} %(#1)-isomorphic type (Hintikka formula) over #2 in first-order logic
\newcommand{\ityp}[3][]{\ifthenelse{\equal{#1}{}}{{\psi^{#2}_{#3}}}{{^{#1}\psi^{#2}_{#3}}}} %(#2)-isomorphism type over #2 in other logics in #1 number of variables
%% >

%%graph, graph functions, graph relations <
\newcommand{\graph}{\mathrm{GRAPH}} %the set of finite graphs
\newcommand{\conn}{\mathrm{CONN}} %the set of connected graphs
\newcommand{\dist}[1][]{\ifthenelse{\equal{#1}{}}{\mathit{d}}{\mathit{d}_{#1}}} %distance between two points in a graph
\newcommand{\tree}{\mathrm{TREE}} %the set of finite trees
\newcommand{\gaifman}{\mathcal{G}} %the Gaifman graph of #1 (structure)
%% >

%%logics <
\newcommand{\folog}[1][]{\ifthenelse{\equal{#1}{}}{\mathrm{FO}}{\mathrm{FO}^{#1}}} %first-order logic
\newcommand{\solog}{\mathrm{SO}} %second-order logic
\newcommand{\msolog}{\mathrm{MSO}} %monadic second-order logic
\newcommand{\inflog}[2][]{\ifthenelse{\equal{#1}{}}{\mathrm{L}_{#2\omega}}{\mathrm{L}^{#1}_{#2\omega}}} %infinitary logic parameterized by the number of distinct variables (#1) and the upper bound on the sizes of sets over which disjunctions or conjunctions can be taken (#2), the upper bound on the lengths of strings of quantifiers is fixed, i.e. \omega
%\newcommand{\logic}[2][]{\ifthenelse{\equal{#1}{}}{\mathrm{L}_{#2}}{\mathrm{L}^{#1}_{#2}}} %parameterized logic (like infinitary logics)
\newcommand{\logic}{\mathcal{L}} %(general) logic
%% >

%%counting specifiers <
\newcommand{\atleast}[1]{\geq #1} %at least #1
\newcommand{\atmost}[1]{\leq #1} %at most #1
\newcommand{\exactly}[1]{= #1} %exactly #1
\newcommand{\cardatleast}[1]{\varphi_{\atleast{#1}}} %cardinality at least #1
\newcommand{\cardatmost}[1]{\varphi_{\atmost{#1}}} %cardinality at most #1
\newcommand{\cardexactly}[1]{\varphi_{\exactly{#1}}} %cardinality exactly #1
%% >

%%vocabulary, connectives, quantifiers, formula operations <
\newcommand{\emptyvoc}{\emptyset} %empty vocabulary
\newcommand{\lthen}{\rightarrow} %logicial then
\newcommand{\liff}{\leftrightarrow} %logical iff (between two formulas)
\newcommand{\blor}{\bigvee} %big disjunction
\newcommand{\bland}{\bigwedge} %big conjunction
\newcommand{\existatleast}[1]{\exists^{\atleast{#1}}} %exist at least #1
\newcommand{\existatmost}[1]{\exists^{\atmost{#1}}} %exist at most #1
\newcommand{\existexactly}[1]{\exists^{\exactly{#1}}} %exist exactly #1
\newcommand{\free}{\mathrm{free}} %set of free variables of #1
\newcommand{\qr}{\mathrm{qr}} %quantifier rank (of a formula)
\newcommand{\modclass}{\mathrm{Mod}} %model class (of a formula)
%% >

%%substitution <
\newcommand{\sbst}[2]{{\scriptstyle\frac{#1}{#2}}} %substitution of #2 with #1
%% >
\newenvironment{perm}{\left(\begin{array}}{\end{array}\right)} %permutation of variables

%%logic relations <
\newcommand{\satis}{\models} %satisfaction relation
\newcommand{\consq}{\models} %consequence relation
%% >

%%logic operations <
\newcommand{\rel}[1]{#1^\mathit{r}} %#1 relationalized
\newcommand{\invrel}[1]{#1^{-\mathit{r}}} %#1 inverse-relationalized
%% >

%%finite notions <
\newcommand{\fin}{\mathrm{fin}} %finite
\newcommand{\consqfin}{\consq_\fin} %consequence relation restricted to finite structures
%% >

%%nullary relations and predicates <
\newcommand{\true}{\mathrm{TRUE}} %TRUE symbol
\newcommand{\false}{\mathrm{FALSE}} %FALSE symbol
\newcommand{\tr}{\mathrm{T}} %T symbol
\newcommand{\fls}{\mathrm{F}} %F symbol
%% >

%%operations on formulas and structures <
\newcommand{\tuplesby}[2]{{#1^{#2}(\_)}} %tuples defined by formula #1 and structure #2
\newcommand{\rltv}[2]{#1^{#2}} %formula #1 relativized to the set #2
%% >

%%global relations <
\newcommand{\transcls}{\mathrm{TC}} %transitive closure relation
%% >

%%special classes of structures <
\newcommand{\ordclass}{\mathcal{O}} %class of finite ordered structures (in a vocabulary)
\newcommand{\even}{\mathrm{EVEN}} %class of finite structures of even cardinality (in a vocabulary)
%% >

%%extension axioms <
\newcommand{\extaxm}[1][]{\ifthenelse{\equal{#1}{}}{\chi}{\chi_{#1}}} %extension axiom with optional parameter #1 (set of formulas)
\newcommand{\conjextaxm}[1]{\epsilon_{#1}} %conjunction of extension axioms with variables indexed no more than #1
%% >

%%labeled/unlabeled classes and probabilities <
\newcommand{\lclass}[1]{\mathit{L}_{#1}} %labeled class of strcutures
\newcommand{\lprob}[1][]{\ifthenelse{\equal{#1}{}}{\mathit{l}}{\mathit{l}_{#1}}} %labeled probability over the class of structures of cardinality #1 or labeld asymptotic probability
\newcommand{\uclass}[1]{\mathit{U}_{#1}} %unlabeled class of strcutures
\newcommand{\uprob}[1][]{\ifthenelse{\equal{#1}{}}{\mathit{u}}{\mathit{u}_{#1}}} %unlabeled probability over the class of structures of cardinality #1 or unlabeld asymptotic probability
%% >


\newenvironment{quoteno}[1]{(#1)\hfill}{\hfill\phantom{(+)}}


%if resetting of equation counter is required, use the instruction below:
%\setcounter{equation}{0}


\theoremstyle{remark}
\newtheorem*{remark}{Remark}

\theoremstyle{note}
\newtheorem*{note}{Note}

\begin{document}
%\begin{titlepage}
%\noindent\textsc{\LARGE Annotations to ``Finite Model Theory''}\bigskip\\
%\textsc{\large with Solutions Hints}
%\end{titlepage}
\titleformat{\chapter}[hang]{\Large\bfseries}{\thechapter.}{.5em}{}
\titleformat{\section}[hang]{\large\bfseries}{\thesection}{.5em}{}
%%\setcounter{chapter}{1}
\chapter{Preliminaries}
%Paragraph A----------------------------------------------------------------------------------
\paragraph{A Structures}
\begin{enumerate}[1.]
%
\item \header{Cycles Have Length at Least $2$}
%
\item \header{An Example of $\strct{A} \ordsum \strct{B} \not\isom \strct{B} \ordsum \strct{A}$} Let $\tau \defas \sete{{<}, P}$ in which $P$ is unary, $\strct{A} = (\sete{a}, \emptyset, \emptyset), \strct{B} = (\sete{b}, \emptyset, \sete{b})$.
%
\item \header{Note on the Last Paragraph of A3 Operations on Structures} In fact, $\strct{A} \dsjuni \strct{B}$ and $\strct{B} \dsjuni \strct{A}$ are not only isomorphic but also \emph{identical} since $A \union B = B \union A$ and $\intpr{R}{\strct{A}} \union \intpr{R}{\strct{B}} = \intpr{R}{\strct{B}} \union \intpr{R}{\strct{A}}$ for $R \in \tau$.
%
\end{enumerate}
%End of Paragraph A---------------------------------------------------------------------------
%Paragraph C----------------------------------------------------------------------------------
\paragraph{C Some Classical Results of First-Order Logic}
\begin{enumerate}[1.]
%
\item \header{Note on the Proof of Lemma 1.0.6} It is easy to drop the assumption that $\varphi$ is a sentence: All variables occurring in $\varphi$, if any, can be replaced by new constant symbols, i.e.\ constants not in $\tau$.
%
\end{enumerate}
%End of Paragraph C---------------------------------------------------------------------------
%Paragraph D----------------------------------------------------------------------------------
\paragraph{D Model Classes and Global Relations}
\begin{enumerate}[1.]
%
\item \header{Note on Sets of Tuples Defined by Formulas} One can infer by context that in the definition of $\tuplesby{\varphi}{\strct{A}}$ in which $\varphi = \varphi(x_1, \etc, x_n)$, the free variables in $\varphi$ are among $x_1, \etc, x_n$.
%
\end{enumerate}
%End of Paragraph D-----------------------------------------------------------------------------
%%\setcounter{chapter}{1}
\chapter{The Ehrenfeucht-Fra\"{i}ss\'{e} Method}
\setcounter{section}{1}
%section 2------------------------------------------------------------------------------------
\section{Ehrenfeucht's Theorem}
\begin{enumerate}[1.]
%
\item \header{Note on Definition 2.2.1} The condition for a map $p$ to be a partial isomorphism between $\strct{A}$ and $\strct{B}$ is slightly different in \cite{EFT}: the designated constants $\intpr{c}{A}$ are not required to be in $\dom{p}$ but if they are, the condition $p(\intpr{c}{A}) = \intpr{c}{B}$ must be satisfied.
%
\item \header{Note on Remarks 2.2.2} For part (a), note that the empty map $p = \emptyset$ is \emph{not} a partial isomorphism if $\tau$ does contain constants, by Definition 2.2.1.
\medskip\\
For part (c), the notation $\vect{a} \mapsto \vect{b}$ can be seen as a shorthand for the set
\[
\setm{(a_i, b_i)}{i \leq s} \union \setm{(\intpr{c}{\strct{A}}, \intpr{c}{\strct{B}})}{c \in \tau},\]
which may not define a map, nor a partial isomorphism. If the above set does define a map, then it is the graph of this map. Statement (i) says $\vect{a} \mapsto \vect{b}$ not only defines a map but even a partial isomorphism between $\strct{A}$ and $\strct{B}$.
\medskip\\
Also note that if $\tau$ contains constants then the notation $\vect{a} \mapsto \vect{b}$ contains mapping of the constants, which are \emph{hidden} from it; in this case, $\emptyseq \mapsto \emptyseq$ does not equal the empty map $\emptymap$. $\emptyseq \mapsto \emptyseq = \emptymap$ when $\tau$ contains no constants.
%
\item \header{Note on Ehrenfeucht Games.} It is clear from definition that at most one of the players has a winning strategy for the game $\game{m}{\strct{A}, \vect{a}, \strct{B}, \vect{b}}$.
\medskip\\
Moreover, it is also true that at least one of the players has a winning strategy for the game $\game{m}{\strct{A}, \vect{a}, \strct{B}, \vect{b}}$, i.e.\ \emph{if the duplicator does not win $\game{m}{\strct{A}, \vect{a}, \strct{B}, \vect{b}}$ then the spoiler wins it.} See 2.3.16 in text.
%
\item \header{Note on the Proof of Lemma 2.2.3(b)} The statement ``Then $p : \vect{a} \mapsto \vect{b} \in \partisoms{\strct{A}}{\strct{B}}$ with $\dom{p} = A$'' follows from the fact that a submap of a partial isomorphism is also a partial isomorphism (see the next note).
%
\item \header{Note on Lemma 2.2.4} Part (c) is equivalent to say that a submap (i.e.\ a map that is a subset of another map in terms of their graphs) of a partial isomorphism is also a partial isomorphism.
\medskip\\
On the other hand, it is useful to use the alternative logically equivalent forms (such as contraposition) of parts (b) or (c) in deriving properties of the Ehrenfeucht games.
%
\item \header{Note on Lemma 2.2.6 and the Remark below It} It is better to prove the lemma and the remark that ``the conjunctions and disjunctions in the definition of $\ityp{m}{\strct{A}, \vect{a}}$ are finite'' all at once:
\medskip\\
\emph{For $s, m \geq 0$, $\ityp{m}{\strct{A}, \vect{a}}$ is a well-formed formula - i.e.\ the conjunctions and disjunctions (if any) in it are finite - for any structure $\strct{A}$ and $\vect{a} \in \cartpwr{A}{s}$ and the set $\sett{\ityp{m}{\strct{A}, \vect{a}}}{\begin{math}\strct{A}\end{math} a structure and \begin{math}\vect{a} \in \cartpwr{A}{s}\end{math}}$ is finite.}
\medskip\\
The induction is performed on $m$. The base case follows from that for $s > 0$, the set $\sett{\varphi(\seq{v}{s})}{\begin{math}\varphi\end{math} atomic or negated atomic}$ is finite.
%
\item \header{Note on the Proof of Theorem 2.2.8 (Ehrenfeucht's Theorem)} To prove (i) implies (iii), the case $m = 0$ can be handled by applying 2.2.4(a) in addition to the equivalence between (i) and (ii) of 2.2.2(c).
\medskip\\
The induction part of (i) implying (iii) in fact proves $\strct{A} \satis \varphi[\vect{a}]$ iff $\strct{B} \satis \varphi[\vect{b}]$.
%
\item \header{Note on Theorem 2.2.11} There is typo in the statement of this theorem: ``$\vect{a} \in \strct{A}$'' should be replaced by ``$\vect{a} \in \cartpwr{A}{s}$''.
%
\end{enumerate}
%end of section 2-----------------------------------------------------------------------------


%section 3------------------------------------------------------------------------------------
\section{Examples and Fra\"{i}ss\'{e}'s Theorem}
\begin{enumerate}[1.]
%
\item \header{Hint to Exercise 2.3.2} Note that $I_j \subseteq \tilde{I}_j$ and that if $p \in \partisoms{\strct{A}}{\strct{B}}$ and if $q \subseteq p$ then $q \in \partisoms{\strct{A}}{\strct{B}}$.
\begin{remark}
There is a typo: ``$\emptyseq \mapsto \emptyseq \in I_j$'' should be replaced by ``$\emptyset \mapsto \emptyset \in \tilde{I}_j$''.
\end{remark}
%
\item \header{Note on Corollary 2.3.4} For $s = 0$ the statement of 2.3.3(iii) becomes
\begin{quoteno}{$\ast$}
There is $\seqi{I_j}{j \leq m}$ with $\emptyseq \mapsto \emptyseq \in I_m$ such that $\seqi{I_j}{j \leq m} : \strct{A} \isom[m] \strct{B}$.
\end{quoteno}
\medskip\\
It is equivalent to 2.3.4(iii): The latter obviously follows the former; to derive the former from the latter, note that by 2.3.2 if $\seqi{I_j}{j \leq m} : \strct{A} \isom[m] \strct{B}$ then $\seqi{\tilde{I}_j}{j \leq m} : \strct{A} \isom[m] \strct{B}$ and $\emptyseq \mapsto \emptyseq \in \tilde{I}_m$.
%
\item \header{Note on Example 2.3.5} Let $\tau$ be an arbitrary symbol set that consists of relation symbols $\seq{P}{n}$ and constants $\seq{c}{k}$ where $n, k \in \nat$.
\medskip\\
For any $m \in \nat$, consider the two $\tau$-structures $\strct{A}$ and $\strct{B}$ where $A$ consists of elements $\seq{a}{m + 1}$ other than $\seq{\intpr{c}{\strct{A}}}{n}$, $\intpr{c_i}{\strct{A}} \neq \intpr{c_j}{\strct{A}}$ if $i \neq j$, $\intpr{P_i}{\strct{A}} = \emptyset$, $B$ consists of elements $\seq{b}{m + 2}$ other than $\seq{\intpr{c}{\strct{B}}}{n}$, $\intpr{c_i}{\strct{B}} \neq \intpr{c_j}{\strct{B}}$ if $i \neq j$, $\intpr{P_i}{\strct{B}} = \emptyset$.
\medskip\\
Obviously, exactly one between $\strct{A}$ and $\strct{B}$ is a member of $\even{\tau}$. However, it is also true that $\strct{A} \equv[m] \strct{B}$: the map $\enum{a}{m + 1} \mapsto \enum{b}{m + 1}$ can be used as a winning strategy (note that implicitly $\intpr{c_i}{\strct{A}}$ is mapped to $\intpr{c_i}{\strct{B}}$, cf.\ 2.2.2(c)(i)) for the duplicator in the game $\game{m}{\strct{A}, \strct{B}}$. Thus $\even{\tau}$ is not axiomatizable by 2.2.12.
%
\item \header{Note on Example 2.3.8} There is a typo: ``$\dist[j](a, a^\prime)$'' appearing in the definition of distance function should be replaced by ``$\dist[j](a, b)$''.
%
\item \header{Hint to Exercise 2.3.9} For simplicity, take the structure $\strct{C}_l$ that is isomorphic to $\strct{B}_l \dsjuni \strct{D}_l$, where $C_l \defas \sete{0, \ldots, 2l + 1}$ and the substructures $\substr{\sete{0, \ldots, l}}{\strct{C}_l}$ induced by $\sete{0, \ldots, l}$ and $\substr{\sete{l + 1, \ldots, 2l + 1}}{\strct{C}_l}$ induced by $\sete{l + 1, \ldots, 2l + 1}$ are isomorphic to $\strct{B}_l$ and $\strct{D}_l$, respectively.
\medskip\\
For $\strct{B}_l$ and $\strct{C}_l$ define the distance function $d$ on $B_l \cart B_l$ and on $C_l \cart C_l$ as
\[
\dist(h, k) \defas
\begin{cases}
\mbox{length of the shortest path from \begin{math}h\end{math} to \begin{math}k\end{math}} & \mbox{if there is one} \cr
\infty & \mbox{else},
\end{cases}
\]
and take the truncated version $\dist[j]$ where
\[
\dist[j](h, k) \defas
\begin{cases}
\dist(h, k) & \mbox{if \begin{math}\dist(h, k) < 2^j\end{math}} \cr
\infty & \mbox{else}.
\end{cases}
\]
For $m \geq 0$ choose $l \geq 2^m$. Consider $\seqi{I_j}{j \leq m}$ where $p \in I_j$ if and only if $p$ is a partial isomorphism between $\strct{B}_l$ and $\strct{C}_l$ such that $\card{p} \leq m - j + 2$, $p(0) = 0$, $p(l) = l$ and $\dist[j](h, k) = \dist[j](p(h), p(k))$ for $h, k \in \dom{p}$.
\medskip\\
It remains to verify $\seqi{I_j}{j \leq m} : \strct{B}_l \isom[m] \strct{C}_l$, which is omitted here. (For the forth-property, if $p \in I_{j + 1}$ and $b \in B_l$ then distinguish two cases according to whether it is true that ``there is a $b^\prime \in B_l$ such that $\dist[j](b, b^\prime) < 2^j$ or $\dist[j](b^\prime, b) < 2^j$'', a technique used in 2.3.6.)
%
\item \header{Note on Corollary 2.3.11} Here $(\strct{A}, \vect{a}) \equv[m] (\strct{B}, \vect{b})$ means ``$\vect{a}$ satisfies in $\strct{A}$ the same formulas of quantifier rank $\leq m$ as $\vect{b}$ in $\strct{B}$.'' (Consider the transition from 2.2.8(iii) to 2.2.9(iii).)
%
\item \header{Note on Corollary 2.3.11} Is it true that
\begin{quote}
``\emph{If $(\strct{A}_1, \vect{a}_1) \equv[m] (\strct{B}_1, \vect{b}_1)$ and $(\strct{A}_2, \vect{a}_2) \equv[m] (\strct{B}_2, \vect{b}_2)$ then $(\strct{A}_1 \cart \strct{A}_2, \vect{a}_1 \cart \vect{a}_2) \equv[m] (\strct{B}_1 \cart \strct{B}_2, \vect{b}_1 \cart \vect{b}_2)$}''?
\end{quote}
My guess is yes. Note that if $\vect{a}_1 \isom \vect{b}_1$ and $\vect{a}_2 \isom \vect{b}_2$ then $\vect{a}_1 \cart \vect{a}_2 \isom \vect{b}_1 \cart \vect{b}_2$.
%
\item \header{Hint to Exercise 2.3.12} An equivalent condition to ``$\min\sete{\card{A_\alpha}, m} = \min\sete{\card{B_\alpha}, m}$'' is:
\begin{quote}
$\card{B_\alpha} = \card{A_\alpha}$ if $\card{A_\alpha} < m$, and $\card{B_\alpha} \geq m$ otherwise.
\end{quote}
In addition, an alternative statement to the condition ``$\card{B_\alpha} \geq m$'' is:
\begin{quote}
For $0 \leq j < m$, $\card{B_\alpha} \neq j$.
\end{quote}
And for this exercise it is appropriate to define $\existexactly{0}x Rx \defas \neg\exists x Rx$.
\begin{remark}
A similar exercise is XII.3.17 in \cite{EFT}.
\end{remark}
%
\item \header{Hint to Exercise 2.3.13} (INCOMPLETE)
Think of $\ordsum^n \strct{A}$ as an $n$-element linear ordering in which every point ``expands'' to $\strct{A}$.\\
\medskip\\
More precisely, define for $a, a^\prime$ in the domain of $\ordsum^n \strct{A}$ the distance function
\[
\dist(a, a^\prime) \defas \abs{i - j},
\]
where $a$ is an element from the $i$th copy of $\strct{A}$ and $a^\prime$ from the $j$th. The truncated versions of distance function are defined analogously. A winning strategy for the duplicator is the same as that in 2.3.6 except that if the spoiler chooses an element from a copy $\strct{A}$ of $\ordsum^l \strct{A}$ (or $\ordsum^k \strct{A}$) then the duplicator chooses exactly the same element from the corresponding copy $\strct{A}$ of $\ordsum^k \strct{A}$ (or $\ordsum^l \strct{A}$, respectively).
%
\item \header{Hint to Exercise 2.3.14} (INCOMPLETE)
The notation ``$(\strct{A}, \vect{a}) \isom[m] (\strct{B}, \vect{b})$'' is undefined in text, however it can be understood as:
\begin{quote}
There is $\seqi{I_j}{j \leq m}$ with $\vect{a} \mapsto \vect{b} \in I_m$ such that $\seqi{I_j}{j \leq m} : \strct{A} \isom[m] \strct{B}$.
\end{quote}
This is statement (iii) of 2.3.3. Therefore this exercise is an immediate consequence of 2.3.3.
\begin{remark}
The premise ``for $\vect{a} \mapsto \vect{b} \in \partisoms{\strct{A}}{\strct{B}}$'' is implied by the statements on both sides of ``iff'', so it can be weakened to ``for $\vect{a} \in A, \vect{b} \in B$''.
\end{remark}
%
\item \header{Note on Remark 2.3.16} As an alternative proof for the last part of 2.2.8, one can show, contrapositively, that \emph{not \refitem{(iii)} implies not \refitem{(i)}}. To be more precise, in the inductive step let $m > 0, \qr{\varphi} = m$, and suppose that $\varphi(\vect{x}) = \exists y \psi$ where $\strct{A} \satis \varphi[\vect{a}]$ but not $\strct{B} \satis \varphi[\vect{b}]$. It follows that there is an $a \in A$ such that for all $b \in B$, $\strct{A} \satis \psi[\vect{a}a]$ but not $\strct{B} \satis \psi[\vect{b}b]$, hence by induction hypothesis the duplicator does not win the game $\game{m - 1}{\strct{A}, \vect{a}a, \strct{B}, \vect{b}b}$. By 2.2.4(b), therefore, we have that the duplicator does not win the game $\game{m}{\strct{A}, \vect{a}, \strct{B}, \vect{b}}$. The other case where $\strct{B} \satis \varphi[\vect{b}]$ but not $\strct{A} \satis \varphi[\vect{a}]$ can be done symmetrically.
\medskip\\
The above argument suggests (the essential part of) a winning strategy of the game $\game{m}{\strct{A}, \vect{a}, \strct{B}, \vect{b}}$ for the spoiler when (iii) in 2.2.8 is false: If a formula $\varphi = \exists y \psi$ has quantifier rank $m > 0$ and if, say, $\strct{A} \satis \varphi[\vect{a}]$, then he picks an element $a \in A$ such that $\strct{A} \satis \psi[\vect{a}a]$; if $\varphi = \forall y \chi$ has quantifier rank $m > 0$ and if, say, not $\strct{B} \satis \varphi[\vect{b}]$ (namely if $\strct{B} \satis \exists y \neg\chi[\vect{b}]$), then he picks an element $b \in B$ such that $\strct{B} \satis \neg\chi[\vect{b}b]$. The spoiler then makes his successive choices accordingly in his turns in every play of the game.
%
\end{enumerate}
%end of section 3-----------------------------------------------------------------------------


%section 4------------------------------------------------------------------------------------
\section{Hanf's Theorem}
\begin{enumerate}[1.]
%
\item \header{Note on the First Paragraph on Page 27} Here the \emph{isomorphism type} of $(\ballstrct{r, a}, a)$ might refer to the equivalent class of $(\ballstrct{r, a}, a)$ induced by the isomorphism relation, i.e.\ the set of structures that are isomorphic to $(\ballstrct{r, a}, a)$, where $(\ballstrct{r, a}, a)$ might refer to the expansion of the structure $\ballstrct{r, a}$ in which $a$ is a distinguished constant.
%
\item \header{Note on the Proof of 2.4.1} Here $\length{\vect{a}}$ refers to the length of the tuple $\vect{a}$, see the first paragraph on page 6.
\medskip\\
On the other hand, by the proof the requirement on the cardinality of $3^m$-balls may be weakened to ``at most $e$ elements.''
%
\item \header{Note on 2.4.2 and 2.4.3} For the proof of 2.4.3 to be valid, in the structure $(D_l, E^\prime_l, P_1, \etl, P_r)$ there must be a point on each of the paths from $a$ to $b_-$ and from $b$ to $a_-$ that is in neither of the $3^m$-balls of $a$ and $b$, in other words, both cycles in the structure must have length greater than $2 \mul 3^m + 1$; otherwise the $3^m$-ball type of $a$ (or $b$) would be different from that of $a$ (or $b$, respectively) in $(\strct{D}_l, P_1, \etl, P_r)$ - the former $3^m$-ball is a cycle, whereas the latter is not.
\medskip\\
In fact, 2.4.3 can be strengthened to allow such points.
%
\item \header{Note on 2.4.4} By definition, the Gaifman graph $\gaifman{\strct{A}}$ of a digraph $\strct{A}$ is the associated (undirected) graph of $\strct{A}$.
%
\item \header{Note on 2.4.5} By the same argument in the proof, it follows that \emph{the class of finite graphs that are not connected cannot be axiomatized by a formula of the form $\exists P_1 \etl \exists P_r \psi$, either.}
\medskip\\
As an immediate consequence, we have that \emph{both classes cannot be axiomatized by a formula of the form $\forall P_1 \etl \forall P_r \chi$.}
%
\item \header{Note on 2.4.6} In the proof there is a (possible) typo: $\intpr{R}{A}$ in $(\strct{G}, \intpr{R}{A})$ should be replaced by $\intpr{R}{G}$ or $\intpr{R}{\strct{G}}$.
\medskip\\
On the other hand, this proposition implies that \emph{the class of finite graphs that are not connected can be axiomatized by a formula of the form $\forall R \chi$.} (Just take the negation of $\exists R \psi$.)
%
\item \header{Brief Solution to Exercise 2.4.7} Using a similar method (basically the pigeonhole principle) we can obtain the corresponding result to 2.4.2, in which the distance $\dist(a, b)$ is greater than $2 \mul 3^m + 1$ (cf.\ \header{Note on 2.4.2 and 2.4.3}). Now let $a_-$ and $b_-$ be the two points such that $(a_-, a)$ and $(b_-, b)$ are edges in $\strct{H}_l$. Obtain the structure $\strct{H}_l^\prime$ from $\strct{H}_l$ by removing these two edges and adding $(b_-, a)$ and $(a_-, b)$. Then likewise we obtain the corresponding result to 2.4.3, and hence that to 2.4.5.
\medskip\\
As for the corresponding result to 2.4.6, note that a digraph is cyclic if and only if there is a linear ordering over a set of at least two points in which there is an edge from $x$ to $y$ if and only if $y$ is immediately greater than $x$ or $x$ is the greatest element and $y$ the least in the linear ordering. Finally, formulate the condition and take the negation.
\begin{note}
It does not seem possible to axiomatize this class by a formula of the form $\forall P \psi$ where $P$ is unary and $\psi$ first-order (by the corresponding result to 2.4.5 mentioned above and the discussion in \header{Note on 2.4.5}) or the form $\exists R \psi$ where $R$ is binary and $\psi$ first-order as required (this is conjectured, however).
\end{note}
%
\item \header{Brief Solution to Exercise 2.4.8} An informal formulation for $\psi(x, y)$ is already present in the description of the exercise.
\medskip\\
Also observe that the formula $\forall x \forall y \exists P \varphi$ axiomatizes the class of finite and connected graphs, and hence by 2.4.5 is not equivalent to a sentence of the form $\exists P_1 \etl \exists P_r \chi$.
%
\item \header{Note on the Definition of Basic Local Sentences} In the definition given in textbook, the condition $\rltv{\psi}{\ball{r, x_n}}$ is missing; therefore, it should be replaced by
\[
\exists x_1 \etl \exists x_n (\bland_{1 \leq i < j \leq n}\dist(x_i, x_j) > 2r \land \bland_{1 \leq i \leq n} \rltv{\psi}{\ball{r, x_i}}(x_i)),
\]
where $n \geq 1$. (The sentence that precedes 2.5.1 thus should be modified accordingly.)
%
\end{enumerate}
%end of section 4-----------------------------------------------------------------------------


%section 5------------------------------------------------------------------------------------
\section{Gaifman's Theorem}
\begin{enumerate}[1.]
%
\item \header{Note on the Proof of 2.5.2} In case 1, note that, as in the proof of 2.4.1, $\ball{7^j, \vect{a}a} \subseteq \ball{7^{j + 1}, \vect{a}}$, so $\ballstrct[{\ballstrct[\strct{A}]{7^{j + 1}, \vect{a}}}]{7^j, \vect{a}a} = \ballstrct[\strct{A}]{7^j, \vect{a}a}$.
\medskip\\
In case 2, it seems sufficient to have $g(j + 1)$ no smaller than the quantifier rank of sentences in (2) and (3). Moreover, we have $\strct{A} \satis \psi^j_a(a)$ and hence $\strct{A} \satis \exists x_1 \delta_1(x_1)$ (here $\exists x_1 \delta_1(x_1) = \exists x_1 \psi^j_a(x_1)$), so $i \geq 1$. Also note the following:
\begin{enumerate}[(a)]
%%
\item In case 2.1, we have $e \geq 1$ (recall that $i \geq 1$) so the argument on the upper bound of the distance of any element (in $A$) from $\vect{a}$ satisfying $\psi^j_a$ is valid. There is a typo: ``$a \not\in \ball{2 \mul 7^{j + 1}, \vect{a}}$'' should be replaced with ``$a \not\in \ball{2 \mul 7^j, \vect{a}}$''. The condition $\dist(\vect{a}, a) \leq 6 \mul 7^j$ implies that $\ball{7^j, a} \subseteq \ball{7^{j + 1}, \vect{a}}$ and hence $\ballstrct[\ballstrct{7^{j + 1}, \vect{a}}]{7^j, a} = \ballstrct[\strct{A}]{7^j, a}$, which -- together with the assumption that $a \not\in \ball{2 \mul 7^j, \vect{a}}$ and the fact that $\ballstrct[\ballstrct{7^{j + 1}, \vect{a}}]{7^j, \vect{a}} = \ballstrct[\strct{A}]{7^j, \vect{a}}$ (since $\ball{7^j, \vect{a}} \subseteq \ball{7^{j + 1}, \vect{a}}$) -- gives
\[
\ballstrct{7^{j + 1}, \vect{a}} \satis \exists z (2 \mul 7^j < \dist(\vect{a}, z) \leq 6 \mul 7^j \land \psi^j_a(z) \land \psi^j_{\vect{a}}(\vect{a})).
\]
Also, the condition
\[
\ballstrct{7^{j + 1}, \vect{b}} \satis \exists z (2 \mul 7^j < \dist(\vect{b}, z) \leq 6 \mul 7^j \land \psi^j_a(z) \land \psi^j_{\vect{a}}(\vect{b}))
\]
guarantees a $b \in B$ such that $\ball{7^j, \vect{b}} \intsc \ball{7^j, b} = \emptyset$ and $\ball{7^j, b} \subseteq \ball{7^{j + 1}, \vect{b}}$ and hence $\ballstrct[\ballstrct{7^{j + 1}, \vect{b}}]{7^j, b} = \ballstrct[\strct{B}]{7^j, b}$ which justifies (5); this condition also justifies (6) because $\ball{7^j, \vect{b}} \subseteq \ball{7^{j + 1}, \vect{b}}$ which implies $\ballstrct[\ballstrct{7^{j + 1}, \vect{b}}]{7^j, \vect{b}} = \ballstrct[\strct{B}]{7^j, \vect{b}}$.
%%
\item In case 2.2, we also have
\[
\ballstrct{7^{j + 1}, \vect{b}} \satis \psi^j_{\vect{a}}[\vect{b}]
\]
because $\ballstrct{7^{j + 1}, \vect{a}} \satis \psi^j_{\vect{a}}[\vect{a}]$ (since $\ball{7^j, \vect{a}} \subseteq \ball{7^{j + 1}, \vect{a}}$ and hence $\ballstrct[{\ballstrct[\strct{A}]{7^{j + 1}, \vect{a}}}]{7^j, \vect{a}} = \ballstrct[\strct{A}]{7^j, \vect{a}}$) and (1);\footnote{The quantifier rank of $\psi^j_{\vect{a}}$ is smaller than $g(j + 1)$, by the third condition on the value of $g(j + 1)$ mentioned in case 2.1; although this condition is specified there, it is also applicable here because $g$ depends only on $\strct{A}$ and $\strct{B}$ but not on specific cases among 1, 2.1 or 2.2.} this implies
\[
(\ballstrct{7^j, \vect{a}}, \vect{a}) \equv[g(j)] (\ballstrct{7^j, \vect{b}}, \vect{b})
\]
since, similarly, $\ball{7^j, \vect{b}} \subseteq \ball{7^{j + 1}, \vect{b}}$ and hence $\ballstrct[\ballstrct{7^{j + 1}, \vect{b}}]{7^j, \vect{b}} = \ballstrct[\strct{B}]{7^j, \vect{b}}$.
%%
\end{enumerate}
%
\end{enumerate}
%end of section 5-----------------------------------------------------------------------------
%\setcounter{chapter}{2}
\chapter{More on Games}
%section 1-------------------------------------------------------------------------------------
\section{Second-Order Logic}
\begin{enumerate}[1.]
%
\item \header{Hint to Exercise 3.1.2} Argue as in 2.2.2, 2.2.4, and 2.2.6-8.
%
\end{enumerate}
%end of section 1-----------------------------------------------------------------------------


%section 2------------------------------------------------------------------------------------
\section{Infinitary Logic: The Logics $\inflog$ and $\dlog{\omega_1}$}
\begin{enumerate}[1.]
%
\item \header{Subformulas of $\inflog$-Sentences Only Have Finitely Many Free Variables} In fact, one can show by induction on the formation of formulas that if $\varphi \in \inflog$ has infinitely many free variables, then any $\inflog$-formula having $\varphi$ as a subformula must also have infinitely many free variables.
%
\item \header{Note on 3.2.3} One can easily see that $\varphi(\vect{x})$ is equivalent to a countable conjunction of first-order formulas:
\[
\bland \setm{\cardexactly{n} \lthen \blor\setm{\hint{\card{A} + 1}{\strct{A}, \vect{a}}(\vect{x})}{\card{A} = n, \vect{a} \in A, \strct{A} \satis \varphi[\vect{a}]}}{n \geq 1}.
\]
Note that the disjunction in every conjunct above is finite.
%
\item \header{Note on 3.2.4 and 3.2.6} In fact, for any (finite) vocabulary $\tau$ and any $\tau$-structures $\strct{A}, \strct{B}$ the following statements are equivalent (cf.\ Chapter XII in \cite{EFT} for more details):
\begin{enumerate}[(1)]
%%
\item $\strct{A} \equv \strct{B}$, i.e.\ $\strct{A}$ and $\strct{B}$ are \emph{elementarily equivalent} ($\strct{A}$ and $\strct{B}$ satisfy the same $\folog\of{\tau}$-sentences)
%%
\item $\strct{A} \isom_\fin \strct{B}$, namely $\strct{A}$ and $\strct{B}$ are \emph{finitely isomorphic} (there is an infinite sequence $\seq{I_j}{j \in \nat}$ that satisfies the properties \refitem{(a) - (c)} in Definition 2.3.1 except that ``$j < m$'' is replaced by ``$j \in \nat$'' in conditions \refitem{(b)} and \refitem{(c)})
%%
\item For $j \in \nat$, the duplicator wins the game $\game{j}(\strct{A}, \strct{B})$
%%
\item The duplicator wins the game $\fingame(\strct{A}, \strct{B})$ in which the spoiler begins the game by first choosing a natural number $r$ and then the game proceeds as in $\game{r}(\strct{A}, \strct{B})$.
%%
\end{enumerate}
Using Corollary 2.3.4, we obtain a further equivalent statement:
\begin{enumerate}[(1)]
\setcounter{enumii}{4}
%%
\item $\strct{B} \satis \bland_{j \in \nat} \hint{j}{\strct{A}}$,
%%
\end{enumerate}
and, using Lemma 2.2.4(b), yet another equivalent statement, in parallel to Corollary 2.3.4(ii):
\begin{enumerate}[(1)]
\setcounter{enumii}{5}
%%
\item $\seq{\winpos{j}(\strct{A}, \strct{B})}{j \in \nat} : \strct{A} \isom_\fin \strct{B}$.
%%
\end{enumerate}

The above should be easy to generalize so that we have $(\strct{A}, \vect{a}), (\strct{B}, \vect{b})$ in place of $\strct{A}, \strct{B}$, and $\vect{a} \mapsto \vect{b} \in I_j$ (or $\vect{a} \mapsto \vect{b} \in \winpos{j}(\strct{A}, \strct{B})$) for all $j$ in \refitem{(2)} (or \refitem{(6)}, respectively). For example, we can regard $(\strct{A}, \vect{a})$ and $(\strct{B}, \vect{b})$ as the $\tau \union \sete{\vect{c}}$-expansion of $\strct{A}$ and of $\strct{B}$ in which $\vect{c}$ is interpreted by $\vect{a}$ and $\vect{b}$, respectively.

Notice that, however, the condition that the duplicator wins $\fingame(\strct{A}, \vect{a}, \strct{B}, \vect{b})$ generally does not imply that he wins $\game{\infty}(\strct{A}, \vect{a}, \strct{B}, \vect{b})$ although the converse is obviously true. See Exercise XII.1.10 in \cite{EFT} to give an example of two finitely isomorphic structures that are not partially isomorphic; the same exercise problem also asks to give an example of two partially isomorphic structures that are not isomorphic.

On the other hand, consider the $\emptyvoc$-structures (i.e.\ sets) $\strct{A}, \strct{B}$ that consist of the domains $A = \sete{0, 1}, B = \sete{0, 1, 2}$, respectively. Then it follows that $\strct{A} \isom_2 \strct{B}$ but not $\strct{A} \isom_\fin \strct{B}$ (since not $\strct{A} \isom_3 \strct{B}$).

To summarize, we have the successively weaker notions (where \refitem{(1)} clearly implies \refitem{(2)}):
\begin{enumerate}[(1)]
%%
\item $\strct{A} \isom \strct{B}$
%%
\item $\strct{A} \isom_\partially \strct{B}$
%%
\item $\strct{A} \isom_\fin \strct{B}$
%%
\item $\strct{A} \isom_j \strct{B}$ in which $j \in \nat$.
%%
\end{enumerate}
%
\item \header{Note on Theorem 3.2.7} In the direction from \refitem{(iii)} to \refitem{(iv)} of the proof, the length $s$ of tuples in \refitem{($\ast$)} is better to be replaced by say $r$, since $s$ is fixed for the length of $\vect{a}$ and $\vect{b}$ whereas the tuples in \refitem{($\ast$)} may have different lengths. Also, there is a typo in the direction from \refitem{(iv)} to \refitem{(iii)}: ``$a \in I$'' should change to ``$a \in A$''.

On the other hand, this theorem does not seem to have a corresponding statement \refitem{(iv)} to that of Theorem 2.3.3. The statement $(\strct{B}, \vect{b}) \satis \bland_{j \in \nat} \hint{j}{\strct{A}, \vect{a}}$ does not work because it is equivalent to $(\strct{A}, \vect{a}) \equv (\strct{B}, \vect{b})$, namely $(\strct{A}, \vect{a}), (\strct{B}, \vect{b})$ satisfy the same first-order formulas $\varphi$ in which it is decisive whether or not $\strct{A} \satis \varphi[\vect{a}]$ (or equivalently, $\strct{B} \satis \varphi[\vect{b}]$) is the case (see the discussion in Note on 3.2.4 and 3.2.6).
%
\item \header{Note on \refitem{(iii)} of 3.2.8} To transition from \refitem{(iii)} of 3.2.7 for $s = 0$, one refers to 2.3.2 (also cf.\ the transition from 2.3.3 to 2.3.4 using 2.3.2).
%
\item \header{Note on 3.2.11}
\begin{enumerate}[(1)]
%%
\item For $r \geq 0$, the set $\Delta_{r + 1}$ is finite.
%%
\item A $2$-extension axiom has the form
\[
\forall v_1 \exists v_2 (v_1 \neq v_2 \land \bland_{\varphi \in \Phi} \varphi \land \bland_{\varphi \in \cmpl{\Phi}} \neg\varphi).
\]
%%
\item In fact, the condition ``$\hint{0}{\strct{A}, \vect{a}} = \hint{0}{\strct{B}, \vect{b}}$'' in the definition of the set $I$ of maps is equivalent to ``$\vect{a} \mapsto \vect{b}$ is a partial isomorphism from $\strct{A}$ to $\strct{B}$'', by 2.2.5 and parts (b) and (c) of 2.2.7, whether or not $\tau$ is relational; in other words, $I = \partisoms(\strct{A}, \strct{B})$.

Moreover, by the assumption that $\tau$ is relational (containing no constants), the empty map $\emptymap$ is a partial isomorphism from $\strct{A}$ to $\strct{B}$ (see 2.2.2(a)) and coincides with $\emptyseq \mapsto \emptyseq$; it follows that $\hint{0}{\strct{A}, \emptyseq} = \hint{0}{\strct{B}, \emptyseq}$ and is equal to $\tr \land \neg\fls$ (cf.\ Part B in Chapter 1).

In case $\tau$ is not relational, however, the empty map $\emptymap$ and $\emptyseq \mapsto \emptyseq$ are not identical. If in addition $\tau$ contains two constants $c_1, c_2$ so that $\intpr{c_1}{\strct{A}} = \intpr{c_2}{\strct{A}}$ but $\intpr{c_1}{\strct{B}} \neq \intpr{c_2}{\strct{B}}$, then $\emptyseq \mapsto \emptyseq$ is not even a partial isomorphism and $I = \emptyset$.
%%
\item In the proof $I$ has the forth property, note that
\[
\begin{array}{lll}
\ & \hint{0}{\strct{B}, \vect{b}b_{r + 1}} & \cr
= & \hint{0}{\strct{B}, \vect{b}} \land \bland\limits_{1 \leq i \leq r} \neg v_i = v_{r + 1} \land \bland\limits_{\varphi \in \Phi} \varphi \land \bland\limits_{\varphi \in \cmpl{\Phi}} \neg\varphi & \text{(since \mathmode{(\strct{B}, \vect{b}b_{r + 1}) \satis \Phi})} \cr
= & \hint{0}{\strct{A}, \vect{a}} \land \bland\limits_{1 \leq i \leq r} \neg v_i = v_{r + 1}\land \bland\limits_{\varphi \in \Phi} \varphi \land \bland\limits_{\varphi \in \cmpl{\Phi}} \neg\varphi & \text{(since \mathmode{\vect{a} \mapsto \vect{b} \in I})} \cr
= & \hint{0}{\strct{A}, \vect{a}a_{r + 1}} & \text{(since \mathmode{(\strct{A}, \vect{a}a_{r + 1}) \satis \Phi})}. \cr
\end{array}
\]
%%
\item In the construction of a countable model $\strct{A}$ of $\randstrtheory$ by means of the infinite sequence of $\strct{A}_n$'s, note that $r \leq n + 1$ is implied by the requirement that $\vect{m}$ consists of distinct entries and all entries are not greater than $n$.

However, this construction procedure does not specify, given $\strct{A}_n$ and $\alpha_n = (\vect{m}, \extaxm)$, the relationship in $\strct{A}_{n + 1}$ between the element $n + 1$ and any other tuple than $\vect{m}$ in terms of the interpretation of relation symbols $R \in \tau$. Thus, the relationship can be arbitrarily defined. (Nevertheless, this freedom in defining the relationship is somewhat restricted in the construction procedure for a countable model of $\randstrtheory(\varphi_0)$ mentioned before (3) leading to Theorem 4.2.3.)
%%
\item If $\tau$ only contains unary relation symbols, say $\tau = \sete{R_1, \etc, R_m}$. Then for every model $\strct{A}$ of $\randstrtheory$, the the sets $\intpr{R_1}{\strct{A}}, \etc, \intpr{R_m}{\strct{A}}$ yield a partition of the universe $A$ of $\strct{A}$ into $2^m$ subsets, each of which consists of infinitely many elements.

\begin{remark}
The results given in textbook remain valid in this case.
\end{remark}
%%
\end{enumerate}
%
\item \header{Hint to Exercise 3.2.13} Note that $x_1$ is the first element of the tuple $\vect{x}$ and therefore the sentence $\forall \vect{x} (\existexactly{1}x F\vect{x}x \land F\vect{x}x_1)$ formalizes the idea that $F$ is a (total) function that projects a tuple of parameter(s) onto its first parameter.
%
\item \header{Solution to Exercise 3.2.14} Without loss of generality let us assume that $A$ is finite and that $\card{A} = \min\sete{\card{A}, \card{B}}$. We also assume, for simplicity, that partial isomorphisms from $\strct{A}$ to $\strct{B}$ take the form $\vect{a} \mapsto \vect{b}$ (having finite domains and ranges) in which $\vect{a}$ and $\vect{b}$ consist of distinct elements.
\medskip\\
Now we distinguish three cases:
\begin{enumerate}[(1)]
%%
\item $\strct{A} \isom \strct{B}$.
%%
\item For any $m \in \nat$, $\vect{a} \in A$ and $\vect{b} \in B$, not $(\strct{A}, \vect{a}) \isom_m (\strct{B}, \vect{b})$.
%%
\item Not $\strct{A} \isom \strct{B}$, and there are $m \in \nat$, $\vect{a} \in A$ and $\vect{b} \in B$ such that $(\strct{A}, \vect{a}) \isom_m (\strct{B}, \vect{b})$.
%%
\end{enumerate}
In case (1), we have that $\winpos{\infty}(\strct{A}, \strct{B}) = \winpos{0}(\strct{A}, \strct{B}) = \partisoms(\strct{A}, \strct{B}) \neq \emptyset$, which contains all isomorphisms from $\strct{A}$ to $\strct{B}$. Thus, set $m_0 \defas 0$.
\medskip\\
In case (2), it follows in particular that for any $\vect{a} \in A$ and $\vect{b} \in B$, not $(\strct{A}, \vect{a}) \isom_0 (\strct{B}, \vect{b})$. Hence $\winpos{\infty}(\strct{A}, \strct{B}) = \winpos{0}(\strct{A}, \strct{B}) = \partisoms(\strct{A}, \strct{B}) = \emptyset$. Thus, set $m_0 \defas 0$.
\medskip\\
In case (3), we may assume, with no loss of generality, that $m \leq \card{A}$ and for any $\vect{a'} \in A$ and $\vect{b'} \in B$, not $(\strct{A}, \vect{a'}) \isom_{m + 1} (\strct{B}, \vect{b'})$. So we have
\[
\winpos{0}(\strct{A}, \strct{B}) \neq \emptyset, \etc, \winpos{m}(\strct{A}, \strct{B}) \neq \emptyset
\]
(cf.\ 2.2.4(c)) but
\[
\winpos{\infty}(\strct{A}, \strct{B}) = \winpos{m + 1}(\strct{A}, \strct{B}) = \emptyset.
\]
It then suffices to show that for $j < m$, 
\[
\winpos{j}(\strct{A}, \strct{B}) \neq \winpos{j + 1}(\strct{A}, \strct{B}).
\]
In fact, in any $\winpos{j + 1}(\strct{A}, \strct{B})$ there must be a maximal -- in the sense of set inclusion -- partial isomorphism $q$ (i.e.\ there is $q \in \winpos{j + 1}(\strct{A}, \strct{B})$ such that there is no $q' \in \winpos{j + 1}(\strct{A}, \strct{B})$ with $q' \supset q$) for which there is an $a \in A$ with $a \notin \dm(q)$, since $A$ is finite and $\strct{A}$ and $\strct{B}$ are not isomorphic (also cf.\ the proof of 2.2.3(b)). By 2.2.4(b) there is a $p \in \winpos{j}(\strct{A}, \strct{B})$ with $\dm(p) = \dm(q) \union \sete{a}$; note that $p \notin \winpos{j + 1}(\strct{A}, \strct{B})$ because $p \supset q$ and $q$ is maximal in $\winpos{j + 1}(\strct{A}, \strct{B})$. Thus, set $m_0 \defas m + 1$.
\begin{remark}
By definition (cf.\ 3.2.4) and 2.2.4(c), we immediately have
\[
\winpos{0}(\strct{A}, \strct{B}) \supseteq \etc \supseteq \winpos{m}(\strct{A}, \strct{B}) \supseteq \etc \supseteq \winpos{\infty}(\strct{A}, \strct{B}).
\]
\end{remark}
%
\end{enumerate}
%end of section 2-----------------------------------------------------------------------------


%section 3------------------------------------------------------------------------------------
\section{The Logics $\folog[s]$ and $\inflog[s]$}
\begin{enumerate}[1.]
%
\item \header{Note on Pebble Games $\game[s]{m}(\strct{A}, \vect{a}, \strct{B}, \vect{b})$} They are different from the usual Ehrenfeucht-Fra\"iss\'e games $\game{m}(\strct{A}, \vect{a}, \strct{B}, \vect{b})$ in that the number of pebbles are fixed ($s$ for each of $\strct{A}$ and $\strct{B}$ here) and the pebbles are on or off (denoted by $\ast$) the board; once a pebble is placed onto the board, it is never removed off of it. Thus, pebble games consist of putting the pebbles onto the board (extensions) and moving them around on the board (moves).
%
\item \header{Note on \thesection.6}
\begin{enumerate}[(a)]
%%
\item To be clear, the last statement should be ``for arbitrary \emph{$\emptyvoc$-structures} $\strct{A}$ and $\strct{B}$, the duplicator wins $\game[s]{\infty}(\strct{A}, \strct{B})$ iff he wins $\game[s]{s}(\strct{A}, \strct{B})$.'' This is not true for arbitrary $\tau$, however, as can be seen in the example given in the next part.
%%
\item The spoiler wins $\game[3]{\infty}(\strct{G}_l, \strct{G}_l \dunion \strct{G}_l)$ but does not win $\game[3]{3}(\strct{G}_l, \strct{G}_l \dunion \strct{G}_l)$. As remarked in 2.3.16, the information that exactly one of the structures $\strct{G}_l$ and $\strct{G}_l \dunion \strct{G}_l$ satisfy the $\inflog[3]$-sentence expressing connectivity can be transformed into a winning strategy for the spoiler.
%%
\end{enumerate}

%
\item \header{Hint to Exercise \thesection.7}
\begin{enumerate}[(a)]
%%
\item For the `if' direction, assume that $A = \sete{a_0, \etc, a_k}$, where
\[
a_0 \mathrel{\intpr{<}{A}} \etc \mathrel{\intpr{<}{A}} a_k.
\]
Then take the conjunction of the following $\folog[2]$-sentences that together describe $\strct{A}$ up to isomorphism (where $\psi'_n$ is as in Examples \thesection.1).
\begin{itemize}
%%%
\item The size of $A$ ($\card{A} = k + 1$):
\[
\exists x \psi'_{k - 1} \land \forall x \neg\psi'_k.
\]
%%%
\item The interpretation of $c \in \tau$ in $\strct{A}$ ($\intpr{c}{\strct{A}} = a_i$):
\[
\exists x (x = c \land \psi'_i).
\]
%%%
\item The elements of the interpretation $\intpr{P}{\strct{A}}$ of unary $P \in \tau$ in $\strct{A}$:
\[
\forall x (Px \liff \blor \setm{\psi'_i}{a_i \in \intpr{P}{\strct{A}}}).
\]
%%%
\item The elements of the interpretation $\intpr{R}{\strct{A}}$ of binary $R \in \tau$ in $\strct{A}$:
\[
\forall x \forall y (Rxy \liff \blor \setm{\psi'_i \land \psi'_j\begin{perm}{c} yx \cr xy \end{perm}}{(a_i, a_j) \in \intpr{R}{\strct{A}}}).
\]
%%%
\end{itemize}
%%
\item For the `if' direction, note that the partial isomorphisms of size $s$ from $\strct{A}$ to $\strct{B}$ resulted from all possible rounds of the game $\game[s]{m}(\strct{A}, \strct{B})$ with the additional condition after $s$ moves provide a winning strategy for the duplicator in the same game without that condition.
%%
\item Suppose that $\tau$ is relational and all its relation symbols have arity $\leq s$, where $s \geq 1$ by definition (cf.\ Part A in Chapter 1). Moreover, suppose that $\card{A} = \card{B} \leq s + 1$ and that the duplicator wins $\game[s]{\infty}(\strct{A}, \strct{B})$. Our goal then is to show that $\strct{A} \isom \strct{B}$. We further assume $\tau \neq \emptyvoc$, since otherwise it is trivial that the duplicator wins $\game[s]{\infty}(\strct{A}, \strct{B})$ and $\strct{A} \isom \strct{B}$.
\medskip\\
First observe the assumption that the duplicator wins $\game[s]{\infty}(\strct{A}, \strct{B})$ immediately implies that he wins $\game[s]{m}(\strct{A}, \strct{B})$ for $m \geq 1$. By Theorem \thesection.5, then, we have for every $m \geq 1$, $\strct{A} \equv^s_m \strct{B}$. Therefore, it suffices to find a sentence $\varphi_\strct{A} \in \folog[s]\of{\tau}$ that characterizes $\strct{A}$ up to isomorphism given the fixed cardinality $\card{A}$, i.e.
\begin{center}
for every structure $\strct{A}'$ with $\card{A'} = \card{A}$, \ $\strct{A}' \satis \varphi_\strct{A}$ \ iff \ $\strct{A}' \isom \strct{A}$.
\end{center}
Then we are done: The first-order sentence $\varphi_\strct{A}$ has a (finite) quantifier rank, say $m$, and it follows that $\strct{A} \isom \strct{B}$ because $\strct{B} \satis \varphi_\strct{A}$ (implied by $\strct{A} \equv^s_m \strct{B}$ above).
\medskip\\
Next, the sentence $\varphi_\strct{A}$ depends on whether $\card{A} \leq s$ or $\card{A} = s + 1$. Recall the set of atomic formulas
\[
\Theta_n \defas \sett{\psi}{\mathmode{\psi} has the form \mathmode{Rx_1 \etc x_k}, \mathmode{x = y} and variables among \mathmode{v_1, \etc, v_n}}
\]
used in the proof of 2.1.1. (There are no formulas of the form $c = x$ in $\Theta_n$ as $\tau$ is assumed to be relational.)
\medskip\\
The case $\card{A} \leq s$ is simple: If $\card{A} = n$, i.e.\ $A = \sete{a_1, \etc, a_n}$ then we choose
\[
\varphi_\strct{A} \defas \exists v_1 \etc \exists v_n (\bland\setm{\psi}{\psi \in \Theta_n, \strct{A} \satis \psi[\vect{a}]} \land \bland\setm{\neg\psi}{\psi \in \Theta_n, \strct{A} \satis \neg\psi[\vect{a}]})
\]
where $\vect{a}$ denotes $a_1 \etc a_n$.
\medskip\\
It remains to deal with the case $\card{A} = s + 1$, i.e.\ $A = \sete{a_1, \etc, a_{s + 1}}$. Remember that we only have $s$ distinct variables $v_1, \etc, v_s$ to describe $\strct{A}$. The idea is that, using these $s$ (quantified) variables, we describe the relations and (in)equalities
\begin{enumerate}[(1)]
%%
\item among $a_1, \etc, a_s$ first, where $v_1, \etc, v_s$ are intended to represent $a_1, \etc, a_s$ respectively, and then
%%
\item among $a_1, \etc, a_{i - 1}, a_{i + 1}, \etc, a_{s + 1}$ for each $1 \leq i \leq s$, where $v_1, \etc, v_{i - 1}, v_i, v_{i + 1}, \etc, v_s$ are intended to represent $a_1, \etc, a_{i - 1}, a_{s + 1}, a_{i + 1}, \etc, a_s$ ($a_{s + 1}$ is between $a_{i - 1}$ and $a_{i + 1}$ in this order), respectively.
%%
\end{enumerate}
Note that this is feasible as every relation symbol has arity at most $s$.
\medskip\\
For brevity, let us denote $a_1, \etc, a_{i - 1}, a_{s + 1}, a_{i + 1}, \etc, a_s$ ($a_{s + 1}$ is between $a_{i - 1}$ and $a_{i + 1}$ in this order) by $\vect{a}_i$ for $1 \leq i \leq s$; moreover, we also let $\vect{a}_0$ denote $a_1, \etc, a_s$. The desired sentence is thus
\[
\varphi_\strct{A} \defas \exists v_1 \etc \exists v_s \bland^s_{i = 0} \varphi_i,
\]
where
\[
\varphi_0 \defas \bland \setm{\psi}{\psi \in \Theta_s, \strct{A} \satis \psi[\vect{a}_0]} \land \bland \setm{\neg\psi}{\psi \in \Theta_s, \strct{A} \satis \neg\psi[\vect{a}_0]}
\]
and for $1 \leq i \leq s$,
\[
\varphi_i \defas
\begin{cases}
\forall v_i (\bland_{1 \leq j \leq s, j \neq i} \neg v_i = v_j \lthen \varphi_0) & \text{if \mathmode{\strct{A} \satis \varphi_0[\vect{a}_i]}}\cr
\exists v_i (\bland \setm{\psi}{\psi \in \Theta_s, \strct{A} \satis \psi[\vect{a}_i]} \land \bland \setm{\neg\psi}{\psi \in \Theta_s, \strct{A} \satis \neg\psi[\vect{a}_i]}) & \text{otherwise}.\cr
\end{cases}
\]
%%
\end{enumerate}
%
\item \header{Note on the $s$-$m$-Isomorphism Type $\ityp[s]{m}{\strct{A}, \vect{a}}$} Unlike $\game{m}(\strct{A}, \vect{a}, \strct{B}, \vect{b})$, the game $\game[s]{m}(\strct{A}, \vect{a}, \strct{B}, \vect{b})$ consists of moves that \emph{substitute} pairs of elements in partial isomorphisms rather than \emph{extend} partial isomorphisms with pairs of elements. This is reflected in the definition of $\ityp{m + 1}{\strct{A}, \vect{a}}$:
\begin{enumerate}[(1)]
%%
\item The running conjunction $\bland\limits_{1 \leq i \leq s}$ specifies the valid positions for substitution, and the formula $\ityp{m}{\strct{A}, \vect{a}\sbst{a}{i}}$ specifies that $\vect{a}\sbst{a}{i} \mapsto \vect{b}\sbst{b}{i}$ is an $s$-partial isomorphism whose constituent pairs can be substituted $m$ times (given the premise $\strct{B} \satis \ityp{m}{\strct{A}, \vect{a}\sbst{a}{i}}[\vect{b}\sbst{b}{i}]$).
%%
\item The conjunct $\ityp{0}{\strct{A}, \vect{a}}$ is necessary because it describes the relation among the elements in $\vect{a}$ so that if $\strct{B} \satis \ityp{0}{\strct{A}, \vect{a}}[\vect{b}]$ then $\vect{a} \mapsto \vect{b}$ is an $s$-partial isomorphism from $\strct{A}$ to $\strct{B}$ (and vice versa), whereas the other conjunct $\bland\limits_{1 \leq i \leq s}(\bland\limits_{a \in A} \exists v_i \ityp{m}{\strct{A}, \vect{a}\sbst{a}{i}} \land \forall v_i \blor\limits_{a \in A} \ityp{m}{\strct{A}, \vect{a}\sbst{a}{i}})$ does not have this effect. In fact, we have
\[
\consq \hint{m + 1}{\strct{A}, \vect{a}} \lthen \hint{0}{\strct{A}, \vect{a}}
\]
(cf.\ 2.2.4(c)) but
\begin{center}
not $\consq \bland\limits_{1 \leq i \leq s}(\bland\limits_{a \in A} \exists v_i \ityp{m}{\strct{A}, \vect{a}\sbst{a}{i}} \land \forall v_i \blor\limits_{a \in A} \ityp{m}{\strct{A}, \vect{a}\sbst{a}{i}}) \lthen \ityp{0}{\strct{A}, \vect{a}}$
\end{center}
(for $\vect{a} \mapsto \vect{b}$ there may be an $a' \neq a_i$ or a $b' \neq b_i$ such that $\vect{a}\sbst{a'}{i} \mapsto \vect{b}$ or $\vect{a} \mapsto \vect{b}\sbst{b'}{i}$ is an $s$-partial isomorphism).

Note that the conjunct $\ityp{0}{\strct{A}, \vect{a}}$ is redundant and is implied by the other conjunct only when $\vect{a} = \ast \etc \ast$.
%%
\end{enumerate}
%
\item \header{Note on Part (a) of 3.3.9 and 3.3.10} It gives that
\begin{center}
\begin{tabular}{ll}
\   & $(\strct{A}, \vect{a}) \equv^s (\strct{B}, \vect{b})$ \cr
iff & for every $m \in \nat$, $(\strct{A}, \vect{a}) \equv^s_m (\strct{B}, \vect{b})$ \cr
iff & for every $m \in \nat$, $\strct{B} \satis \ityp[s]{m}{\strct{A}, \vect{a}}[\vect{b}]$ \cr
iff & $\strct{B} \satis (\bland\limits_{m \in \nat} \ityp[s]{m}{\strct{A}, \vect{a}})[\vect{b}]$. \cr
\end{tabular}
\end{center}
%
\item \header{Note on 3.3.11} As in 3.2.11, the underlying vocabulary $\tau$ is relational and the results remain valid when $\tau$ contains only unary relation symbols.

The first statement that \emph{every model of $\conjextaxm{s}$ has at least $s$ elements} can be proved by induction on $s$, using the fact that $\conjextaxm{s + 1}$ is the conjunction of all $(s + 1)$-extension axioms with $\conjextaxm{s}$.

To prove the second statement that \emph{every two models $\strct{A}$ and $\strct{B}$ of $\epsilon_s$ are $s$-partially isomorphic}, consider the set
\[
I \defas \sett{\vect{a} \mapsto \vect{b}}{\mathmode{\vect{a} \in \cartpwr{(A \union \sete{\ast})}{s}}, \mathmode{\vect{b} \in \cartpwr{(B \union \sete{\ast})}{s}}, and \mathmode{\ityp{0}{\strct{A}, \vect{a}} = \ityp{0}{\strct{B}, \vect{b}}}}
\]
and show that it has the back and forth properties. In proving the forth property, for example, suppose that $\vect{a} \mapsto \vect{b}$ is an $s$-partial isomorphism from $\strct{A}$ to $\strct{B}$ where $\vect{a}$ and $\vect{b}$ can be assumed to consist of distinct elements and $a$ is not an entry of $\vect{a}$, to show there is a $b$ (that is different from all entries in $\vect{b}$) such that, say, $\vect{a}\sbst{a}{i} \mapsto \vect{b}\sbst{b}{i}$ is an $s$-partial isomorphism, there are two cases to consider:
\begin{enumerate}[(1)]
%%
\item If $a_i = \ast$ (\emph{extension} case), then we argue as for usual partial isomorphisms.
%%
\item If $a_i \neq \ast$ (\emph{substitution} case), then observe that $\vect{a}' \mapsto \vect{b}'$ is also an $s$-partial isomorphism where $\vect{a}'$ and $\vect{b}'$ are $\vect{a}$ and $\vect{b}$ with $a_i$ and $b_i$ removed, respectively. We then reduce this to the above extension case.
%%
\end{enumerate}
Since $\tau$ is relational, we have $\ast\etc\ast \mapsto \ast\etc\ast \in I$ and hence $I \neq \emptyset$.

Finally, observe for $s \geq 1$, we have $\randstrtheory \consq \conjextaxm{s}$, thus $\conjextaxm{s}$ is satisfiable (since $\randstrtheory$ has a model). With the third statement in the book, it also implies that for any $\inflog[\omega]$-sentence $\varphi$, $\randstrtheory \consq \varphi$ or $\randstrtheory \consq \neg\varphi$.
%
\item \header{Note on 3.3.13} See Note on Example 2.3.5 on how to extend the result to arbitrary $\tau$.
%
\item \header{Hint to 3.3.14} In part (a), note that for $s, m \in \nat$, the set
\[
\sett{\ityp{m}{\strct{A}, \vect{a}}}{\mathmode{\strct{A}} a structure and \mathmode{\vect{a} \in \cartpwr{A}{s}}}
\]
is finite.
%
\end{enumerate}
%end of section 3-----------------------------------------------------------------------------
\setcounter{chapter}{3}
\chapter{0-1 Laws}
Almost everywhere in this chapter the vocabulary $\tau$ is assumed to be nonempty and relational. In case $\tau = \emptyset$ (and hence obviously $\tau$ is relational), the results derived in section 4.1 are valid, and so are those in 4.2 and 4.3 but then are obvious because the only parametric class is the class $\modclass(\emptyset)$ of merely finite sets.
%section 1------------------------------------------------------------------------------------
\setcounter{section}{0}
\section{0-1 Laws for $\folog$ and $\inflog[\omega]$}
\begin{enumerate}[1.]
%
\item \header{Note on the Proof of Lemma 4.1.2} Denote
\[
\psi \defas \blor_{\varphi \in \Phi} \neg\varphi \lor \blor_{\varphi \in \cmpl{\Phi}} \varphi
\]
and let $\strct{A}$ with domain $A = \sete{1, \etc, n}$ be a random structure. Then
\[
\begin{array}{ll}
\    & \lprob[n](\neg\extaxm[\Phi]) \cr
=    & \paren{\mbox{the probability that \mathmode{\strct{A} \satis \neg\extaxm[\Phi]}}} \cr
=    & \paren{\parbox{40em}{the probability that there is an injective \mathmode{f : \sete{1, \etc, r} \to A} such that for every \mathmode{a \in A \setminus \rg(f)}, \mathmode{\strct{A} \satis \psi[f(1), \etc, f(r), a]}}} \cr
\leq & \sum\limits_{\text{injective \mathmode{f : \sete{1, \etc, r} \to A}}} \paren{\mbox{the probability that for every \mathmode{a \in A \setminus \rg(f)}, \mathmode{\strct{A} \satis \psi[f(1), \etc, f(r), a]}}} \cr
=    & \prmt{n}{r} \mul \paren{\mbox{the probability that for every \mathmode{a \in A \setminus \rg(f)} given injective \mathmode{f}, \mathmode{\strct{A} \satis \psi[f(1), \etc, f(r), a]}}} \cr
=    & \prmt{n}{r} \mul \paren{\parbox{35em}{the probability that for every \mathmode{a \in A \setminus \rg(f)} given injective \mathmode{f}, the assignment \mathmode{f(1), \etc, f(r), a} does not satisfy in \mathmode{\strct{A}} the type characterized by \mathmode{\Phi}}} \cr
=    & \prmt{n}{r} \mul \prod\limits_{\text{\mathmode{a \in A \setminus \rg(f)} given injective \mathmode{f}}} \paren{\parbox{25em}{the probability that the assignment \mathmode{f(1), \etc, f(r), a} does not satisfy in \mathmode{\strct{A}} the type characterized by \mathmode{\Phi}}} \cr
=    & \prmt{n}{r} \mul (\frac{c - 1}{c})^{n - r} \cr
\leq & n^r \mul (\frac{c - 1}{c})^{n - r}. \cr
\end{array}
\]

\begin{note}
The proof yields that for sufficiently large $n$, every extension axiom $\extaxm$ has a model of cardinality $n$.
\end{note}
%
\item \header{Hint to Exercise 4.1.6} By 2.2.6, the set
\[
\Phi \defas \sett{\hint{m}{\strct{C}}}{\mathmode{\strct{C}} is a structure}
\]
is finite. Thus, for $s \geq 1$, the binary relation $\isom_m$ induces a finite partition of $\lclass{s}(\tau)$ due to the equivalence between \refitem{(iii)} and \refitem{(iv)} of 2.3.4. Also by 4.1.5 the 0-1 law of $\folog\of{\tau}$, there is exactly one sentence $\varphi_0 \in \Phi$ with $\lprob(\varphi_0) = 1$ because
\[
1 = \lprob(\bunion_{\varphi \in \Phi} \modclass(\varphi)) = \sum_{\varphi \in \Phi} \lprob(\varphi).
\]
Finally, observe that the probability $\strct{A} \isom_m \strct{B}$ is $\geq$ the probability that both $\strct{A} \satis \varphi_0$ and $\strct{B} \satis \varphi_0$.
%
\item \header{Solution to Exercise 4.1.7} Let $\varphi \defas \extaxm[\Phi]$ be an $(r + 1)$-extension axiom, where $\Phi$ is a subset of $\Delta_{r + 1}$ as defined in 3.2.11. Assume that $\card{\Delta_{r + 1}} = c$ and $\card{\Phi} = k$.

Let $\strct{A}$ with domain $A = \sete{1, \etc, n}$ be a random structure constructed with a biased coin so that $R i_1 \etc i_m$ holds with probability $p$ where $0 < p < 1$. Then
\[
\begin{array}{ll}
\    & \lprob[n](\neg\extaxm[\Phi]) \cr
=    & \paren{\mbox{the probability that \mathmode{\strct{A} \satis \neg\extaxm[\Phi]}}} \cr
=    & \paren{\parbox{40em}{the probability that there is an injective \mathmode{f : \sete{1, \etc, r} \to A} such that for every \mathmode{a \in A \setminus \rg(f)}, \mathmode{\strct{A} \satis \psi[f(1), \etc, f(r), a]}}} \cr
\leq & \sum\limits_{\text{injective \mathmode{f : \sete{1, \etc, r} \to A}}} \paren{\mbox{the probability that for every \mathmode{a \in A \setminus \rg(f)}, \mathmode{\strct{A} \satis \psi[f(1), \etc, f(r), a]}}} \cr
=    & \prmt{n}{r} \mul \paren{\mbox{the probability that for every \mathmode{a \in A \setminus \rg(f)} given injective \mathmode{f}, \mathmode{\strct{A} \satis \psi[f(1), \etc, f(r), a]}}} \cr
=    & \prmt{n}{r} \mul \paren{\parbox{35em}{the probability that for every \mathmode{a \in A \setminus \rg(f)} given injective \mathmode{f}, the assignment \mathmode{f(1), \etc, f(r), a} does not satisfy in \mathmode{\strct{A}} the type characterized by \mathmode{\Phi}}} \cr
=    & \prmt{n}{r} \mul \prod\limits_{\text{\mathmode{a \in A \setminus \rg(f)} given injective \mathmode{f}}} \paren{\parbox{25em}{the probability that the assignment \mathmode{f(1), \etc, f(r), a} does not satisfy in \mathmode{\strct{A}} the type characterized by \mathmode{\Phi}}} \cr
=    & \prmt{n}{r} \mul (1 - p^k(1 - p)^{c - k})^{n - r} \cr
\leq & n^r \mul (1 - p^k(1 - p)^{c - k})^{n - r}. \cr
\end{array}
\]
It follows that $\lprob(\neg\extaxm[\Phi]) = \lim_{n \to \infty} \lprob[n](\neg\extaxm[\Phi]) = 0$.
%
\item \header{Solution to Exercise 4.1.8} Let $\strct{R}'$ denote the random structure defined as in the description of the exercise. Note that $\strct{R}'$ is a \emph{random variable}. In the following for $n > 0$ and for any (finite or infinite) structure $\strct{A}$ we let $\strct{A}_n$ be the substructure of $\strct{A}$ induced by the domain $\sete{1, \etc, n}$, provided that the domain $A$ of $\strct{A}$ contains $\sete{1, \etc, n}$ as a subset.

Observe that there are finitely many $\tau$-structures with domain $\sete{1, \etc, n}$, i.e.\ the class $\lclass{n}(\tau)$ is finite. Since $\tau$ is relational, the structures $\strct{B} \in \lclass{n}(\tau)$ altogether induce an \emph{equal partition} of
\[
C \defas \sett{\strct{A}}{\mathmode{\strct{A}} is a \mathmode{\tau}-structure of domain \mathmode{A = \sete{1, 2, \etc}}}
\]
such that the equivalence classes
\[
\setm{\strct{A} \in C}{\strct{A}_n = \strct{B}}
\]
for the structures $\strct{B}$ have the same size. Thus, if $K$ is a class of (finite) $\tau$-structures, then
\begin{equation}\label{e4_1_8+1}
(\text{the probability that for the random structure \mathmode{\strct{R}'}, its substructure \mathmode{\strct{R}'_n} is in \mathmode{K}}) = \lprob[n](K). \tag{$+$}
\end{equation}

Now consider any $(r + 1)$-extension axiom $\chi$ so that $\neg\chi$ is logically equivalent to
\[
\exists v_1 \etc \exists v_r \rho,
\]
where
\[
\rho = (\bland_{1 \leq i < j \leq r} \neg v_i = v_j \land \forall v_{r + 1} (\bland_{1 \leq i \leq r} \neg v_{r + 1} = v_i \lthen (\blor_{\varphi \in \Phi} \neg\varphi \lor \blor_{\varphi \in \cmpl{\Phi}} \varphi)))
\]
and $\Phi \subseteq \Delta_{r + 1}$. For every $r$-tuple $\vect{i} = (i_1, \etc, i_r) \in \cartpwr{\posint}{r}$ of positive integers, let
\[
m(\vect{i}) \defas
\begin{cases}
\max(i_1, \etc, i_r) & \text{if \mathmode{r > 0}} \cr
1                    & \text{otherwise (i.e.\ \mathmode{\vect{i} = \emptyseq})} \cr
\end{cases}
\]
and further for every positive integer $j \in \posint$ let $E^{\vect{i}}_j$ be the event that $\strct{R}'_{m(\vect{i}) + j} \satis \rho[\vect{i}]$. Then we have
\[
E^{\vect{i}}_1 \supseteq E^{\vect{i}}_2 \supseteq \etc
\]
and hence
\begin{equation}\label{e4_1_8+2}
(\text{the probability for the event \mathmode{\bintsc^\infty_{j = 1} E^{\vect{i}}_j}}) = \lim_{j \to \infty} (\text{the probability for the event \mathmode{E^{\vect{i}}_j}}). \tag{$\ast$}
\end{equation}
Moreover, it is true that
\[
\begin{array}{lll}
\    & (\text{the probability for the event \mathmode{E^{\vect{i}}_j}}) & \cr
\leq & (\text{the probability for the event that \mathmode{\strct{R}'_{m(\vect{i}) + j} \satis \exists v_1 \etc \exists v_r \rho}}) & \cr
=    & (\text{the probability for the event that \mathmode{\strct{R}'_{m(\vect{i}) + j} \satis \neg\chi}}) & \cr
=    & \lprob[m(\vect{i}) + j](\neg\chi) & \text{(by (\ref{e4_1_8+1}))} \cr
\end{array} 
\]
and hence
\begin{equation}\label{e4_1_8+3}
\lim_{j \to \infty} (\text{the probability for the event \mathmode{E^{\vect{i}}_j}}) \leq \lim_{j \to \infty} \lprob[m(\vect{i}) + j](\neg\chi) = 0 \tag{$\ast\ast$}
\end{equation}
by (the proof of) Lemma 4.1.2.

Finally, for brevity, for every $\vect{i} \in \cartpwr{\posint}{r}$ let $F^{\vect{i}}$ be the event that $\strct{R}' \satis \rho[\vect{i}]$; then we have
\begin{equation}\label{e4_1_8+4}
F^{\vect{i}} = \bintsc^\infty_{j = 1} E^{\vect{i}}_j. \tag{$\ast\ast\ast$}
\end{equation}
It follows that
\[
\begin{array}{lll}
\    & (\text{the probability that \mathmode{\strct{R}' \satis \neg\chi}}) & \cr
=    & (\text{the probability for the event \mathmode{\bunion_{\vect{i} \in \cartpwr{\posint}{r}} F^{\vect{i}}}}) & \cr
\leq & \sum_{\vect{i} \in \cartpwr{\posint}{r}} (\text{the probability for the event \mathmode{F^{\vect{i}}}}) & \cr
=    & \sum_{\vect{i} \in \cartpwr{\posint}{r}} (\text{the probability for the event \mathmode{\bintsc^\infty_{j = 1} E^{\vect{i}}_j}}) & \text{(by (\ref{e4_1_8+4}))} \cr
=    & \sum_{\vect{i} \in \cartpwr{\posint}{r}} \lim_{j \to \infty} (\text{the probability for the event \mathmode{E^{\vect{i}}_j}}) & \text{(by (\ref{e4_1_8+2}))} \cr
\leq & \sum_{\vect{i} \in \cartpwr{\posint}{r}} 0 & \text{(by (\ref{e4_1_8+3}))} \cr
=    & 0.
\end{array}
\]

Therefore, we conclude that, with probability $1$, $\strct{R}'$ is a model of any extension axiom $\chi$ and hence of $\randstrtheory$.

\begin{note}
The \emph{values} of the random variable $\strct{R}'$ for which $\strct{R}'$ is a model of $\randstrtheory$ are exactly those $\tau$-structures that are isomorphic to the infinite random structure $\infrandstr$.
\end{note}
%
\item \header{Hint to Exercise 4.1.9} (INCOMPLETE)
%
\end{enumerate}
%end of section 1-----------------------------------------------------------------------------


%section 2------------------------------------------------------------------------------------
\setcounter{section}{1}
\section{Parametric Classes}
\begin{enumerate}[1.]
%
\item \header{Note on Definition 4.2.1 and the Arguments Leading to Theorem 4.2.3} A parametric sentence may be a conjunction of sentences that involve different numbers of variables, in which the boolean combinations may contain different relation symbols. For example,
\[
\forall x Rxx \land \forall \distinct x y (\neg Rxy \lor Tyxy)
\]
is a parametric sentence.

In the construction process leading to statement (1) before 4.2.3, the definition of parametric sentences gives us that whether or not an $s$-tuple $b_1, \etc, b_s$ is in a relation $\intpr{R}{B}$ is independent of other tuples (not necessarily having length $s$) due to the fact that every atomic subformula of $\varphi_0$ contains exactly the variables that appear in its corresponding quantification and that every conjunct of $\varphi_0$ is a universal formula. This explains why the process works for parametric classes and why it may not work for classes of transitive properties (hence those properties are excluded in the consideration of parametric classes).

To obtain (2), let $\strct{A}, \strct{B}$ be two models of $\randstrtheory(\varphi_0)$ and choose $I$ as in 3.2.11. Obviously $I \neq \emptyset$ as $\emptymap \mapsto \emptymap \in I$ since $\strct{A} \satis \tr \land \neg\fls$ and $\strct{B} \satis \tr \land \neg\fls$. For the forth property, let $\vect{a} \mapsto \vect{b} \in I$ and $\vect{a} = a_1, \etc, a_r$ and $\vect{b} = b_1, \etc, b_r$ each consist of distinct entries. Let $a_{r + 1} \in A$ be different from entries of $\vect{a}$ and consider $\Phi \subseteq \Delta_{r + 1}$ such that
\[
\strct{A} \satis (\bland_{\varphi \in \Phi} \varphi \land \bland_{\varphi \in \cmpl{\Phi}} \neg\varphi)\withassgn{\vect{a}a_{r + 1}}.
\]
It follows that $\strct{A} \satis \exists \distinct v_1 \etc v_rv_{r + 1} (\bland_{\varphi \in \Phi} \varphi \land \bland_{\varphi \in \cmpl{\Phi}} \neg\varphi)$ and hence $\extaxm[\Phi]$ is compatible with $\varphi_0$, which gives $\extaxm[\Phi] \in \randstrtheory(\varphi_0)$. As $\strct{B} \satis \randstrtheory(\varphi_0)$, there is a $b_{r + 1} \in B$ different from the entries of $\vect{b}$ such that
\[
\strct{B} \satis (\bland_{\varphi \in \Phi} \varphi \land \bland_{\varphi \in \cmpl{\Phi}} \neg\varphi)\withassgn{\vect{b}b_{r + 1}}.
\]
Therefore, $\vect{a}a_{r + 1} \mapsto \vect{b}b_{r + 1} \in I$. Similarly for the back property.

As for (3), a slight modification can be made to the construction procedure for the countable model of $\randstrtheory$ in 3.2.11 to obtain one for $\randstrtheory(\varphi_0)$: Let $\seq{\alpha_n}{n \geq 0}$ be as in 3.2.11 except that now we only consider extension axioms $\extaxm \in \randstrtheory(\varphi_0)$. We choose $\strct{A}_0$ with $A_0 = \sete{0}$ so that $\strct{A}_0 \satis \varphi_0$ according to (1). Suppose that $\strct{A}_n$ has been defined such that $\strct{A}_n \satis \varphi_0$. For $\alpha_n = (\vect{m}, \extaxm[\Phi])$ where $\vect{m} = m_1, \etc, m_r$ that are distinct elements not greater than $n$ (if $r = 0$ then $\vect{m} = \emptyseq$) and $\Phi \subseteq \Delta_{r + 1}$, we show how to construct a model $\strct{A}_{n + 1}$ of $\varphi_0$ so that $(\strct{A}_{n + 1}, m_1, \etc, m_r, n + 1) \satis \Phi$:
\begin{itemize}
%%
\item Let, as in 3.2.11,
\begin{center}
$\strct{A}_{n + 1} \satis \varphi \withassgn{m_1, \etc, m_r, n + 1}$ \ \ \ iff \ \ \ $\varphi \in \Phi$.
\end{center}
%%
\item In case $n > r$ and $k \geq 2$, for every $2 \leq s \leq k$ and for every $s$-tuple $\vect{m}'$ that contains $n + 1$ and an entry $\leq n$ not in $\vect{m}$, we apply the construction process for (1) to $\vect{m}'$.
%%
\end{itemize}
Note that the second step is independent of $\Phi$, and thus in order to verify $\strct{A}_{n + 1} \satis \varphi_0$, it remains to show that for every conjunct
\[
\forall \distinct v_1 \etc v_t \psi
\]
of $\varphi_0$ and every $t$-tuple $\vect{a} = a_1, \etc, a_t$ (not necessarily distinct) that contains $n + 1$ and has all its other entries (if any) from $\vect{m}$, we have $\strct{A}_{n + 1} \satis \psi \withassgn{\vect{a}}$. In fact, letting $\psi'$ be obtained from $\psi$ by changing $v_i$ ($1 \leq i \leq t$) to $v_j$ ($1 \leq j \leq r$) if $a_i = m_j$ and to $v_{r + 1}$ if $a_i = n + 1$, we have that $\psi'$ is a conjunction of some formulas in $\Delta_{r + 1}$ and
\begin{center}
$\strct{A}_{n + 1} \satis \psi \withassgn{\vect{a}}$ \ \ \ iff \ \ \ $\strct{A}_{n + 1} \satis \psi' \withassgn{\vect{m}, n + 1}$.
\end{center}
As $\extaxm[\Phi]$ is compatible with $\varphi_0$, there is a model $\strct{B}$ of
\[
\varphi_0 \land \exists \distinct v_1 \etc v_r v_{r + 1} (\bland_{\varphi \in \Phi} \varphi \land \bland_{\varphi \in \cmpl{\Phi}} \neg\varphi).
\]
Consider an $(r + 1)$-tuple $\vect{b} = b_1, \etc, b_{r + 1}$ of distinct elements such that
\[
\strct{B} \satis (\bland_{\varphi \in \Phi} \varphi \land \bland_{\varphi \in \cmpl{\Phi}} \neg\varphi) \withassgn{\vect{b}},
\]
then
\begin{center}
$\strct{A}_{n + 1} \satis \psi' \withassgn{\vect{m}, n + 1}$ \ \ \ iff \ \ \ $\strct{B} \satis \psi' \withassgn{\vect{b}}$
\end{center}
(recall $\psi'$ is a conjunction of some formulas in $\Delta_{r + 1}$). But it is true that $\strct{B} \satis \psi' \withassgn{\vect{b}}$ because $\strct{B}$ is a model of $\varphi_0$. We conclude that $\strct{A}_{n + 1} \satis \psi \withassgn{\vect{a}}$.

On the other hand, for every $r \geq 0$ the theory $\randstrtheory(\varphi_0)$ contains at least one $(r + 1)$-extension axiom: By (1) there is a model $\strct{A}$ of $\varphi_0$ of cardinality $r + 1$, say, having domain $A = \sete{a_1, \etc, a_{r + 1}}$. Let $\Phi \subseteq \Delta_{r + 1}$ such that
\[
\strct{A} \satis (\bland_{\varphi \in \Phi} \varphi \land \bland_{\varphi \in \cmpl{\Phi}} \neg\varphi) \withassgn{a_1, \etc, a_{r + 1}}.
\]
We then have $\strct{A} \satis \exists \distinct v_1 \etc v_{r + 1} (\bland_{\varphi \in \Phi} \varphi \land \bland_{\varphi \in \cmpl{\Phi}} \neg\varphi)$ and hence the $(r + 1)$-extension axiom $\extaxm[\Phi]$ is compatible with $\varphi_0$ and is in $\randstrtheory(\varphi_0)$.

The above argument also yields that for every $s \geq 1$ and every $r < s$, $\conjextaxm[\varphi_0]{s}$ contains at least one $(r + 1)$-extension axiom as a conjunct.

Statement (3) immediately implies that for $s \geq 1$, $\conjextaxm[\varphi_0]{s}$ is satisfiable by a model of $\varphi_0$, because $\randstrtheory(\varphi_0) \consq \conjextaxm[\varphi_0]{s}$.

As in 3.3.11, we can conclude that
\begin{itemize}
%%
\item Every model of $\conjextaxm[\varphi_0]{s}$ has at least $s$ elements (by induction on $s$, using the fact that $\conjextaxm[\varphi_0]{s + 1}$ is the conjunction of $\conjextaxm[\varphi_0]{s}$ and at least one $(s + 1)$-extension axiom).
%%
\item Every two models of $\conjextaxm[\varphi_0]{s}$ are $s$-partially isomorphic (argue as for (2) above and as in 3.3.11 simultaneously), from which we obtain (4).
%%
\end{itemize}

Finally, let $A$ be a set such that $\card{A} = n \geq r + 1$ and let $\Phi \subseteq \Delta_{r + 1}$ such that $\extaxm[\Phi]$ is compatible with $\varphi_0$, we show how to construct a model $\strct{A}$ of $\varphi_0$ with domain $A$ such that $(\strct{A}, a_1 \etc a_ra) \satis \Phi$ in which $a_1, \etc, a_r, a$ are all distinct.\footnote{I suspect that there is a typo in the parenthesized argument after (5).} The construction consists of two steps:
\begin{itemize}
%%
\item For every $0 \leq m \leq r$ and $(m + 1)$-tuple $\vect{a} = a_{i_1} \etc a_{i_m}a$ with $1 \leq i_1 < i_2 < \etc < i_m \leq r$ (if $m = 0$ then $\vect{a} = a$) and $\varphi(v_{i_1}, \etc, v_{i_m}, v_{r + 1}) = Ry_1 \etc y_n$ with $\sete{y_1, \etc, y_n} = \sete{v_{i_1}, \etc, v_{i_m}, v_{r + 1}}$, let
\begin{center}
$\strct{A} \satis \varphi \withassgn{\vect{a}}$ \ \ \ iff \ \ \ $\varphi \in \Phi$.
\end{center}
%%
\item For all other tuples $\vect{a}'$ (which may or may not contain $a$ and is not a permutation of those $\vect{a}$ considered above) and all $R \in \tau$ we use the construction procedure leading to (1) for the $\vect{a}$-part of $\intpr{R}{A}$.
%%
\end{itemize}
It is easy to show that $\strct{A} \satis \varphi_0$ using the same argumentation as for (3) and thus omitted. Also observe that the second step is independent of $\Phi$, so we can focus on $\Phi$ when proving (5) using the same method as for 4.1.2.
%
\end{enumerate}
%end of section 2-----------------------------------------------------------------------------


%section 3------------------------------------------------------------------------------------
\setcounter{section}{2}
\section{Unlabeled 0-1 Laws}
\begin{enumerate}[1.]
%
\item \header{Two Methods to Check Automorphisms} Let $\strct{A}$ be a $\tau$-structure with domain $A = \sete{a_1, \etc, a_n}$ that is represented as a collection of tables one for each relation and constant symbol so that
\begin{itemize}
%%
\item the table for a $k$-ary $R$ has an entry for every $k$-tuple $(a_{i_1}, \etc, a_{i_k})$, with a $\bullet$ in it if $(a_{i_1}, \etc, a_{i_k}) \in \intpr{R}{A}$ and a $\circ$ otherwise; and
%%
\item the table for a constant $c$ has an entry for every element $a_i$, with a $\bullet$ in it if $\intpr{c}{A} = a_i$ and a $\circ$ otherwise.
%%
\end{itemize}
Below are two methods to check whether a permutation $\pi$ over $A$ is an automorphism of $\strct{A}$. For simplicity, let $\pi$ be a permutation over the set of indices $\sete{1, \etc, n}$ of elements rather than over $A$ itself.
\begin{enumerate}[(1)]
%%
\item In all tables change every element $a_i$ to $a_{\pi(i)}$ and then rearrange all entries into the usual lexicographic order based on the ordering $a_1, \etc, a_n$. Check if the resulting collection of tables is identical to that for $\strct{A}$ entrywise.
%%
\item In all tables rearrange the entries into the lexicographic order based on the ordering $a_{\pi(1)}, \etc, a_{\pi(n)}$. Check if the resulting collection of tables is identical to that for $\strct{A}$ entrywise.
%%
\end{enumerate}
In fact, it is easy to verify the correctness of the two methods. In the first method, we can think of every element $a_i$ as having a more informative subscript $i \to \pi(i)$ (it is intended to mean moving from \emph{source} $i$ to \emph{destination} $\pi(i)$) so that $a_i$ is actually $a_{i \to \pi(i)}$, and by rearranging all entries by the $\pi(i)$'s following the arrow $\to$ in the subscripts and comparing the resulting tables with the original ones we are in effect checking whether the structure is preserved or altered under $\pi$. Likewise, in the second method, we think of $a_i$ as subscripted $a_{i \from \inv{\pi}(i)}$ (recall the inverse $\inv{\pi}$ of a permutation $\pi$ is also a permutation), and rearranging all entries by the $\inv{\pi}(i)$'s (in the usual order of $1, \etc, n$) and comparing the resultant tables with the original ones does the same job.
%
\item \header{Note on Lemma 4.3.1} This lemma gives us the intuitive ideas:
\begin{enumerate}[\rm(1)]
%%
\item The more rigid a structure $\strct{A}$ is, the more automorphisms it has.
%%
\item The higher proportion of rigid structures a class $H$ has (in the unlabeled sense), the larger factor $\lclass{n}(H)$ is to $\uclass{n}(H)$. (This factor ranges between $1$ and $n!$.)
%%
\end{enumerate}

In the proof for part (c), a better choice for the symbol denoting a nontrivial automorphism in this context is $\sigma : \strct{A} \isom \strct{A}$. Since
\[
\sigma = (\inv{\pi} \cmps \pi) \cmps \sigma = \inv{\pi} \cmps (\pi \cmps \sigma),
\]
we have by ($\ast$) that
\[
\strct{A}_\pi = \strct{A}_{\pi \cmps \sigma}
\]
(here $\pi \cmps \sigma$ takes the place of $\rho$). In fact, if we view two automorphisms $\pi, \rho$ as equivalent when $\strct{A}_\pi = \strct{A}_\rho$, then the above argument gives that every equivalence class has the same size and furthermore,
\[
(\text{number of structures isomorphic to \mathmode{\strct{A}}}) \times (\text{number of automorphisms of \mathmode{\strct{A}}}) = n!
\]
in which only structures with domain $\sete{1, \etc, n}$ are considered.

Finally, note that $\lclass{n}$ is not necessarily a multiple of $\uclass{n}$: Consider the class $H$ of digraphs, then we have $\lclass{2}(H) = 4$ and $\uclass{2}(H) = 3$.
%
\item \header{Note on Lemma 4.3.2} Consider the vocabulary $\tau = \sete{P_1, \etc, P_k}$ of unary relation symbols ($k \geq 1$) and the two $\tau$-structures $\strct{A}, \strct{B}$ with $A = B = \sete{1, \etc, 2^k}$ such that every subset $S_1 \intsc \etc \intsc S_k$ of $A$ contains exactly one element where $S_i = \intpr{P_i}{A}$ or $S_i = A \setminus \intpr{P_i}{A}$ and $\intpr{P_i}{B} = \emptyset$.

Let $H$ consist of structures that are isomorphic to $\strct{A}$ or $\strct{B}$. Then we have that
\begin{center}
$\lprob[2^k](\rigid \wrt H) = \frac{\lclass{2^k}(\rigid \intsc H)}{\lclass{2^k}(H)} = \frac{(2^k)!}{(2^k)! + 1}$, \ and $\uprob[2^k](\rigid \wrt H) = \frac{\uclass{2^k}(\rigid \intsc H)}{\uclass{2^k}(H)} = \frac{1}{2}$.
\end{center}

This example shows that for some $n$ (in this case $n = 2^k$) the labeled and unlabeled probabilities can differ greatly.
%
\item \header{Note on Theorem 4.3.4} As explained in the footnote, here
\[
\lprob(K \wrt H) = \uprob(K \wrt H)
\]
actually stands for
\[
\lim_{n \to \infty} \lprob[n](K \wrt H) = \lim_{n \to \infty} \uprob[n](K \wrt H),
\]
which means that either both sides converge to the same value or both diverge.

Below we justify that $\lprob(K \wrt H) = \lprob(K \wrt \rigid \intsc H)$ and $\uprob(K \wrt H) = \uprob(K \wrt \rigid \intsc H)$ given $\uprob(\rigid \wrt H) = 1$. Since
\[
\lim_{n \to \infty} \frac{\lclass{n}(\cmpl{\rigid} \intsc H)}{\lclass{n}(H)} = \lim_{n \to \infty} \frac{\lclass{n}(H)}{\lclass{n}(H)} - \lim_{n \to \infty} \frac{\lclass{n}(\rigid \intsc H)}{\lclass{n}(H)} = 1 - \lprob(\rigid \wrt H) = 1 - 1 = 0
\]
and $0 \leq \frac{\lclass{n}(K \intsc \cmpl{\rigid} \intsc H)}{\lclass{n}(H)} \leq \frac{\lclass{n}(\cmpl{\rigid} \intsc H)}{\lclass{n}(H)}$, we have $\lim_{n \to \infty} \frac{\lclass{n}(K \intsc \cmpl{\rigid} \intsc H)}{\lclass{n}(H)} = 0$ and hence
\[
\begin{array}{lll}
\lprob(K \wrt H) & = & \displaystyle \lim_{n \to \infty} \frac{\lclass{n}(K \intsc H)}{\lclass{n}(H)} \cr
\ & = & \displaystyle \lim_{n \to \infty} \frac{\lclass{n}(K \intsc \rigid \intsc H)}{\lclass{n}(H)} + \lim_{n \to \infty} \frac{\lclass{n}(K \intsc \cmpl{\rigid} \intsc H)}{\lclass{n}(H)} \cr
\ & = & \displaystyle \lim_{n \to \infty} \frac{\lclass{n}(K \intsc \rigid \intsc H)}{\lclass{n}(\rigid \intsc H)} \mul \lim_{n \to \infty} \frac{\lclass{n}(\rigid \intsc H)}{\lclass{n}(H)} + \lim_{n \to \infty} \frac{\lclass{n}(K \intsc \cmpl{\rigid} \intsc H)}{\lclass{n}(H)} \cr
\ & = & \lprob(K \wrt \rigid \intsc H) \mul \lprob(\rigid \wrt H) + 0 \cr
\ & = & \lprob(K \wrt \rigid \intsc H).
\end{array}
\]
Likewise for $\uprob(K \wrt H) = \uprob(K \wrt \rigid \intsc H)$.
%
\item \header{Note on the Proof of 4.3.7} By definition, there are three cases where a nontrivial parametric class $H$ is nonfree:
\begin{enumerate}[(1)]
%%
\item The underlying vocabulary $\tau$ is empty.
%%
\item The underlying vocabulary $\tau$ contains only unary relation symbols.
%%
\item The underlying vocabulary $\tau$ contains a nonunary relation symbol, and for every $m \geq 2$, every $r$-ary $R$ ($r \geq m$), and every surjection $i : \sete{1, \etc, r} \to \sete{1, \etc, m}$,
\begin{center}
either \ \ $\varphi_0 \consq \forall \distinct x_1 \etc x_m Rx_{i(1)} \etc x_{i(r)}$ \ \ or \ \ $\varphi_0 \consq \forall \distinct x_1 \etc x_m \neg Rx_{i(1)} \etc x_{i(r)}$.
\end{center}
(Since $\varphi_0$ is nontrivial parametric, it has a model of size $m$ and hence at least one of the sentences
\[
\varphi_0 \land \exists \distinct x_1 \etc x_m Rx_{i(1)} \etc x_{i(r)}
\]
and
\[
\varphi_0 \land \exists \distinct x_1 \etc x_m \neg Rx_{i(1)} \etc x_{i(r)}
\]
is satisfiable.)
%%
\end{enumerate}
Case (1) is obvious because the only parametric class is $\modclass(\tau)$ (for the parametric sentence that is the empty conjunction) and $\lclass{n}(\tau) = \uclass{n}(\tau) = 1$. Case (3) can be reduced to case (2) because the ``shape'' of the relationship among different elements of the universe is fixed by $\varphi_0$; thus, $\tau$-structures $\strct{A}$ of the same size can only differ from each other in the induced unary relations $\setm{a}{\intpr{R}{\strct{A}} a \etc a}$, making those induced unary relations all that is relevant in the counting (cf.\ the last paragraph of the proof). The proof mainly handles case (2).

On the other hand, the statement given in the parentheses that \emph{any boolean combination of formulas $R_i x$ can be written as a disjunction of formulas $R^\alpha x$} (essentially) follows from the theorem on disjunctive normal form for propositional logic.
%
\end{enumerate}
%end of section 3-----------------------------------------------------------------------------


%section 5------------------------------------------------------------------------------------
\setcounter{section}{4}
\section{Probabilities of Monadic Second Order Properties}
\begin{enumerate}[1.]
%
\item \header{Note on the Proof of Lemma 4.5.3} From the proof of Claim 1, it looks like either $\psi^Y_<(x, y)$, $\psi^Z_<(x, y)$ and $\psi^U_<(x, y)$ mean $y <^Y x$, $y <^Z x$ and $y <^U x$, respectively or the statement on the right-hand side of ($\ast$) should be $a \geq b$.

There is a typo on line 4 of page 92: ``$i \in \sete{1, \etc, n}$'' should be ``$i \in \sete{1, \etc, t}$''. In fact, the probability $q$ that a given $i \in \sete{1, \etc, t}$ satisfies ($\ast$) is $p^{\frac{1}{2}r(r + 1)}(1 - p)^{\frac{1}{2}r(r - 1)}$.
%
\end{enumerate}
%end of section 5-----------------------------------------------------------------------------




\begin{thebibliography}{10}
\bibitem{EFT} H.-D.\ Ebbinghaus, J.\ Flum and W.\ Thomas \textsl{Mathematical Logic} 2nd edition, Springer, 1995.
%
\end{thebibliography}
\end{document}