\setcounter{chapter}{3}
\chapter{0-1 Laws}
%section 1------------------------------------------------------------------------------------
\begin{enumerate}[1.]
%
\item \header{Hint to Exercise 4.1.6} By 2.2.6, the set
\[
\Phi \defas \sett{\hint{m}{\strct{C}}}{\mathmode{\strct{C}} is a structure}
\]
is finite. Thus, for $s \geq 1$, the binary relation $\isom_m$ induces a finite partition of $\lclass{s}(\tau)$ due to the equivalence between \refitem{(iii)} and \refitem{(iv)} of 2.3.4. Also by 4.1.5 the 0-1 law of $\folog\of{\tau}$, there is exactly one sentence $\varphi_0 \in \Phi$ with $\lprob(\varphi_0) = 1$ because
\[
1 = \lprob(\bunion_{\varphi \in \Phi} \modclass(\varphi)) = \sum_{\varphi \in \Phi} \lprob(\varphi).
\]
Finally, observe that the probability $\strct{A} \isom_m \strct{B}$ is $\geq$ the probability that both $\strct{A} \satis \varphi_0$ and $\strct{B} \satis \varphi_0$.
%
\end{enumerate}
%end of section 1-----------------------------------------------------------------------------


%section 5------------------------------------------------------------------------------------
\setcounter{section}{4}
\section{Probabilities of Monadic Second Order Properties}
\begin{enumerate}[1.]
%
\item \header{Note on the Proof of Lemma 4.5.3} From the proof of Claim 1, it looks like either $\psi^Y_<(x, y)$, $\psi^Z_<(x, y)$ and $\psi^U_<(x, y)$ mean $y <^Y x$, $y <^Z x$ and $y <^U x$, respectively or the statement on the right-hand side of ($\ast$) should be $a \geq b$.

There is a typo on line 4 of page 92: ``$i \in \sete{1, \etc, n}$'' should be ``$i \in \sete{1, \etc, t}$''. In fact, the probability $q$ that a given $i \in \sete{1, \etc, t}$ satisfies ($\ast$) is $p^{\frac{1}{2}r(r + 1)}(1 - p)^{\frac{1}{2}r(r - 1)}$.
%
\end{enumerate}
%end of section 5-----------------------------------------------------------------------------


