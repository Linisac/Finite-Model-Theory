\setcounter{chapter}{3}
\chapter{0-1 Laws}
%section 1------------------------------------------------------------------------------------
\begin{enumerate}[1.]
%
\item \header{Note on the Proof of Lemma 4.1.2} Denote
\[
\psi \defas \blor_{\varphi \in \Phi} \neg\varphi \lor \blor_{\varphi \in \cmpl{\Phi}} \varphi
\]
and let $\strct{A}$ with domain $A = \sete{1, \etc, n}$ be a random structure. Then
\[
\begin{array}{ll}
\    & \lprob[n](\neg\extaxm[\Phi]) \cr
=    & \paren{\mbox{the probability that \mathmode{\strct{A} \satis \neg\extaxm[\Phi]}}} \cr
=    & \paren{\parbox{40em}{the probability that there is an injective \mathmode{f : \sete{1, \etc, r} \to A} such that for every \mathmode{a \in A \setminus \rg(f)}, \mathmode{\strct{A} \satis \psi[f(1), \etc, f(r), a]}}} \cr
\leq & \sum\limits_{\text{injective \mathmode{f : \sete{1, \etc, r} \to A}}} \paren{\mbox{the probability that for every \mathmode{a \in A \setminus \rg(f)}, \mathmode{\strct{A} \satis \psi[f(1), \etc, f(r), a]}}} \cr
=    & \prmt{n}{r} \mul \paren{\mbox{the probability that for every \mathmode{a \in A \setminus \rg(f)} given injective \mathmode{f}, \mathmode{\strct{A} \satis \psi[f(1), \etc, f(r), a]}}} \cr
=    & \prmt{n}{r} \mul \paren{\parbox{35em}{the probability that for every \mathmode{a \in A \setminus \rg(f)} given injective \mathmode{f}, the assignment \mathmode{f(1), \etc, f(r), a} does not satisfy in \mathmode{\strct{A}} the type characterized by \mathmode{\Phi}}} \cr
=    & \prmt{n}{r} \mul \prod\limits_{\text{\mathmode{a \in A \setminus \rg(f)} given injective \mathmode{f}}} \paren{\parbox{25em}{the probability that the assignment \mathmode{f(1), \etc, f(r), a} does not satisfy in \mathmode{\strct{A}} the type characterized by \mathmode{\Phi}}} \cr
=    & \prmt{n}{r} \mul (\frac{c - 1}{c})^{n - r} \cr
\leq & n^r \mul (\frac{c - 1}{c})^{n - r}. \cr
\end{array}
\]
%
\item \header{Hint to Exercise 4.1.6} By 2.2.6, the set
\[
\Phi \defas \sett{\hint{m}{\strct{C}}}{\mathmode{\strct{C}} is a structure}
\]
is finite. Thus, for $s \geq 1$, the binary relation $\isom_m$ induces a finite partition of $\lclass{s}(\tau)$ due to the equivalence between \refitem{(iii)} and \refitem{(iv)} of 2.3.4. Also by 4.1.5 the 0-1 law of $\folog\of{\tau}$, there is exactly one sentence $\varphi_0 \in \Phi$ with $\lprob(\varphi_0) = 1$ because
\[
1 = \lprob(\bunion_{\varphi \in \Phi} \modclass(\varphi)) = \sum_{\varphi \in \Phi} \lprob(\varphi).
\]
Finally, observe that the probability $\strct{A} \isom_m \strct{B}$ is $\geq$ the probability that both $\strct{A} \satis \varphi_0$ and $\strct{B} \satis \varphi_0$.
%
\item \header{Solution to Exercise 4.1.7} Let $\varphi \defas \extaxm[\Phi]$ be an $(r + 1)$-extension axiom, where $\Phi$ is a subset of $\Delta_{r + 1}$ as defined in 3.2.11. Assume that $\card{\Delta_{r + 1}} = c$ and $\card{\Phi} = k$.

Let $\strct{A}$ with domain $A = \sete{1, \etc, n}$ be a random structure constructed with a biased coin so that $R i_1 \etc i_m$ holds with probability $p$ where $0 < p < 1$. Then
\[
\begin{array}{ll}
\    & \lprob[n](\neg\extaxm[\Phi]) \cr
=    & \paren{\mbox{the probability that \mathmode{\strct{A} \satis \neg\extaxm[\Phi]}}} \cr
=    & \paren{\parbox{40em}{the probability that there is an injective \mathmode{f : \sete{1, \etc, r} \to A} such that for every \mathmode{a \in A \setminus \rg(f)}, \mathmode{\strct{A} \satis \psi[f(1), \etc, f(r), a]}}} \cr
\leq & \sum\limits_{\text{injective \mathmode{f : \sete{1, \etc, r} \to A}}} \paren{\mbox{the probability that for every \mathmode{a \in A \setminus \rg(f)}, \mathmode{\strct{A} \satis \psi[f(1), \etc, f(r), a]}}} \cr
=    & \prmt{n}{r} \mul \paren{\mbox{the probability that for every \mathmode{a \in A \setminus \rg(f)} given injective \mathmode{f}, \mathmode{\strct{A} \satis \psi[f(1), \etc, f(r), a]}}} \cr
=    & \prmt{n}{r} \mul \paren{\parbox{35em}{the probability that for every \mathmode{a \in A \setminus \rg(f)} given injective \mathmode{f}, the assignment \mathmode{f(1), \etc, f(r), a} does not satisfy in \mathmode{\strct{A}} the type characterized by \mathmode{\Phi}}} \cr
=    & \prmt{n}{r} \mul \prod\limits_{\text{\mathmode{a \in A \setminus \rg(f)} given injective \mathmode{f}}} \paren{\parbox{25em}{the probability that the assignment \mathmode{f(1), \etc, f(r), a} does not satisfy in \mathmode{\strct{A}} the type characterized by \mathmode{\Phi}}} \cr
=    & \prmt{n}{r} \mul (1 - p^k(1 - p)^{c - k})^{n - r} \cr
\leq & n^r \mul (1 - p^k(1 - p)^{c - k})^{n - r}. \cr
\end{array}
\]
It follows that $\lprob(\neg\extaxm[\Phi]) = \lim_{n \to \infty} \lprob[n](\neg\extaxm[\Phi]) = 0$.
%
\item \header{Solution to Exercise 4.1.8} Let $\strct{R}'$ denote the random structure defined as in the description of the exercise. Note that $\strct{R}'$ is a \emph{random variable}. In the following for $n > 0$ and for any (finite or infinite) structure $\strct{A}$ we let $\strct{A}_n$ be the substructure of $\strct{A}$ induced by the domain $\sete{1, \etc, n}$, provided that the domain $A$ of $\strct{A}$ contains $\sete{1, \etc, n}$ as a subset.

Observe that there are finitely many $\tau$-structures with domain $\sete{1, \etc, n}$, i.e.\ the class $\lclass{n}(\tau)$ is finite. Since $\tau$ is relational, the structures $\strct{B} \in \lclass{n}(\tau)$ altogether induce an \emph{equal partition} of
\[
C \defas \sett{\strct{A}}{\mathmode{\strct{A}} is a \mathmode{\tau}-structure of domain \mathmode{A = \sete{1, 2, \etc}}}
\]
such that the equivalence classes
\[
\setm{\strct{A} \in C}{\strct{A}_n = \strct{B}}
\]
for the structures $\strct{B}$ have the same size. Thus, if $K$ is a class of (finite) $\tau$-structures, then
\begin{equation}\label{e4_1_8+1}
(\text{the probability that for the random structure \mathmode{\strct{R}'}, its substructure \mathmode{\strct{R}'_n} is in \mathmode{K}}) = \lprob[n](K). \tag{$+$}
\end{equation}

Now consider any $(r + 1)$-extension axiom $\chi$ so that $\neg\chi$ is logically equivalent to
\[
\exists v_1 \etc \exists v_r \rho,
\]
where
\[
\rho = (\bland_{1 \leq i < j \leq r} \neg v_i = v_j \land \forall v_{r + 1} (\bland_{1 \leq i \leq r} \neg v_{r + 1} = v_i \lthen (\blor_{\varphi \in \Phi} \neg\varphi \lor \blor_{\varphi \in \cmpl{\Phi}} \varphi)))
\]
and $\Phi \subseteq \Delta_{r + 1}$. For every $r$-tuple $\vect{i} = (i_1, \etc, i_r) \in \cartpwr{\posint}{r}$ of positive integers, let
\[
m(\vect{i}) \defas
\begin{cases}
\max(i_1, \etc, i_r) & \text{if \mathmode{r > 0}} \cr
1                    & \text{otherwise (i.e.\ \mathmode{\vect{i} = \emptyseq})} \cr
\end{cases}
\]
and further for every positive integer $j \in \posint$ let $E^{\vect{i}}_j$ be the event that $\strct{R}'_{m(\vect{i}) + j} \satis \rho[\vect{i}]$. Then we have
\[
E^{\vect{i}}_1 \supseteq E^{\vect{i}}_2 \supseteq \etc
\]
and hence
\begin{equation}\label{e4_1_8+2}
(\text{the probability for the event \mathmode{\bintsc^\infty_{j = 1} E^{\vect{i}}_j}}) = \lim_{j \to \infty} (\text{the probability for the event \mathmode{E^{\vect{i}}_j}}). \tag{$\ast$}
\end{equation}
Moreover, it is true that
\[
\begin{array}{lll}
\    & (\text{the probability for the event \mathmode{E^{\vect{i}}_j}}) & \cr
\leq & (\text{the probability for the event that \mathmode{\strct{R}'_{m(\vect{i}) + j} \satis \exists v_1 \etc \exists v_r \rho}}) & \cr
=    & (\text{the probability for the event that \mathmode{\strct{R}'_{m(\vect{i}) + j} \satis \neg\chi}}) & \cr
=    & \lprob[m(\vect{i}) + j](\neg\chi) & \text{(by (\ref{e4_1_8+1}))} \cr
\end{array} 
\]
and hence
\begin{equation}\label{e4_1_8+3}
\lim_{j \to \infty} (\text{the probability for the event \mathmode{E^{\vect{i}}_j}}) \leq \lim_{j \to \infty} \lprob[m(\vect{i}) + j](\neg\chi) = 0 \tag{$\ast\ast$}
\end{equation}
by (the proof of) Lemma 4.1.2.

Finally, for brevity, for every $\vect{i} \in \cartpwr{\posint}{r}$ let $F^{\vect{i}}$ be the event that $\strct{R}' \satis \rho[\vect{i}]$; then we have
\begin{equation}\label{e4_1_8+4}
F^{\vect{i}} = \bintsc^\infty_{j = 1} E^{\vect{i}}_j. \tag{$\ast\ast\ast$}
\end{equation}
It follows that
\[
\begin{array}{lll}
\    & (\text{the probability that \mathmode{\strct{R} \satis \neg\chi}}) & \cr
=    & (\text{the probability for the event \mathmode{\bunion_{\vect{i} \in \cartpwr{\posint}{r}} F^{\vect{i}}}}) & \cr
\leq & \sum_{\vect{i} \in \cartpwr{\posint}{r}} (\text{the probability for the event \mathmode{F^{\vect{i}}}}) & \cr
=    & \sum_{\vect{i} \in \cartpwr{\posint}{r}} (\text{the probability for the event \mathmode{\bintsc^\infty_{j = 1} E^{\vect{i}}_j}}) & \text{(by (\ref{e4_1_8+4}))} \cr
=    & \sum_{\vect{i} \in \cartpwr{\posint}{r}} \lim_{j \to \infty} (\text{the probability for the event \mathmode{E^{\vect{i}}_j}}) & \text{(by (\ref{e4_1_8+2}))} \cr
\leq & \sum_{\vect{i} \in \cartpwr{\posint}{r}} 0 & \text{(by (\ref{e4_1_8+3}))} \cr
=    & 0.
\end{array}
\]

Therefore, we conclude that, with probability $1$, $\strct{R}'$ is a model of any extension axiom $\chi$ and hence of $\theory_\rand$.

\begin{note}
The \emph{values} of the random variable $\strct{R}'$ for which $\strct{R}'$ is a model of $\theory_\rand$ are exactly those $\tau$-structures that are isomorphic to the infinite random structure $\strct{R}$.
\end{note}
%
\item \header{Hint to Exercise 4.1.9}
%
\end{enumerate}
%end of section 1-----------------------------------------------------------------------------


%section 5------------------------------------------------------------------------------------
\setcounter{section}{4}
\section{Probabilities of Monadic Second Order Properties}
\begin{enumerate}[1.]
%
\item \header{Note on the Proof of Lemma 4.5.3} From the proof of Claim 1, it looks like either $\psi^Y_<(x, y)$, $\psi^Z_<(x, y)$ and $\psi^U_<(x, y)$ mean $y <^Y x$, $y <^Z x$ and $y <^U x$, respectively or the statement on the right-hand side of ($\ast$) should be $a \geq b$.

There is a typo on line 4 of page 92: ``$i \in \sete{1, \etc, n}$'' should be ``$i \in \sete{1, \etc, t}$''. In fact, the probability $q$ that a given $i \in \sete{1, \etc, t}$ satisfies ($\ast$) is $p^{\frac{1}{2}r(r + 1)}(1 - p)^{\frac{1}{2}r(r - 1)}$.
%
\end{enumerate}
%end of section 5-----------------------------------------------------------------------------


