\setcounter{chapter}{2}
\chapter{More on Games}
%section 1-------------------------------------------------------------------------------------
\section{Second-Order Logic}
\begin{enumerate}[1.]
%
\item \header{Hint to Exercise 3.1.2} Argue as in 2.2.2, 2.2.4, and 2.2.6-8.
%
\end{enumerate}
%end of section 1-----------------------------------------------------------------------------


%section 2------------------------------------------------------------------------------------
\section{Infinitary Logic: The Logics $\inflog$ and $\dlog{\omega_1}$}
\begin{enumerate}[1.]
%
\item \header{Subformulas of $\inflog$-Sentences Only Have Finitely Many Free Variables} In fact, one can show by induction on the formation of formulas that if $\varphi \in \inflog$ has infinitely many free variables, then any $\inflog$-formula having $\varphi$ as a subformula must also have infinitely many free variables.
%
\item \header{Note on 3.2.3} One can easily see that $\varphi(\vect{x})$ is equivalent to a countable conjunction of first-order formulas:
\[
\bland \setm{\cardexactly{n} \lthen \blor\setm{\hint{\card{A} + 1}{\strct{A}, \vect{a}}(\vect{x})}{\card{A} = n, \vect{a} \in A, \strct{A} \satis \varphi[\vect{a}]}}{n \geq 1}.
\]
Note that the disjunction in every conjunct above is finite.
%
\item \header{Note on 3.2.4 and 3.2.6} In fact, for any (finite) vocabulary $\tau$ and any $\tau$-structures $\strct{A}, \strct{B}$ the following statements are equivalent (cf.\ Chapter XII in \cite{EFT} for more details):
\begin{enumerate}[(1)]
%%
\item $\strct{A} \equv \strct{B}$, i.e.\ $\strct{A}$ and $\strct{B}$ are \emph{elementarily equivalent} ($\strct{A}$ and $\strct{B}$ satisfy the same $\folog\of{\tau}$-sentences)
%%
\item $\strct{A} \isom_\fin \strct{B}$, namely $\strct{A}$ and $\strct{B}$ are \emph{finitely isomorphic} (there is an infinite sequence $\seq{I_j}{j \in \nat}$ that satisfies the properties \refitem{(a) - (c)} in Definition 2.3.1 except that ``$j < m$'' is replaced by ``$j \in \nat$'' in conditions \refitem{(b)} and \refitem{(c)})
%%
\item For $j \in \nat$, the duplicator wins the game $\game{j}(\strct{A}, \strct{B})$
%%
\item The duplicator wins the game $\fingame(\strct{A}, \strct{B})$ in which the spoiler begins the game by first choosing a natural number $r$ and then the game proceeds as in $\game{r}(\strct{A}, \strct{B})$.
%%
\end{enumerate}
Using Corollary 2.3.4, we obtain a further equivalent statement:
\begin{enumerate}[(1)]
\setcounter{enumii}{4}
%%
\item $\strct{B} \satis \bland_{j \in \nat} \hint{j}{\strct{A}}$,
%%
\end{enumerate}
and, using Lemma 2.2.4(b), yet another equivalent statement, in parallel to Corollary 2.3.4(ii):
\begin{enumerate}[(1)]
\setcounter{enumii}{5}
%%
\item $\seq{\winpos{j}(\strct{A}, \strct{B})}{j \in \nat} : \strct{A} \isom_\fin \strct{B}$.
%%
\end{enumerate}

The above should be easy to generalize so that we have $(\strct{A}, \vect{a}), (\strct{B}, \vect{b})$ in place of $\strct{A}, \strct{B}$, and $\vect{a} \mapsto \vect{b} \in I_j$ (or $\vect{a} \mapsto \vect{b} \in \winpos{j}(\strct{A}, \strct{B})$) for all $j$ in \refitem{(2)} (or \refitem{(6)}, respectively). For example, we can regard $(\strct{A}, \vect{a})$ and $(\strct{B}, \vect{b})$ as the $\tau \union \sete{\vect{c}}$-expansion of $\strct{A}$ and of $\strct{B}$ in which $\vect{c}$ is interpreted by $\vect{a}$ and $\vect{b}$, respectively.

Notice that, however, the condition that the duplicator wins $\fingame(\strct{A}, \vect{a}, \strct{B}, \vect{b})$ generally does not imply that he wins $\game{\infty}(\strct{A}, \vect{a}, \strct{B}, \vect{b})$ although the converse is obviously true. See Exercise XII.1.10 in \cite{EFT} to give an example of two finitely isomorphic structures that are not partially isomorphic; the same exercise problem also asks to give an example of two partially isomorphic structures that are not isomorphic.

On the other hand, consider the $\emptyvoc$-structures (i.e.\ sets) $\strct{A}, \strct{B}$ that consist of the domains $A = \sete{0, 1}, B = \sete{0, 1, 2}$, respectively. Then it follows that $\strct{A} \isom_2 \strct{B}$ but not $\strct{A} \isom_\fin \strct{B}$ (since not $\strct{A} \isom_3 \strct{B}$).

To summarize, we have the successively weaker notions (where \refitem{(1)} clearly implies \refitem{(2)}):
\begin{enumerate}[(1)]
%%
\item $\strct{A} \isom \strct{B}$
%%
\item $\strct{A} \isom_\partially \strct{B}$
%%
\item $\strct{A} \isom_\fin \strct{B}$
%%
\item $\strct{A} \isom_j \strct{B}$ in which $j \in \nat$.
%%
\end{enumerate}
%
\item \header{Note on Theorem 3.2.7} In the direction from \refitem{(iii)} to \refitem{(iv)} of the proof, the length $s$ of tuples in \refitem{($\ast$)} is better to be replaced by say $r$, since $s$ is fixed for the length of $\vect{a}$ and $\vect{b}$ whereas the tuples in \refitem{($\ast$)} may have different lengths. Also, there is a typo in the direction from \refitem{(iv)} to \refitem{(iii)}: ``$a \in I$'' should change to ``$a \in A$''.

On the other hand, this theorem does not seem to have a corresponding statement \refitem{(iv)} to that of Theorem 2.3.3. The statement $(\strct{B}, \vect{b}) \satis \bland_{j \in \nat} \hint{j}{\strct{A}, \vect{a}}$ does not work because it is equivalent to $(\strct{A}, \vect{a}) \equv (\strct{B}, \vect{b})$, namely $(\strct{A}, \vect{a}), (\strct{B}, \vect{b})$ satisfy the same first-order formulas $\varphi$ in which it is decisive whether or not $\strct{A} \satis \varphi[\vect{a}]$ (or equivalently, $\strct{B} \satis \varphi[\vect{b}]$) is the case (see the discussion in Note on 3.2.4 and 3.2.6).
%
\item \header{Note on \refitem{(iii)} of 3.2.8} To transition from \refitem{(iii)} of 3.2.7 for $s = 0$, one refers to 2.3.2 (also cf.\ the transition from 2.3.3 to 2.3.4 using 2.3.2).
%
\item \header{Note on 3.2.11}
\begin{enumerate}[(1)]
%%
\item For $r \geq 0$, the set $\Delta_{r + 1}$ is finite.
%%
\item A $2$-extension axiom has the form
\[
\forall v_1 \exists v_2 (v_1 \neq v_2 \land \bland_{\varphi \in \Phi} \varphi \land \bland_{\varphi \in \cmpl{\Phi}} \neg\varphi).
\]
%%
\item In fact, the condition ``$\hint{0}{\strct{A}, \vect{a}} = \hint{0}{\strct{B}, \vect{b}}$'' in the definition of the set $I$ of maps is equivalent to ``$\vect{a} \mapsto \vect{b}$ is a partial isomorphism from $\strct{A}$ to $\strct{B}$'', by 2.2.5 and parts (b) and (c) of 2.2.7, whether or not $\tau$ is relational; in other words, $I = \partisoms(\strct{A}, \strct{B})$.

Moreover, by the assumption that $\tau$ is relational (containing no constants), the empty map $\emptymap$ is a partial isomorphism from $\strct{A}$ to $\strct{B}$ (see 2.2.2(a)) and coincides with $\emptyseq \mapsto \emptyseq$; it follows that $\hint{0}{\strct{A}, \emptyseq} = \hint{0}{\strct{B}, \emptyseq}$ and is equal to $\tr \land \neg\fls$ (cf.\ Part B in Chapter 1).

In case $\tau$ is not relational, however, the empty map $\emptymap$ and $\emptyseq \mapsto \emptyseq$ are not identical. If in addition $\tau$ contains two constants $c_1, c_2$ so that $\intpr{c_1}{\strct{A}} = \intpr{c_2}{\strct{A}}$ but $\intpr{c_1}{\strct{B}} \neq \intpr{c_2}{\strct{B}}$, then $\emptyseq \mapsto \emptyseq$ is not even a partial isomorphism and $I = \emptyset$.
%%
\item In the proof $I$ has the forth property, note that
\[
\begin{array}{lll}
\ & \hint{0}{\strct{B}, \vect{b}b_{r + 1}} & \cr
= & \hint{0}{\strct{B}, \vect{b}} \land \bland\limits_{1 \leq i \leq r} \neg v_i = v_{r + 1} \land \bland\limits_{\varphi \in \Phi} \varphi \land \bland\limits_{\varphi \in \cmpl{\Phi}} \neg\varphi & \text{(since \mathmode{(\strct{B}, \vect{b}b_{r + 1}) \satis \Phi})} \cr
= & \hint{0}{\strct{A}, \vect{a}} \land \bland\limits_{1 \leq i \leq r} \neg v_i = v_{r + 1}\land \bland\limits_{\varphi \in \Phi} \varphi \land \bland\limits_{\varphi \in \cmpl{\Phi}} \neg\varphi & \text{(since \mathmode{\vect{a} \mapsto \vect{b} \in I})} \cr
= & \hint{0}{\strct{A}, \vect{a}a_{r + 1}} & \text{(since \mathmode{(\strct{A}, \vect{a}a_{r + 1}) \satis \Phi})}. \cr
\end{array}
\]
%%
\item In the construction of a countable model $\strct{A}$ of $\randstrtheory$ by means of the infinite sequence of $\strct{A}_n$'s, note that $r \leq n + 1$ is implied by the requirement that $\vect{m}$ consists of distinct entries and all entries are not greater than $n$.

However, this construction procedure does not specify, given $\strct{A}_n$ and $\alpha_n = (\vect{m}, \extaxm)$, the relationship in $\strct{A}_{n + 1}$ between the element $n + 1$ and any other tuple than $\vect{m}$ in terms of the interpretation of relation symbols $R \in \tau$. Thus, the relationship can be arbitrarily defined. (Nevertheless, this freedom in defining the relationship is somewhat restricted in the construction procedure for a countable model of $\randstrtheory(\varphi_0)$ mentioned before (3) leading to Theorem 4.2.3.)
%%
\item If $\tau$ only contains unary relation symbols, say $\tau = \sete{R_1, \etc, R_m}$. Then for every model $\strct{A}$ of $\randstrtheory$, the the sets $\intpr{R_1}{\strct{A}}, \etc, \intpr{R_m}{\strct{A}}$ yield a partition of the universe $A$ of $\strct{A}$ into $2^m$ subsets, each of which consists of infinitely many elements.

\begin{remark}
The results given in textbook remain valid in this case.
\end{remark}
%%
\end{enumerate}
%
\item \header{Hint to Exercise 3.2.13} Note that $x_1$ is the first element of the tuple $\vect{x}$ and therefore the sentence $\forall \vect{x} (\existexactly{1}x F\vect{x}x \land F\vect{x}x_1)$ formalizes the idea that $F$ is a (total) function that projects a tuple of parameter(s) onto its first parameter.
%
\item \header{Solution to Exercise 3.2.14} Without loss of generality let us assume that $A$ is finite and that $\card{A} = \min\sete{\card{A}, \card{B}}$. We also assume, for simplicity, that partial isomorphisms from $\strct{A}$ to $\strct{B}$ take the form $\vect{a} \mapsto \vect{b}$ (having finite domains and ranges) in which $\vect{a}$ and $\vect{b}$ consist of distinct elements.
\medskip\\
Now we distinguish three cases:
\begin{enumerate}[(1)]
%%
\item $\strct{A} \isom \strct{B}$.
%%
\item For any $m \in \nat$, $\vect{a} \in A$ and $\vect{b} \in B$, not $(\strct{A}, \vect{a}) \isom_m (\strct{B}, \vect{b})$.
%%
\item Not $\strct{A} \isom \strct{B}$, and there are $m \in \nat$, $\vect{a} \in A$ and $\vect{b} \in B$ such that $(\strct{A}, \vect{a}) \isom_m (\strct{B}, \vect{b})$.
%%
\end{enumerate}
In case (1), we have that $\winpos{\infty}(\strct{A}, \strct{B}) = \winpos{0}(\strct{A}, \strct{B}) = \partisoms(\strct{A}, \strct{B}) \neq \emptyset$, which contains all isomorphisms from $\strct{A}$ to $\strct{B}$. Thus, set $m_0 \defas 0$.
\medskip\\
In case (2), it follows in particular that for any $\vect{a} \in A$ and $\vect{b} \in B$, not $(\strct{A}, \vect{a}) \isom_0 (\strct{B}, \vect{b})$. Hence $\winpos{\infty}(\strct{A}, \strct{B}) = \winpos{0}(\strct{A}, \strct{B}) = \partisoms(\strct{A}, \strct{B}) = \emptyset$. Thus, set $m_0 \defas 0$.
\medskip\\
In case (3), we may assume, with no loss of generality, that $m \leq \card{A}$ and for any $\vect{a'} \in A$ and $\vect{b'} \in B$, not $(\strct{A}, \vect{a'}) \isom_{m + 1} (\strct{B}, \vect{b'})$. So we have
\[
\winpos{0}(\strct{A}, \strct{B}) \neq \emptyset, \etc, \winpos{m}(\strct{A}, \strct{B}) \neq \emptyset
\]
(cf.\ 2.2.4(c)) but
\[
\winpos{\infty}(\strct{A}, \strct{B}) = \winpos{m + 1}(\strct{A}, \strct{B}) = \emptyset.
\]
It then suffices to show that for $j < m$, 
\[
\winpos{j}(\strct{A}, \strct{B}) \neq \winpos{j + 1}(\strct{A}, \strct{B}).
\]
In fact, in any $\winpos{j + 1}(\strct{A}, \strct{B})$ there must be a maximal -- in the sense of set inclusion -- partial isomorphism $q$ (i.e.\ there is $q \in \winpos{j + 1}(\strct{A}, \strct{B})$ such that there is no $q' \in \winpos{j + 1}(\strct{A}, \strct{B})$ with $q' \supset q$) for which there is an $a \in A$ with $a \notin \dm(q)$, since $A$ is finite and $\strct{A}$ and $\strct{B}$ are not isomorphic (also cf.\ the proof of 2.2.3(b)). By 2.2.4(b) there is a $p \in \winpos{j}(\strct{A}, \strct{B})$ with $\dm(p) = \dm(q) \union \sete{a}$; note that $p \notin \winpos{j + 1}(\strct{A}, \strct{B})$ because $p \supset q$ and $q$ is maximal in $\winpos{j + 1}(\strct{A}, \strct{B})$. Thus, set $m_0 \defas m + 1$.
\begin{remark}
By definition (cf.\ 3.2.4) and 2.2.4(c), we immediately have
\[
\winpos{0}(\strct{A}, \strct{B}) \supseteq \etc \supseteq \winpos{m}(\strct{A}, \strct{B}) \supseteq \etc \supseteq \winpos{\infty}(\strct{A}, \strct{B}).
\]
\end{remark}
%
\end{enumerate}
%end of section 2-----------------------------------------------------------------------------


%section 3------------------------------------------------------------------------------------
\section{The Logics $\folog[s]$ and $\inflog[s]$}
\begin{enumerate}[1.]
%
\item \header{Note on Pebble Games $\game[s]{m}(\strct{A}, \vect{a}, \strct{B}, \vect{b})$} They are different from the usual Ehrenfeucht-Fra\"iss\'e games $\game{m}(\strct{A}, \vect{a}, \strct{B}, \vect{b})$ in that the number of pebbles are fixed ($s$ for each of $\strct{A}$ and $\strct{B}$ here) and the pebbles are on or off (denoted by $\ast$) the board; once a pebble is placed onto the board, it is never removed off of it. Thus, pebble games consist of putting the pebbles onto the board (extensions) and moving them around on the board (moves).
%
\item \header{Note on \thesection.6}
\begin{enumerate}[(a)]
%%
\item To be clear, the last statement should be ``for arbitrary \emph{$\emptyvoc$-structures} $\strct{A}$ and $\strct{B}$, the duplicator wins $\game[s]{\infty}(\strct{A}, \strct{B})$ iff he wins $\game[s]{s}(\strct{A}, \strct{B})$.'' This is not true for arbitrary $\tau$, however, as can be seen in the example given in the next part.
%%
\item The spoiler wins $\game[3]{\infty}(\strct{G}_l, \strct{G}_l \dunion \strct{G}_l)$ but does not win $\game[3]{3}(\strct{G}_l, \strct{G}_l \dunion \strct{G}_l)$. As remarked in 2.3.16, the information that exactly one of the structures $\strct{G}_l$ and $\strct{G}_l \dunion \strct{G}_l$ satisfy the $\inflog[3]$-sentence expressing connectivity can be transformed into a winning strategy for the spoiler.
%%
\end{enumerate}

%
\item \header{Hint to Exercise \thesection.7}
\begin{enumerate}[(a)]
%%
\item For the `if' direction, assume that $A = \sete{a_0, \etc, a_k}$, where
\[
a_0 \mathrel{\intpr{<}{A}} \etc \mathrel{\intpr{<}{A}} a_k.
\]
Then take the conjunction of the following $\folog[2]$-sentences that together describe $\strct{A}$ up to isomorphism (where $\psi'_n$ is as in Examples \thesection.1).
\begin{itemize}
%%%
\item The size of $A$ ($\card{A} = k + 1$):
\[
\exists x \psi'_{k - 1} \land \forall x \neg\psi'_k.
\]
%%%
\item The interpretation of $c \in \tau$ in $\strct{A}$ ($\intpr{c}{\strct{A}} = a_i$):
\[
\exists x (x = c \land \psi'_i).
\]
%%%
\item The elements of the interpretation $\intpr{P}{\strct{A}}$ of unary $P \in \tau$ in $\strct{A}$:
\[
\forall x (Px \liff \blor \setm{\psi'_i}{a_i \in \intpr{P}{\strct{A}}}).
\]
%%%
\item The elements of the interpretation $\intpr{R}{\strct{A}}$ of binary $R \in \tau$ in $\strct{A}$:
\[
\forall x \forall y (Rxy \liff \blor \setm{\psi'_i \land \psi'_j\begin{perm}{c} yx \cr xy \end{perm}}{(a_i, a_j) \in \intpr{R}{\strct{A}}}).
\]
%%%
\end{itemize}
%%
\item For the `if' direction, note that the partial isomorphisms of size $s$ from $\strct{A}$ to $\strct{B}$ resulted from all possible rounds of the game $\game[s]{m}(\strct{A}, \strct{B})$ with the additional condition after $s$ moves provide a winning strategy for the duplicator in the same game without that condition.
%%
\item Suppose that $\tau$ is relational and all its relation symbols have arity $\leq s$, where $s \geq 1$ by definition (cf.\ Part A in Chapter 1). Moreover, suppose that $\card{A} = \card{B} \leq s + 1$ and that the duplicator wins $\game[s]{\infty}(\strct{A}, \strct{B})$. Our goal then is to show that $\strct{A} \isom \strct{B}$. We further assume $\tau \neq \emptyvoc$, since otherwise it is trivial that the duplicator wins $\game[s]{\infty}(\strct{A}, \strct{B})$ and $\strct{A} \isom \strct{B}$.
\medskip\\
First observe the assumption that the duplicator wins $\game[s]{\infty}(\strct{A}, \strct{B})$ immediately implies that he wins $\game[s]{m}(\strct{A}, \strct{B})$ for $m \geq 1$. By Theorem \thesection.5, then, we have for every $m \geq 1$, $\strct{A} \equv^s_m \strct{B}$. Therefore, it suffices to find a sentence $\varphi_\strct{A} \in \folog[s]\of{\tau}$ that characterizes $\strct{A}$ up to isomorphism given the fixed cardinality $\card{A}$, i.e.
\begin{center}
for every structure $\strct{A}'$ with $\card{A'} = \card{A}$, \ $\strct{A}' \satis \varphi_\strct{A}$ \ iff \ $\strct{A}' \isom \strct{A}$.
\end{center}
Then we are done: The first-order sentence $\varphi_\strct{A}$ has a (finite) quantifier rank, say $m$, and it follows that $\strct{A} \isom \strct{B}$ because $\strct{B} \satis \varphi_\strct{A}$ (implied by $\strct{A} \equv^s_m \strct{B}$ above).
\medskip\\
Next, the sentence $\varphi_\strct{A}$ depends on whether $\card{A} \leq s$ or $\card{A} = s + 1$. Recall the set of atomic formulas
\[
\Theta_n \defas \sett{\psi}{\mathmode{\psi} has the form \mathmode{Rx_1 \etc x_k}, \mathmode{x = y} and variables among \mathmode{v_1, \etc, v_n}}
\]
used in the proof of 2.1.1. (There are no formulas of the form $c = x$ in $\Theta_n$ as $\tau$ is assumed to be relational.)
\medskip\\
The case $\card{A} \leq s$ is simple: If $\card{A} = n$, i.e.\ $A = \sete{a_1, \etc, a_n}$ then we choose
\[
\varphi_\strct{A} \defas \exists v_1 \etc \exists v_n (\bland\setm{\psi}{\psi \in \Theta_n, \strct{A} \satis \psi[\vect{a}]} \land \bland\setm{\neg\psi}{\psi \in \Theta_n, \strct{A} \satis \neg\psi[\vect{a}]})
\]
where $\vect{a}$ denotes $a_1 \etc a_n$.
\medskip\\
It remains to deal with the case $\card{A} = s + 1$, i.e.\ $A = \sete{a_1, \etc, a_{s + 1}}$. Remember that we only have $s$ distinct variables $v_1, \etc, v_s$ to describe $\strct{A}$. The idea is that, using these $s$ (quantified) variables, we describe the relations and (in)equalities
\begin{enumerate}[(1)]
%%
\item among $a_1, \etc, a_s$ first, where $v_1, \etc, v_s$ are intended to represent $a_1, \etc, a_s$ respectively, and then
%%
\item among $a_1, \etc, a_{i - 1}, a_{i + 1}, \etc, a_{s + 1}$ for each $1 \leq i \leq s$, where $v_1, \etc, v_{i - 1}, v_i, v_{i + 1}, \etc, v_s$ are intended to represent $a_1, \etc, a_{i - 1}, a_{s + 1}, a_{i + 1}, \etc, a_s$ ($a_{s + 1}$ is between $a_{i - 1}$ and $a_{i + 1}$ in this order), respectively.
%%
\end{enumerate}
Note that this is feasible as every relation symbol has arity at most $s$.
\medskip\\
For brevity, let us denote $a_1, \etc, a_{i - 1}, a_{s + 1}, a_{i + 1}, \etc, a_s$ ($a_{s + 1}$ is between $a_{i - 1}$ and $a_{i + 1}$ in this order) by $\vect{a}_i$ for $1 \leq i \leq s$; moreover, we also let $\vect{a}_0$ denote $a_1, \etc, a_s$. The desired sentence is thus
\[
\varphi_\strct{A} \defas \exists v_1 \etc \exists v_s \bland^s_{i = 0} \varphi_i,
\]
where
\[
\varphi_0 \defas \bland \setm{\psi}{\psi \in \Theta_s, \strct{A} \satis \psi[\vect{a}_0]} \land \bland \setm{\neg\psi}{\psi \in \Theta_s, \strct{A} \satis \neg\psi[\vect{a}_0]}
\]
and for $1 \leq i \leq s$,
\[
\varphi_i \defas
\begin{cases}
\forall v_i (\bland_{1 \leq j \leq s, j \neq i} \neg v_i = v_j \lthen \varphi_0) & \text{if \mathmode{\strct{A} \satis \varphi_0[\vect{a}_i]}}\cr
\exists v_i (\bland \setm{\psi}{\psi \in \Theta_s, \strct{A} \satis \psi[\vect{a}_i]} \land \bland \setm{\neg\psi}{\psi \in \Theta_s, \strct{A} \satis \neg\psi[\vect{a}_i]}) & \text{otherwise}.\cr
\end{cases}
\]
%%
\end{enumerate}
%
\item \header{Note on the $s$-$m$-Isomorphism Type $\ityp[s]{m}{\strct{A}, \vect{a}}$} Unlike $\game{m}(\strct{A}, \vect{a}, \strct{B}, \vect{b})$, the game $\game[s]{m}(\strct{A}, \vect{a}, \strct{B}, \vect{b})$ consists of moves that \emph{substitute} pairs of elements in partial isomorphisms rather than \emph{extend} partial isomorphisms with pairs of elements. This is reflected in the definition of $\ityp{m + 1}{\strct{A}, \vect{a}}$:
\begin{enumerate}[(1)]
%%
\item The running conjunction $\bland\limits_{1 \leq i \leq s}$ specifies the valid positions for substitution, and the formula $\ityp{m}{\strct{A}, \vect{a}\sbst{a}{i}}$ specifies that $\vect{a}\sbst{a}{i} \mapsto \vect{b}\sbst{b}{i}$ is an $s$-partial isomorphism whose constituent pairs can be substituted $m$ times (given the premise $\strct{B} \satis \ityp{m}{\strct{A}, \vect{a}\sbst{a}{i}}[\vect{b}\sbst{b}{i}]$).
%%
\item The conjunct $\ityp{0}{\strct{A}, \vect{a}}$ is necessary because it describes the relation among the elements in $\vect{a}$ so that if $\strct{B} \satis \ityp{0}{\strct{A}, \vect{a}}[\vect{b}]$ then $\vect{a} \mapsto \vect{b}$ is an $s$-partial isomorphism from $\strct{A}$ to $\strct{B}$ (and vice versa), whereas the other conjunct $\bland\limits_{1 \leq i \leq s}(\bland\limits_{a \in A} \exists v_i \ityp{m}{\strct{A}, \vect{a}\sbst{a}{i}} \land \forall v_i \blor\limits_{a \in A} \ityp{m}{\strct{A}, \vect{a}\sbst{a}{i}})$ does not have this effect. In fact, we have
\[
\consq \hint{m + 1}{\strct{A}, \vect{a}} \lthen \hint{0}{\strct{A}, \vect{a}}
\]
(cf.\ 2.2.4(c)) but
\begin{center}
not $\consq \bland\limits_{1 \leq i \leq s}(\bland\limits_{a \in A} \exists v_i \ityp{m}{\strct{A}, \vect{a}\sbst{a}{i}} \land \forall v_i \blor\limits_{a \in A} \ityp{m}{\strct{A}, \vect{a}\sbst{a}{i}}) \lthen \ityp{0}{\strct{A}, \vect{a}}$
\end{center}
(for $\vect{a} \mapsto \vect{b}$ there may be an $a' \neq a_i$ or a $b' \neq b_i$ such that $\vect{a}\sbst{a'}{i} \mapsto \vect{b}$ or $\vect{a} \mapsto \vect{b}\sbst{b'}{i}$ is an $s$-partial isomorphism).

Note that the conjunct $\ityp{0}{\strct{A}, \vect{a}}$ is redundant and is implied by the other conjunct only when $\vect{a} = \ast \etc \ast$.
%%
\end{enumerate}
%
\item \header{Note on Part (a) of 3.3.9 and 3.3.10} It gives that
\begin{center}
\begin{tabular}{ll}
\   & $(\strct{A}, \vect{a}) \equv^s (\strct{B}, \vect{b})$ \cr
iff & for every $m \in \nat$, $(\strct{A}, \vect{a}) \equv^s_m (\strct{B}, \vect{b})$ \cr
iff & for every $m \in \nat$, $\strct{B} \satis \ityp[s]{m}{\strct{A}, \vect{a}}[\vect{b}]$ \cr
iff & $\strct{B} \satis (\bland\limits_{m \in \nat} \ityp[s]{m}{\strct{A}, \vect{a}})[\vect{b}]$. \cr
\end{tabular}
\end{center}
%
\item \header{Note on 3.3.11} As in 3.2.11, the underlying vocabulary $\tau$ is relational and the results remain valid when $\tau$ contains only unary relation symbols.

The first statement that \emph{every model of $\conjextaxm{s}$ has at least $s$ elements} can be proved by induction on $s$, using the fact that $\conjextaxm{s + 1}$ is the conjunction of all $(s + 1)$-extension axioms with $\conjextaxm{s}$.

To prove the second statement that \emph{every two models $\strct{A}$ and $\strct{B}$ of $\epsilon_s$ are $s$-partially isomorphic}, consider the set
\[
I \defas \sett{\vect{a} \mapsto \vect{b}}{\mathmode{\vect{a} \in \cartpwr{(A \union \sete{\ast})}{s}}, \mathmode{\vect{b} \in \cartpwr{(B \union \sete{\ast})}{s}}, and \mathmode{\ityp{0}{\strct{A}, \vect{a}} = \ityp{0}{\strct{B}, \vect{b}}}}
\]
and show that it has the back and forth properties. In proving the forth property, for example, suppose that $\vect{a} \mapsto \vect{b}$ is an $s$-partial isomorphism from $\strct{A}$ to $\strct{B}$ where $\vect{a}$ and $\vect{b}$ can be assumed to consist of distinct elements and $a$ is not an entry of $\vect{a}$, to show there is a $b$ (that is different from all entries in $\vect{b}$) such that, say, $\vect{a}\sbst{a}{i} \mapsto \vect{b}\sbst{b}{i}$ is an $s$-partial isomorphism, there are two cases to consider:
\begin{enumerate}[(1)]
%%
\item If $a_i = \ast$ (\emph{extension} case), then we argue as for usual partial isomorphisms.
%%
\item If $a_i \neq \ast$ (\emph{substitution} case), then observe that $\vect{a}' \mapsto \vect{b}'$ is also an $s$-partial isomorphism where $\vect{a}'$ and $\vect{b}'$ are $\vect{a}$ and $\vect{b}$ with $a_i$ and $b_i$ removed, respectively. We then reduce this to the above extension case.
%%
\end{enumerate}
Since $\tau$ is relational, we have $\ast\etc\ast \mapsto \ast\etc\ast \in I$ and hence $I \neq \emptyset$.

Finally, observe for $s \geq 1$, we have $\randstrtheory \consq \conjextaxm{s}$, thus $\conjextaxm{s}$ is satisfiable (since $\randstrtheory$ has a model). With the third statement in the book, it also implies that for any $\inflog[\omega]$-sentence $\varphi$, $\randstrtheory \consq \varphi$ or $\randstrtheory \consq \neg\varphi$.
%
\item \header{Note on 3.3.13} See Note on Example 2.3.5 on how to extend the result to arbitrary $\tau$.
%
\item \header{Hint to 3.3.14} In part (a), note that for $s, m \in \nat$, the set
\[
\sett{\ityp{m}{\strct{A}, \vect{a}}}{\mathmode{\strct{A}} a structure and \mathmode{\vect{a} \in \cartpwr{A}{s}}}
\]
is finite.
%
\end{enumerate}
%end of section 3-----------------------------------------------------------------------------