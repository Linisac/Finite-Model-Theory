%\setcounter{chapter}{1}
\chapter{The Ehrenfeucht-Fra\"{i}ss\'{e} Method}
\setcounter{section}{1}
%section 2------------------------------------------------------------------------------------
\section{Ehrenfeucht's Theorem}
\begin{enumerate}[1.]
%
\item \header{Note on Definition 2.2.1} The condition for a map $p$ to be a partial isomorphism between $\strct{A}$ and $\strct{B}$ is slightly different in \cite{EFT}: the designated constants $\intpr{c}{A}$ are not required to be in $\dom{p}$ but if they are, the condition $p(\intpr{c}{A}) = \intpr{c}{B}$ must be satisfied.
%
\item \header{Note on Remarks 2.2.2} For part (a), note that the empty map $p = \emptyset$ is \emph{not} a partial isomorphism if $\tau$ does contain constants, by Definition 2.2.1.
\medskip\\
For part (c), the notation $\vect{a} \mapsto \vect{b}$ can be seen as a shorthand for the set
\[
\setm{(a_i, b_i)}{i \leq s} \union \setm{(\intpr{c}{\strct{A}}, \intpr{c}{\strct{B}})}{c \in \tau},\]
which may not define a map, nor a partial isomorphism. If the above set does define a map, then it is the graph of this map. Statement (i) says $\vect{a} \mapsto \vect{b}$ not only defines a map but even a partial isomorphism between $\strct{A}$ and $\strct{B}$.
\medskip\\
Also note that if $\tau$ contains constants then the notation $\vect{a} \mapsto \vect{b}$ contains mapping of the constants, which are \emph{hidden} from it; in this case, $\emptyseq \mapsto \emptyseq$ does not equal the empty map $\emptymap$. $\emptyseq \mapsto \emptyseq = \emptymap$ when $\tau$ contains no constants.
%
\item \header{Note on Ehrenfeucht Games.} It is clear from definition that at most one of the players has a winning strategy for the game $\game{m}{\strct{A}, \vect{a}, \strct{B}, \vect{b}}$.
\medskip\\
Moreover, it is also true that at least one of the players has a winning strategy for the game $\game{m}{\strct{A}, \vect{a}, \strct{B}, \vect{b}}$, i.e.\ \emph{if the duplicator does not win $\game{m}{\strct{A}, \vect{a}, \strct{B}, \vect{b}}$ then the spoiler wins it.} See 2.3.16 in text.
%
\item \header{Note on the Proof of Lemma 2.2.3(b)} The statement ``Then $p : \vect{a} \mapsto \vect{b} \in \partisoms{\strct{A}}{\strct{B}}$ with $\dom{p} = A$'' follows from the fact that a submap of a partial isomorphism is also a partial isomorphism (see the next note).
%
\item \header{Note on Lemma 2.2.4} Part (c) is equivalent to say that a submap (i.e.\ a map that is a subset of another map in terms of their graphs) of a partial isomorphism is also a partial isomorphism.
\medskip\\
On the other hand, it is useful to use the alternative logically equivalent forms (such as contraposition) of parts (b) or (c) in deriving properties of the Ehrenfeucht games.
%
\item \header{Note on Lemma 2.2.6 and the Remark below It} It is better to prove the lemma and the remark that ``the conjunctions and disjunctions in the definition of $\ityp{m}{\strct{A}, \vect{a}}$ are finite'' all at once:
\medskip\\
\emph{For $s, m \geq 0$, $\ityp{m}{\strct{A}, \vect{a}}$ is a well-formed formula - i.e.\ the conjunctions and disjunctions (if any) in it are finite - for any structure $\strct{A}$ and $\vect{a} \in \cartpwr{A}{s}$ and the set $\sett{\ityp{m}{\strct{A}, \vect{a}}}{\begin{math}\strct{A}\end{math} a structure and \begin{math}\vect{a} \in \cartpwr{A}{s}\end{math}}$ is finite.}
\medskip\\
The induction is performed on $m$. The base case follows from that for $s > 0$, the set $\sett{\varphi(\seq{v}{s})}{\begin{math}\varphi\end{math} atomic or negated atomic}$ is finite.
%
\item \header{Note on the Proof of Theorem 2.2.8 (Ehrenfeucht's Theorem)} To prove (i) implies (iii), the case $m = 0$ can be handled by applying 2.2.4(a) in addition to the equivalence between (i) and (ii) of 2.2.2(c).
\medskip\\
The induction part of (i) implying (iii) in fact proves $\strct{A} \satis \varphi[\vect{a}]$ iff $\strct{B} \satis \varphi[\vect{b}]$.
%
\item \header{Note on Theorem 2.2.11} There is typo in the statement of this theorem: ``$\vect{a} \in \strct{A}$'' should be replaced by ``$\vect{a} \in \cartpwr{A}{s}$''.
%
\end{enumerate}
%end of section 2-----------------------------------------------------------------------------


%section 3------------------------------------------------------------------------------------
\section{Examples and Fra\"{i}ss\'{e}'s Theorem}
\begin{enumerate}[1.]
%
\item \header{Hint to Exercise 2.3.2} Note that $I_j \subseteq \tilde{I}_j$ and that if $p \in \partisoms{\strct{A}}{\strct{B}}$ and if $q \subseteq p$ then $q \in \partisoms{\strct{A}}{\strct{B}}$.
\begin{remark}
There is a typo: ``$\emptyseq \mapsto \emptyseq \in I_j$'' should be replaced by ``$\emptyset \mapsto \emptyset \in \tilde{I}_j$''.
\end{remark}
%
\item \header{Note on Corollary 2.3.4} For $s = 0$ the statement of 2.3.3(iii) becomes
\begin{quoteno}{$\ast$}
There is $\seqi{I_j}{j \leq m}$ with $\emptyseq \mapsto \emptyseq \in I_m$ such that $\seqi{I_j}{j \leq m} : \strct{A} \isom[m] \strct{B}$.
\end{quoteno}
\medskip\\
It is equivalent to 2.3.4(iii): The latter obviously follows the former; to derive the former from the latter, note that by 2.3.2 if $\seqi{I_j}{j \leq m} : \strct{A} \isom[m] \strct{B}$ then $\seqi{\tilde{I}_j}{j \leq m} : \strct{A} \isom[m] \strct{B}$ and $\emptyseq \mapsto \emptyseq \in \tilde{I}_m$.
%
\item \header{Note on Example 2.3.5} Let $\tau$ be an arbitrary symbol set that consists of relation symbols $\seq{P}{n}$ and constants $\seq{c}{k}$ where $n, k \in \nat$.
\medskip\\
For any $m \in \nat$, consider the two $\tau$-structures $\strct{A}$ and $\strct{B}$ where $A$ consists of elements $\seq{a}{m + 1}$ other than $\seq{\intpr{c}{\strct{A}}}{n}$, $\intpr{c_i}{\strct{A}} \neq \intpr{c_j}{\strct{A}}$ if $i \neq j$, $\intpr{P_i}{\strct{A}} = \emptyset$, $B$ consists of elements $\seq{b}{m + 2}$ other than $\seq{\intpr{c}{\strct{B}}}{n}$, $\intpr{c_i}{\strct{B}} \neq \intpr{c_j}{\strct{B}}$ if $i \neq j$, $\intpr{P_i}{\strct{B}} = \emptyset$.
\medskip\\
Obviously, exactly one between $\strct{A}$ and $\strct{B}$ is a member of $\even{\tau}$. However, it is also true that $\strct{A} \equv[m] \strct{B}$: the map $\enum{a}{m + 1} \mapsto \enum{b}{m + 1}$ can be used as a winning strategy (note that implicitly $\intpr{c_i}{\strct{A}}$ is mapped to $\intpr{c_i}{\strct{B}}$, cf.\ 2.2.2(c)(i)) for the duplicator in the game $\game{m}{\strct{A}, \strct{B}}$. Thus $\even{\tau}$ is not axiomatizable by 2.2.12.
%
\item \header{Note on Example 2.3.8} There is a typo: ``$\dist[j](a, a^\prime)$'' appearing in the definition of distance function should be replaced by ``$\dist[j](a, b)$''.
%
\item \header{Hint to Exercise 2.3.9} For simplicity, take the structure $\strct{C}_l$ that is isomorphic to $\strct{B}_l \dsjuni \strct{D}_l$, where $C_l \defas \sete{0, \ldots, 2l + 1}$ and the substructures $\substr{\sete{0, \ldots, l}}{\strct{C}_l}$ induced by $\sete{0, \ldots, l}$ and $\substr{\sete{l + 1, \ldots, 2l + 1}}{\strct{C}_l}$ induced by $\sete{l + 1, \ldots, 2l + 1}$ are isomorphic to $\strct{B}_l$ and $\strct{D}_l$, respectively.
\medskip\\
For $\strct{B}_l$ and $\strct{C}_l$ define the distance function $d$ on $B_l \cart B_l$ and on $C_l \cart C_l$ as
\[
\dist(h, k) \defas
\begin{cases}
\mbox{length of the shortest path from \begin{math}h\end{math} to \begin{math}k\end{math}} & \mbox{if there is one} \cr
\infty & \mbox{else},
\end{cases}
\]
and take the truncated version $\dist[j]$ where
\[
\dist[j](h, k) \defas
\begin{cases}
\dist(h, k) & \mbox{if \begin{math}\dist(h, k) < 2^j\end{math}} \cr
\infty & \mbox{else}.
\end{cases}
\]
For $m \geq 0$ choose $l \geq 2^m$. Consider $\seqi{I_j}{j \leq m}$ where $p \in I_j$ if and only if $p$ is a partial isomorphism between $\strct{B}_l$ and $\strct{C}_l$ such that $\card{p} \leq m - j + 2$, $p(0) = 0$, $p(l) = l$ and $\dist[j](h, k) = \dist[j](p(h), p(k))$ for $h, k \in \dom{p}$.
\medskip\\
It remains to verify $\seqi{I_j}{j \leq m} : \strct{B}_l \isom[m] \strct{C}_l$, which is omitted here. (For the forth-property, if $p \in I_{j + 1}$ and $b \in B_l$ then distinguish two cases according to whether it is true that ``there is a $b^\prime \in B_l$ such that $\dist[j](b, b^\prime) < 2^j$ or $\dist[j](b^\prime, b) < 2^j$'', a technique used in 2.3.6.)
%
\item \header{Note on Corollary 2.3.11} Here $(\strct{A}, \vect{a}) \equv[m] (\strct{B}, \vect{b})$ means ``$\vect{a}$ satisfies in $\strct{A}$ the same formulas of quantifier rank $\leq m$ as $\vect{b}$ in $\strct{B}$.'' (Consider the transition from 2.2.8(iii) to 2.2.9(iii).)
%
\item \header{Note on Corollary 2.3.11} Is it true that
\begin{quote}
``\emph{If $(\strct{A}_1, \vect{a}_1) \equv[m] (\strct{B}_1, \vect{b}_1)$ and $(\strct{A}_2, \vect{a}_2) \equv[m] (\strct{B}_2, \vect{b}_2)$ then $(\strct{A}_1 \cart \strct{A}_2, \vect{a}_1 \cart \vect{a}_2) \equv[m] (\strct{B}_1 \cart \strct{B}_2, \vect{b}_1 \cart \vect{b}_2)$}''?
\end{quote}
My guess is yes. Note that if $\vect{a}_1 \isom \vect{b}_1$ and $\vect{a}_2 \isom \vect{b}_2$ then $\vect{a}_1 \cart \vect{a}_2 \isom \vect{b}_1 \cart \vect{b}_2$.
%
\item \header{Hint to Exercise 2.3.12} An equivalent condition to ``$\min\sete{\card{A_\alpha}, m} = \min\sete{\card{B_\alpha}, m}$'' is:
\begin{quote}
$\card{B_\alpha} = \card{A_\alpha}$ if $\card{A_\alpha} < m$, and $\card{B_\alpha} \geq m$ otherwise.
\end{quote}
In addition, an alternative statement to the condition ``$\card{B_\alpha} \geq m$'' is:
\begin{quote}
For $0 \leq j < m$, $\card{B_\alpha} \neq j$.
\end{quote}
And for this exercise it is appropriate to define $\existexactly{0}x Rx \defas \neg\exists x Rx$.
\begin{remark}
A similar exercise is XII.3.17 in \cite{EFT}.
\end{remark}
%
\item \header{Hint to Exercise 2.3.13} (INCOMPLETE)
Think of $\ordsum^n \strct{A}$ as an $n$-element linear ordering in which every point ``expands'' to $\strct{A}$.\\
\medskip\\
More precisely, define for $a, a^\prime$ in the domain of $\ordsum^n \strct{A}$ the distance function
\[
\dist(a, a^\prime) \defas \abs{i - j},
\]
where $a$ is an element from the $i$th copy of $\strct{A}$ and $a^\prime$ from the $j$th. The truncated versions of distance function are defined analogously. A winning strategy for the duplicator is the same as that in 2.3.6 except that if the spoiler chooses an element from a copy $\strct{A}$ of $\ordsum^l \strct{A}$ (or $\ordsum^k \strct{A}$) then the duplicator chooses exactly the same element from the corresponding copy $\strct{A}$ of $\ordsum^k \strct{A}$ (or $\ordsum^l \strct{A}$, respectively).
%
\item \header{Hint to Exercise 2.3.14} (INCOMPLETE)
The notation ``$(\strct{A}, \vect{a}) \isom[m] (\strct{B}, \vect{b})$'' is undefined in text, however it can be understood as:
\begin{quote}
There is $\seqi{I_j}{j \leq m}$ with $\vect{a} \mapsto \vect{b} \in I_m$ such that $\seqi{I_j}{j \leq m} : \strct{A} \isom[m] \strct{B}$.
\end{quote}
This is statement (iii) of 2.3.3. Therefore this exercise is an immediate consequence of 2.3.3.
\begin{remark}
The premise ``for $\vect{a} \mapsto \vect{b} \in \partisoms{\strct{A}}{\strct{B}}$'' is implied by the statements on both sides of ``iff'', so it can be weakened to ``for $\vect{a} \in A, \vect{b} \in B$''.
\end{remark}
%
\item \header{Note on Remark 2.3.16} As an alternative proof for the last part of 2.2.8, one can show, contrapositively, that \emph{not \refitem{(iii)} implies not \refitem{(i)}}. To be more precise, in the inductive step let $m > 0, \qr{\varphi} = m$, and suppose that $\varphi(\vect{x}) = \exists y \psi$ where $\strct{A} \satis \varphi[\vect{a}]$ but not $\strct{B} \satis \varphi[\vect{b}]$. It follows that there is an $a \in A$ such that for all $b \in B$, $\strct{A} \satis \psi[\vect{a}a]$ but not $\strct{B} \satis \psi[\vect{b}b]$, hence by induction hypothesis the duplicator does not win the game $\game{m - 1}{\strct{A}, \vect{a}a, \strct{B}, \vect{b}b}$. By 2.2.4(b), therefore, we have that the duplicator does not win the game $\game{m}{\strct{A}, \vect{a}, \strct{B}, \vect{b}}$. The other case where $\strct{B} \satis \varphi[\vect{b}]$ but not $\strct{A} \satis \varphi[\vect{a}]$ can be done symmetrically.
\medskip\\
The above argument suggests (the essential part of) a winning strategy of the game $\game{m}{\strct{A}, \vect{a}, \strct{B}, \vect{b}}$ for the spoiler when (iii) in 2.2.8 is false: If a formula $\varphi = \exists y \psi$ has quantifier rank $m > 0$ and if, say, $\strct{A} \satis \varphi[\vect{a}]$, then he picks an element $a \in A$ such that $\strct{A} \satis \psi[\vect{a}a]$; if $\varphi = \forall y \chi$ has quantifier rank $m > 0$ and if, say, not $\strct{B} \satis \varphi[\vect{b}]$ (namely if $\strct{B} \satis \exists y \neg\chi[\vect{b}]$), then he picks an element $b \in B$ such that $\strct{B} \satis \neg\chi[\vect{b}b]$. The spoiler then makes his successive choices accordingly in his turns in every play of the game.
%
\end{enumerate}
%end of section 3-----------------------------------------------------------------------------


%section 4------------------------------------------------------------------------------------
\section{Hanf's Theorem}
\begin{enumerate}[1.]
%
\item \header{Note on the First Paragraph on Page 27} Here the \emph{isomorphism type} of $(\ballstrct{r, a}, a)$ might refer to the equivalent class of $(\ballstrct{r, a}, a)$ induced by the isomorphism relation, i.e.\ the set of structures that are isomorphic to $(\ballstrct{r, a}, a)$, where $(\ballstrct{r, a}, a)$ might refer to the expansion of the structure $\ballstrct{r, a}$ in which $a$ is a distinguished constant.
%
\item \header{Note on the Proof of 2.4.1} Here $\length{\vect{a}}$ refers to the length of the tuple $\vect{a}$, see the first paragraph on page 6.
\medskip\\
On the other hand, by the proof the requirement on the cardinality of $3^m$-balls may be weakened to ``at most $e$ elements.''
%
\item \header{Note on 2.4.2 and 2.4.3} For the proof of 2.4.3 to be valid, in the structure $(D_l, E^\prime_l, P_1, \etl, P_r)$ there must be a point on each of the paths from $a$ to $b_-$ and from $b$ to $a_-$ that is in neither of the $3^m$-balls of $a$ and $b$, in other words, both cycles in the structure must have length greater than $2 \mul 3^m + 1$; otherwise the $3^m$-ball type of $a$ (or $b$) would be different from that of $a$ (or $b$, respectively) in $(\strct{D}_l, P_1, \etl, P_r)$ - the former $3^m$-ball is a cycle, whereas the latter is not.
\medskip\\
In fact, 2.4.3 can be strengthened to allow such points.
%
\item \header{Note on 2.4.4} By definition, the Gaifman graph $\gaifman{\strct{A}}$ of a digraph $\strct{A}$ is the associated (undirected) graph of $\strct{A}$.
%
\item \header{Note on 2.4.5} By the same argument in the proof, it follows that \emph{the class of finite graphs that are not connected cannot be axiomatized by a formula of the form $\exists P_1 \etl \exists P_r \psi$, either.}
\medskip\\
As an immediate consequence, we have that \emph{both classes cannot be axiomatized by a formula of the form $\forall P_1 \etl \forall P_r \chi$.}
%
\item \header{Note on 2.4.6} In the proof there is a (possible) typo: $\intpr{R}{A}$ in $(\strct{G}, \intpr{R}{A})$ should be replaced by $\intpr{R}{G}$ or $\intpr{R}{\strct{G}}$.
\medskip\\
On the other hand, this proposition implies that \emph{the class of finite graphs that are not connected can be axiomatized by a formula of the form $\forall R \chi$.} (Just take the negation of $\exists R \psi$.)
%
\item \header{Brief Solution to Exercise 2.4.7} Using a similar method (basically the pigeonhole principle) we can obtain the corresponding result to 2.4.2, in which the distance $\dist(a, b)$ is greater than $2 \mul 3^m + 1$ (cf.\ \header{Note on 2.4.2 and 2.4.3}). Now let $a_-$ and $b_-$ be the two points such that $(a_-, a)$ and $(b_-, b)$ are edges in $\strct{H}_l$. Obtain the structure $\strct{H}_l^\prime$ from $\strct{H}_l$ by removing these two edges and adding $(b_-, a)$ and $(a_-, b)$. Then likewise we obtain the corresponding result to 2.4.3, and hence that to 2.4.5.
\medskip\\
As for the corresponding result to 2.4.6, note that a digraph is cyclic if and only if there is a linear ordering over a set of at least two points in which there is an edge from $x$ to $y$ if and only if $y$ is immediately greater than $x$ or $x$ is the greatest element and $y$ the least in the linear ordering. Finally, formulate the condition and take the negation.
\begin{note}
It does not seem possible to axiomatize this class by a formula of the form $\forall P \psi$ where $P$ is unary and $\psi$ first-order (by the corresponding result to 2.4.5 mentioned above and the discussion in \header{Note on 2.4.5}) or the form $\exists R \psi$ where $R$ is binary and $\psi$ first-order as required (this is conjectured, however).
\end{note}
%
\item \header{Brief Solution to Exercise 2.4.8} An informal formulation for $\psi(x, y)$ is already present in the description of the exercise.
\medskip\\
Also observe that the formula $\forall x \forall y \exists P \varphi$ axiomatizes the class of finite and connected graphs, and hence by 2.4.5 is not equivalent to a sentence of the form $\exists P_1 \etl \exists P_r \chi$.
%
\item \header{Note on the Definition of Basic Local Sentences} In the definition given in textbook, the condition $\rltv{\psi}{\ball{r, x_n}}$ is missing; therefore, it should be replaced by
\[
\exists x_1 \etl \exists x_n (\bland_{1 \leq i < j \leq n}\dist(x_i, x_j) > 2r \land \bland_{1 \leq i \leq n} \rltv{\psi}{\ball{r, x_i}}(x_i)),
\]
where $n \geq 1$. (The sentence that precedes 2.5.1 thus should be modified accordingly.)
%
\end{enumerate}
%end of section 4-----------------------------------------------------------------------------


%section 5------------------------------------------------------------------------------------
\section{Gaifman's Theorem}
\begin{enumerate}[1.]
%
\item \header{Note on the Proof of 2.5.2} In case 1, note that, as in the proof of 2.4.1, $\ball{7^j, \vect{a}a} \subseteq \ball{7^{j + 1}, \vect{a}}$, so $\ballstrct[{\ballstrct[\strct{A}]{7^{j + 1}, \vect{a}}}]{7^j, \vect{a}a} = \ballstrct[\strct{A}]{7^j, \vect{a}a}$.
\medskip\\
In case 2, it seems sufficient to have $g(j + 1)$ no smaller than the quantifier rank of sentences in (2) and (3). Moreover, we have $\strct{A} \satis \psi^j_a(a)$ and hence $\strct{A} \satis \exists x_1 \delta_1(x_1)$ (here $\exists x_1 \delta_1(x_1) = \exists x_1 \psi^j_a(x_1)$), so $i \geq 1$. Also note the following:
\begin{enumerate}[(a)]
%%
\item In case 2.1, we have $e \geq 1$ (recall that $i \geq 1$) so the argument on the upper bound of the distance of any element (in $A$) from $\vect{a}$ satisfying $\psi^j_a$ is valid. There is a typo: ``$a \not\in \ball{2 \mul 7^{j + 1}, \vect{a}}$'' should be replaced with ``$a \not\in \ball{2 \mul 7^j, \vect{a}}$''. The condition $\dist(\vect{a}, a) \leq 6 \mul 7^j$ implies that $\ball{7^j, a} \subseteq \ball{7^{j + 1}, \vect{a}}$ and hence $\ballstrct[\ballstrct{7^{j + 1}, \vect{a}}]{7^j, a} = \ballstrct[\strct{A}]{7^j, a}$, which -- together with the assumption that $a \not\in \ball{2 \mul 7^j, \vect{a}}$ and the fact that $\ballstrct[\ballstrct{7^{j + 1}, \vect{a}}]{7^j, \vect{a}} = \ballstrct[\strct{A}]{7^j, \vect{a}}$ (since $\ball{7^j, \vect{a}} \subseteq \ball{7^{j + 1}, \vect{a}}$) -- gives
\[
\ballstrct{7^{j + 1}, \vect{a}} \satis \exists z (2 \mul 7^j < \dist(\vect{a}, z) \leq 6 \mul 7^j \land \psi^j_a(z) \land \psi^j_{\vect{a}}(\vect{a})).
\]
Also, the condition
\[
\ballstrct{7^{j + 1}, \vect{b}} \satis \exists z (2 \mul 7^j < \dist(\vect{b}, z) \leq 6 \mul 7^j \land \psi^j_a(z) \land \psi^j_{\vect{a}}(\vect{b}))
\]
guarantees a $b \in B$ such that $\ball{7^j, \vect{b}} \intsc \ball{7^j, b} = \emptyset$ and $\ball{7^j, b} \subseteq \ball{7^{j + 1}, \vect{b}}$ and hence $\ballstrct[\ballstrct{7^{j + 1}, \vect{b}}]{7^j, b} = \ballstrct[\strct{B}]{7^j, b}$ which justifies (5); this condition also justifies (6) because $\ball{7^j, \vect{b}} \subseteq \ball{7^{j + 1}, \vect{b}}$ which implies $\ballstrct[\ballstrct{7^{j + 1}, \vect{b}}]{7^j, \vect{b}} = \ballstrct[\strct{B}]{7^j, \vect{b}}$.
%%
\item In case 2.2, we also have
\[
\ballstrct{7^{j + 1}, \vect{b}} \satis \psi^j_{\vect{a}}[\vect{b}]
\]
because $\ballstrct{7^{j + 1}, \vect{a}} \satis \psi^j_{\vect{a}}[\vect{a}]$ (since $\ball{7^j, \vect{a}} \subseteq \ball{7^{j + 1}, \vect{a}}$ and hence $\ballstrct[{\ballstrct[\strct{A}]{7^{j + 1}, \vect{a}}}]{7^j, \vect{a}} = \ballstrct[\strct{A}]{7^j, \vect{a}}$) and (1);\footnote{The quantifier rank of $\psi^j_{\vect{a}}$ is smaller than $g(j + 1)$, by the third condition on the value of $g(j + 1)$ mentioned in case 2.1; although this condition is specified there, it is also applicable here because $g$ depends only on $\strct{A}$ and $\strct{B}$ but not on specific cases among 1, 2.1 or 2.2.} this implies
\[
(\ballstrct{7^j, \vect{a}}, \vect{a}) \equv[g(j)] (\ballstrct{7^j, \vect{b}}, \vect{b})
\]
since, similarly, $\ball{7^j, \vect{b}} \subseteq \ball{7^{j + 1}, \vect{b}}$ and hence $\ballstrct[\ballstrct{7^{j + 1}, \vect{b}}]{7^j, \vect{b}} = \ballstrct[\strct{B}]{7^j, \vect{b}}$.
%%
\end{enumerate}
%
\item \header{Hint to Exercise 2.5.3} Prove by induction a more general statement that every existential positive \emph{formula} is preserved under homomorphisms.
%
\item \header{Hint to Exercise 2.5.4} Use the fact that $\nat$ is well-founded.
%
\end{enumerate}
%end of section 5-----------------------------------------------------------------------------